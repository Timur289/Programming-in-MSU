\setcounter{object}{0}
\setcounter{approval}{0}
\setcounter{theorem}{0}
\setcounter{example}{0}

\chapter{Несобственные интегралы второго рода. Интеграл Дирихле.}
\centerline{ \bf Автор: Ромашкина Вероника} \vskip 1cm

Здесь мы сформулируем основные понятия элементарной теории несобственных параметрических интегралов второго рода и приведем формулировки некоторых утверждений, соответствующих доказанным нами теоремам об интегралах первого рода.
\smallskip
Рассмотрим множество $P = X \times Y$, где $ X=(a,b], Y\subset\mathbb R.$ Пусть функция $f(x,y)$ задана на $P$ и не ограничена как функция от $x$ хотя бы при одном фиксированном $y \in Y.$ Далее, пусть при любых $y \in Y$ и $\delta>0, \delta \in (0,b-a) $ функции $f(x,y)$ интегрируема по Риману  на отрезке $ [a+\delta,b]$ как функция от $x$.
\medskip

\begin{object}\itshape Введенное выше формальное выражение вида $ \int\limits_a^b f(x,y)dx$ называется \upshape \bfseries несобственным параметрическим интегралом второго рода \mdseries \itshape с одной особой точкой\upshape {} $x=a$.
\end{object}
\medskip

\begin{object}\itshape Если при любом фиксированном значении $y \in Y$ этот интеграл сходится, то множество $Y$ называется \upshape \bfseries областью сходимости интеграла \mdseries \itshape и его значения $g(y)= \int\limits_a^b f(x,y)dx$ порождают функцию, определенную на множестве $Y$.\upshape\end{object}
\medskip

Подобные определения имеют место и в случае, когда особая точка находится на правом конце промежутка интегрирования $X=[a,b]$, т.е. в точке $b$. В случае когда особая точка $x=x_0$ лежит внутри отрезка $X$ , его можно разбить на две части этой точкой $x_0$ и рассматривать каждую часть отрезка отдельно.\\
Аналогичные рассуждения позволяют рассматривать несобственные интегралы с переменной особой точкой $x_0=x_0(y)$, но здесь мы входить в детали не будем.
\medskip

\bfseries Пример. \mdseries Интеграл $ J=\int\limits_0^1 \frac{dx}{\sqrt{|x|}}$ сходится на $Y=\lbrack0,1\rbrack$ и его можно вычислить.

Действительно, имеем
$$ g(y)=g_1(y)+g_2(y)=\int\limits_0^y \frac{dx}{\sqrt{y-x}} + \int\limits_y^1\frac{dx}{\sqrt{x-y}}=2\sqrt{y}+2\sqrt{1-y}. $$

\begin{object} \itshape Несобственный  интеграл  второго рода
$$ g(y) = \int\limits_a^b f(x,y)dx$$
называется \upshape \bfseries равномерно сходящимся по $y$ на множестве $Y$ \mdseries \itshape, если для функции
$$ g(\delta,y)=\int\limits_{a+\delta}^b f( x,y)dx \quad \delta \to 0+$$
иыполнено условие \upshape
$$ g(\delta,y)\rightrightarrows  g(0,y)= g(y). $$
\end{object}

Исходя из общей формулировки критерия Коши можно сформулировать его для равномерной сходимости несобственного параметрического интеграла второго рода. Но мы ограничимся формулировкой одной сводной теоремы, содержащей утверждения, важные для практических применений.
\medskip

\begin{theorem} \itshape Пусть функция $f(x,y)$ непрерывна на $P = X\times Y,$ где $ X = (a,b], Y = [c,d]$. Пусть $a$ --- особая точка несобственного параметрического интеграла
$$ g(y) = \int\limits_a^b f(x,y)dx.$$
тогда справедливы следущие утверждения:

\bfseries 1. \mdseries Если интеграл $\int\limits_a^b f(x,y)dx$ сходится равномерно на $Y$, то функция $g(y)$ непрерывна при все $y \in Y$.

\bfseries 2. \mdseries В этом случае имеем
$$ \int\limits_c^d g(y)dy=\int\limits_a^b dx\int\limits_c^d f(x,y)dy.$$
\bfseries 3. \mdseries Если интеграл $\int\limits_a^b f(x,y)dx$ сходится,частная производная $f'_y(x,y)$ существует и непрерывна на $P,$ а интеграл $\int\limits_a^b f'_y(x,y)dx$ сходится равномерно на $Y$, то существует $g'(y)$, причем \upshape
$$ g'(y) = \int\limits_a^b f'_y(x,y)dx. $$
\end{theorem}

Если особая точка $x_0$ является внутреней точкой отрезка $X = \lbrack a,b\rbrack,$ то как было отмечено выше, необходимо отрезок $X$ разбить этой точкой на две части и рассматривать каждый из двух получившихся интегралов отдельно. Тот же подход можно применить и в случае, когда бесконечный промежуток интегрирования $X=\lbrack a,+\infty)$ содержит конечное число особых точек $x_1,\ldots,x_n.$ Тогда этот промежуток можно разбить на $2n$ промежутков точками $t_1<t_2<\cdots<t_{2n}$ таким образом, чтобы на каждомотрезке вида $[t_s,t_{s+1}],$ где $s=1,2,\ldots,2n-1,$ лежала бы ровно одна особая точка, а на промежутке $[t_{2n},+\infty)$ особых точек не блыо. В результате получим $2n-1$ несобственных интегралов второго рода и еще один - первого. 
\bigskip

Начнем с вычисления интеграла Дирихле $D(\alpha),$ называемого еще разрывным множителем Дирихле. По определению имеем
$$ D(\alpha)=\int\limits_0^\infty \frac{\sin \alpha x}{x}dx.$$
заметим прежде всего, что точка $x=0$ не является особой, так как подынтегральная
функция ограничена. Очевидно, что $D(0)=0$. Далее, если $\alpha > 0$, то интеграл сходится по признаку Дирихле, поскольку
% tuuuut
$$\left| \int\limits_0^t \sin\alpha xdx\right| =\left|\frac{1-\cos\alpha t}{\alpha}\right|<\frac2\alpha.$$
В этом случае возможна линейная замена переменной интегрирования вида $\alpha x=t$, тогда имеем
$$ D(\alpha) = \int\limits_0^\infty\frac{\sin\alpha x}{\alpha x}dx=\int\limits_0^\infty \frac{\sin t}{t}dt= D(1) = D.$$
Если же $\alpha<0,$ то $\alpha=-|\alpha|, \sin\alpha x=-\sin|\alpha| x,$ откуда
$$ D(\alpha) = \int\limits_0^\infty\frac{\sin\alpha x}{x}dx=-\int\limits_0^\infty \frac{\sin |\alpha| x}{x}dx=-D.$$
Таким образом имеем 
$$ D(\alpha)= 
\left\{\begin{array}{l} D \quad \mbox{при } \alpha>0,\\0 \quad \mbox{при } \alpha=0,\\
-D \quad \mbox{при } \alpha<0
\end{array}\right. $$
Теперь перейдем к вычислению значения $D$.
\medskip

\begin{theorem} \itshape Справедливо равенство $D = \pi/2.$\end{theorem}
\medskip

$\blacktriangleleft$ \upshape Рассмотрим параметрический интеграл $g(y),$ где $y\in Y = \lbrack0,N\rbrack, N\in \mathbb R$ и
$$ g(y) = \int\limits_0^\infty \frac{e^{-yx}\sin x}{x}dx.$$
Подынтегральная функция $f(x,y)=e^{-yx}\sin x/x$ будет непрерывна всюда на $ P = X\times Y,$ где $ X=[0,+\infty), Y=[0,N],$ если положить $f(0,y) = 1.$

Убедимся, что интеграл $g(y)$ сходится равномерно на $Y$. Для этого воспользуемся признаком Абеля. Положим $ \alpha(x,y)=\sin x/x, \beta(x,y)=e^{-yx}.$ Тогда функция $\beta(x,y)$ монотонна и $ 0<\beta(x,y)\le 1,$ а интеграл $\int\limits_0^\infty \alpha(x,y)dx $ сходится равномернона $Y$, поскольку $\alpha(x,y)$ не зависит от $y$.

Возьмем теперь на отрезке $Y$ произвольную точку $y_0\ne 0$ и окружим ее некоторым отрезком $ Y_\delta=\lbrack y_0-\delta,y_0+\delta\rbrack,$ целиком принадлежащим множеству$Y$. На этом отрезке интеграл
$$\int\limits_0^\infty f'_y(x,y)dx=-\int\limits_0^\infty e^{-yx}\sin xdx$$
сходится равномерно. Это следует из признака Вейерштрасса, поскольку $|e^{-xy}\sin x|< e^{-x(y_0-\delta)},$  а интеграл $ \int\limits_0^\infty e^{-x(y_0-\delta)}dx$ сходится. Кроме того, подынтегральная функция $e^{-xy}\sin x$ непрерывна на $ P_\delta = X\times Y_\delta.$ Поэтому по правилу Лейбница для несобственных интегралов имеем 
$$ g'(y) = -\int\limits_0^\infty e^{-yx}\sin xdx.$$
Последний интеграл можно вычислить путем интегрирования по частям. При этом получим
$$ g'(y)=-\frac{1}{1+y^2}.$$
Итак, мы показали, что функция $g(y)$ непрерывна на $ Y = \lbrack 0,N\rbrack,$ а ее производная существует при всех $y\ne 0$. Отсюда по формуле Ньютона-Лейбница при всех $y\in(0,N]$ вытекает равенство
$$ g(y) = g(N) - \int\limits_N^y \frac{dt}{1+t^2}=g(N)+\arctg N+\arctg y.$$
Пользуясь непрерывностью функции $g(y)$ в точке $y=0$, мы получим
$$ g(0)=\lim_{y\to 0+}(g(N)+ \arctg N-\arctg y)=g(N)+\arctg N.$$
Тепреь, устремляя $N$ к $+\infty$ приходим к соотношениям $\arctg N \to \pi/2$,
$$ |g(N)|\le\int\limits_0^\infty e^{-Nx}\frac{|\sin x|}{x}dx\le \int\limits_0^\infty e^{-Nx}dx=\frac1N\to 0.$$
Отсюда следует, что
$$ D=g(0)=\lim_{N\to\infty}(g(N)+\arctg N)=\pi/2. \blacktriangleright$$

