\documentclass[12pt,titlepage]{report}
\usepackage[russian]{babel}
\usepackage{amsmath}
\usepackage{amssymb}
\usepackage{amsthm}
\usepackage{textcomp}
\usepackage[utf8]{inputenc}
\usepackage[koi8-r]{inputenc}
%\usepackage[dvips]{graphics,epsfig}
%\usepackage{weird,queer,latexsym}
%\usepackage{latexsym} 
\pagestyle{empty}
\frenchspacing
\textwidth=16cm
\textheight=25cm
\topmargin=-0.5in
\oddsidemargin=0mm
\newcommand\ds{\displaystyle}
\newtheorem{object}{Определение}
\newtheorem{approval}{Утверждение}
\newtheorem{theorem}{Теорема}
\newtheorem{example}{Пример}
\newtheorem{lemma}{Лемма}
\newtheorem{task}{Задача}
\newtheorem{sledstvie}{Следствие}
\addto\captionsrussian{\renewcommand\chaptername{Лекция}}
\def\R{\mathbb R}
\def\N{\mathbb N}
\def\dx{{\rm d}x}
\def\dxy{{\rm d}xy}

\begin{document}
\pagestyle{plain}
\title{\textbf{ЛЕКЦИИ \\ПО МАТЕМАТИЧЕСКОМУ \\АНАЛИЗУ}}
\author{\textbf{Архипов Г.И., Садовничий В.А., Чубариков В.Н.}}
\date{февраль, 2012}
\maketitle
\newcounter{number}[chapter]
\chapter{Признаки сходимости числовых рядов с неотрицательными членами}
\centerline{ \bf Автор: Велиев Али} \vskip 1cm

\begin{object}
Ряд $\sum{a_n}$ называется \textbf{рядом с неотрицательными членами}, если $\forall\ n$ имеем $a_n\ge0$
\end{object}


\begin{object}
Пусть $\sum{a_n}$ и $\sum{b_n}$ - два ряда с неотрицательными членами, и пусть начиная с некоторого $n_0 \forall\ n>n_0$ имеем $0\le{a_n}\le{b_n}$. Тогда ряд $\sum{b_n}$ называется \textbf{мажорантой} ряда $\sum{a_n}$, а ряд $\sum{a_n}$ - \textbf{минорантой} ряда $\sum{b_n}$.
\end{object}


\begin{theorem}
\label{aw}
Пусть $s_n$ - последовательность частичных сумм ряда $\sum{a_n}$ с неотрицательными членами. $\sum{a_n}$ сходится $\Leftrightarrow$ последовательность $s_n$ ограничена.
\end{theorem}
$\blacktriangleleft$ ($\Rightarrow$) Вытекает из определения суммы ряда.
\par ($\Leftarrow$) Поскольку ряд $\sum{a_n}$ с неотрицательными членами, то последовательность частичных сумм не убывает и неотрицательна. Последовательность $s_n$ монотонна $\Rightarrow$ можно применить теорему Вейерштрасса о сходимости  монотонной последовательности. Далее сходимость последовательности $s_n$, по определению суммы ряда, влечёт за собой сходимость ряда $\sum{a_n}$ $\blacktriangleright$


\begin{theorem}(признак сравнения)
Пусть $\sum{a_n}$ и $\sum{b_n}$ миноранта и мажоранта соответственно. Тогда справедливо:
\begin{enumerate}
\item Сходимость мажоранты ряда влечёт за собой сходимость миноранты ряда.
\item Расходимость миноранты ряда влечёт за собой расходимость мажоранты ряда.
\end{enumerate}
\end{theorem}
$\blacktriangleleft$ Обозначим частичные суммы рядов $\sum{a_n}$ и $\sum{b_n}$ как ${t_n}$ и ${s_n}$ соответственно.
\begin{enumerate}
\item $s_n$ ограничена в виду сходимости ряда $b_n\,\ a_n\le{b_n}\ \Rightarrow\ t_n\le{s_n}$. Но это означает, что $t_n$ ограничена и ряд $\sum{a_n}$ сходится.
\item В этом случае $t_n\to{\infty}$ $\Rightarrow$ $s_n\to{\infty}$ и ряд $\sum{b_n}$ расходится. $\blacktriangleright$
\end{enumerate}


\textit{Замечание}. Также вместо условия в теореме 2 можно рассматривать вместо условия $a_n\le{b_n}$ условие $\frac{a_{n+1}}{a_n} \le \frac{b_{n+1}}{b_n}$.


\begin{theorem}(признак Даламбера)
Пусть для членов ряда $\sum{a_n}$ начиная с некоторого $n_0$ члена выполнены условия:
\begin{enumerate}
\item $a_n>0$;
\item $D_n=\frac{a_{n+1}}{a_n}\le{q}$, где $0<q<1$.
\end{enumerate}
Тогда ряд $\sum{a_n}$ сходится. Если же имеем, что $\frac{a_{n+1}}{a_n}\ge1$, то ряд $\sum{a_n}$ расходится.
\end{theorem}
$\blacktriangleleft$Сравним ряд $\sum{a_n}$ со сходящимся рядом $\sum{b_n}$, где $b_n=q^n$. Тогда $\forall\ n>n_0$ имеем:
$$\frac{a_{n+1}}{a_n}\le q=\frac{q^{n+1}}{q^n}=\frac{b_{n+1}}{b_n} $$
Ряд $\sum{q^n}$ сходится при $q<1$ и расходится при $q\ge 1$. Следовательно, действительно имеем сходимость и расходимость из признака сравнения. $\blacktriangleright$


\begin{theorem}(признак Коши)
Пусть для членов ряда $\sum{a_n}$ начиная с некоторого $n_0$ члена выполнены условия:
\begin{enumerate}
\item $a_n>0$;
\item $\sqrt[n]{a_n}\le q$, где $q<1$.
\end{enumerate}
Тогда ряд $\sum{a_n}$ сходится. Если же имеем, что $\sqrt[n]{a_n}\ge1$, то ряд $\sum{a_n}$ расходится.
\end{theorem}
$\blacktriangleleft$
Имеем $\sqrt[n]{a_n}\le q \Rightarrow a_n\le q^n$. Опять же будем считать, что $b_n=q^n$ и по признаку сравнения имеем справедливость данного признака в обеих случаях.
$\blacktriangleright$


\begin{theorem}(признак Раабе)
\begin{enumerate}
\item Ряд $\sum{a_n}$ сходится, если $\forall\ n>n_0$ $\exists\ \alpha>1$, что имеет место неравенство
$$\frac{a_{n+1}}{a_n}\le1-\frac{\alpha}{n}$$ 
\item Этот же ряд расходится, если начиная с некоторого $n_1$ выполнено неравенство
$$\frac{a_{n+1}}{a_n}\ge1-\frac{1}{n}$$
\end{enumerate}
\end{theorem}
$\blacktriangleleft$ Проведём наше доказательство в два этапа:
\begin{enumerate}
\item Для доказательства опять же воспользуемся признаком сравнения. Сравним ряд $\sum{a_n}$ с рядом $\sum{b_n}$, где $b_n=\frac{1}{n^\beta}$ и $\beta=\frac{\alpha+1}{2}$ с условием $\alpha>\beta>1$. Этот ряд сходится. Тогда при $n\to\infty$ имеем
$$\frac{b_{n+1}}{b_n}=\left(\frac{n+1}{n}\right)^{-\beta}=1-\frac{\beta}{n}+O\left(\frac{1}{n^2}\right)>1-\frac{\alpha}{n}\ge\frac{a_{n+1}}{a_n}$$
Ряд $\sum{a_n}$ мажорируется сходящимся рядом $\Rightarrow$ он сходится.
\item В этом случае перепишем неравенство и будем считать, что $n\ge2$
$$\frac{a_{n+1}}{a_n}\ge1-\frac{1}{n}=\frac{n-1}{n}=\frac{\frac{1}{n}}{\frac{1}{n-1}}=\frac{b_{n+1}}{b_n},\ b_n=\frac{1}{n-1}$$
Но ряд $\sum{b_n}$ - это гармонический ряд, который расходится, и по признаку сравнения имеем, что ряд $\sum{a_n}$ действительно расходится. $\blacktriangleright$
\end{enumerate}


\begin{theorem}(признак Куммера)
Пусть \{$a_n$\} и \{$b_n$\} две числовые последовательности положительных чисел.
\begin{enumerate}
\item Если $\exists\ \alpha>0$ и номер $n_0$ такие, что $\forall\ n>n_0$ выполнено неравенство
\begin{equation}
\label{1}
c_n-c_{n+1}\frac{a_{n+1}}{a_n}\ge\alpha
\end{equation}
то ряд $\sum{a_n}$ сходится.
\item Если $\exists\ n_0$ такое, что $\forall\ n>n_0$ выполнено неравенство
\begin{equation}
\label{2}
c_n-c_{n+1}\frac{a_{n+1}}{a_n}\le0
\end{equation}
и ряд $\sum{\frac{1}{c_n}}$ расходится, то ряд $\sum{a_n}$ тоже расходится.
\end{enumerate}
\end{theorem}
$\blacktriangleleft$ Докажем каждый пункт признака куммера по отдельности.
\begin{enumerate}
\item Умножим неравенство~\ref{1} на $a_n$, имеем
$${c_n}{a_n}-{c_{n+1}}{a_{n+1}}\ge\alpha{a_n}$$
Далее просуммируем это неравенство по $n$ от $n=1,...,m$
$$\sum_{n=1}^m{c_n{a_n}}-\sum_{n=1}^m{c_{n+1}{a_{n+1}}}\ge\alpha\left(a_1+...+a_m\right)$$
Поскольку
$$\sum_{n=1}^m{c_n{a_n}}-\sum_{n=1}^m{c_{n+1}{a_{n+1}}}={c_1}{a_1}+c_2{a_2}+...+c_m{a_m}-$$ $$-c_2{a_2}-...-c_{m+1}a_{m+1}=c_1{a_1}-c_{m+1}a_{m+1}$$
получаем
$$c_1{a_1}-c_{m+1}a_{m+1}\ge\alpha\left(a_1+...+a_m\right)$$
откуда имеем
$$s_m=a_1+...+a_m\le\frac{c_1{a_1}-c_{m+1}a_{m+1}}{\alpha}<\frac{c_1{a_1}}{\alpha}$$
Поэтому согласно теореме \ref{aw} частичные суммы ряда $\sum{a_n}$ ограничены, что означает сходимость этого ряда.
\item В этом случае неравенство~(\ref{2}) можно переписать неравенство в виде
$$-c_{n+1}\frac{a_{n+1}}{a_n}\le-c_n;\ c_{n+1}\frac{a_{n+1}}{a_n}\ge c_n;\ \frac{a_{n+1}}{a_n}\ge\frac{\frac{1}{c_{n+1}}}{\frac{1}{c_n}}$$
Но ряд $\sum{c_n}$ расходится, что по признаку сравнения влечёт расходимость мажорирующего ряда $\sum{a_n}$. $\blacktriangleright$
\end{enumerate}


\begin{theorem}(признак Бертрана)
\begin{enumerate}
\item Ряд $\sum{a_n}$ сходится, если $\exists\ \alpha>0$ и номер $n_0$ такие, что $\forall\ n>n_0$ выполнены неравенства
\begin{equation}
\label{3}
\frac{a_{n+1}}{a_n}\le1-\frac{1}{n}-\frac{1+\alpha}{n\ln{n}}
\end{equation} 
\item Данный ряд расходится, если при всех достаточно больших $n$ имеет место формула
\begin{equation}
\label{4}
\frac{a_{n+1}}{a_n}\ge1-\frac{1}{n}-\frac{1}{n\ln{n}}
\end{equation}
\end{enumerate}
\end{theorem}
$\blacktriangleleft$ Эта теорема по сути является следствием признака Куммера.
\begin{enumerate}
\item В качестве $c_n$ в признаке Куммера положим $c_n=(n-1)\ln{n-1}$. Тогда формула (\ref{3}) записывается в виде
$$(n-1)\ln{n-1}-n\ln{n}\frac{a_{n+1}}{a_n}\ge\alpha$$
$$-n\ln{n}\frac{a_{n+1}}{a_n}\ge\alpha-(n-1)\ln{n-1}$$
Преобразуем правую часть полученного неравенства.
\vskip 5mm
$\frac{a_{n+1}}{a_n}\le\frac{-\alpha}{n\ln{n}}+\frac{(n-1)\ln{(n-1)}}{n\ln{n}}=\frac{-\alpha}{n\ln{n}}+\frac{(n-1)\ln{(n-1)}}{n\ln{n}}+\frac{n-1}{n}-\frac{n-1}{n}=\frac{-\alpha}{n\ln{n}}+\frac{(n-1)\ln{(n-1)}}{n\ln{n}}+\frac{n-1}{n}-\frac{\left(n-1)\right)\ln{n}}{n\ln{n}}=1-\frac{1}{n}+\frac{\left(n-1\right)\left(\ln{(n-1)}+\ln{n}\right)}{n\ln{n}}-\frac{\alpha}{n\ln{n}}=1-\frac{1}{n}-\frac{\alpha}{n\ln{n}}+\frac{\left(n-1\right)\ln{\frac{n-1}{n}}}{n\ln{n}}$
\vskip 5mm
Поскольку $(n-1){\ln{(1-\frac{1}{n})}=\ln{(1-\frac{1}{n})}}^{n-1}>-1$ предыдущее неравенство вытикает из следующего
$$\frac{a_{n+1}}{a_n}\le1-\frac{1}{n}-\frac{1+\alpha}{n\ln{n}}\le1-\frac{1}{n}-\frac{(n-1)\ln{(1-\frac{1}{n})}}{n\ln{n}}-\frac{\alpha}{n\ln{n}}$$
то есть условие сходимости в признаке Бертрана обеспечивает справедливость условия сходимости в признаке Куммера.
\item Положим в признаке Куммера $c_n=(n-2)\ln{n-1}$. Тогда имеем
$$(n-2)\ln{n}\frac{a_{n+1}}{a_n}-(n-2)\ln{(n-1)}\ge0$$
Откуда получаем
$$(\frac{a_{n+1}}{a_n}\ge\frac{(n-2)ln{(n-1)}}{(n-1)\ln{n}}$$
Теперь оценим выличину данную в~(\ref{4})
$$1-\frac{1}{n}-\frac{1}{n\ln{n}}\ge\left(1-\frac{1}{n-1}\right)\left(1-\frac{1}{n\ln{n}}\right)\ge\left(1-\frac{1}{n-1}\right)\left(1+\frac{ln{\left(1+\frac{1}{n}\right)}}{\ln{n}}\right)$$
Проведя соответствующие вычисления легко проверить
$$\left(1-\frac{1}{n-1}\right)\left(1+\frac{ln{\left(1+\frac{1}{n}\right)}}{\ln{n}}\right)=\frac{(n-2)ln{(n-1)}}{(n-1)\ln{n}}$$
Что полностью соответствует правильности признака Бертрана для расходимости ряда.$\blacktriangleright$
\end{enumerate}


\begin{theorem}(интегральный признак Коши-Маклорена)
Пусть функция $f(x)$ определена на промежутке $[1;+\infty)$ и убывает на нём. Тогда:
\begin{enumerate}
\item Если $0\le{a_n}\le{f(n)}\ \forall\ n>n_0$ и несобственный интеграл $\int_1^{\infty}f(x)dx$ сходится, то ряд $\sum{a_n}$ сходится.
\item Если $0\le{f(n)}\le{a_n}\ \forall\ n>n_0$ и несобственный интеграл $\int_1^{\infty}f(x)\dx$ расходится, то и ряд $\sum{a_n}$ расходится.
\end{enumerate}
\end{theorem}
$\blacktriangleleft$ Без ограничения общности будем считать, что $n_0=1$. Поскольку $f(x)$ монотонно убывает $\forall\ k$ и $k\le{x}\le{k+1}$ имеем
$$f(k)\ge{f(x)}\ge{f(k+1)}$$
Интегрируем это неравенство по отрезку $[k,k+1]$ получаем
$$f(k)=\int_k^{k+1}f(k)\dx\ge\int_k^{k+1}f(x)\dx\ge\int_k^{k+1}f(k+1)\dx=f(k+1)$$
Суммируем эти неравенства по $k$ от 1 до $n-1$. Получим:
$$s_{n-1}=\sum_{k=1}^{n-1}f(k)\ge\int_1^n{f(x)}\dx\sum_1^{n-1}f(k+1)=\left(\sum_{k=1}^n{f(k)}\right)-f(1)=s_n-f(1)$$
Далее рассмотрим отдельно 2 случая:
\begin{enumerate}
\item Несобственный интеграл $I=\int_1^{\infty}f(x)dx$ и $\forall\ n>2$ для частичных сумм $s_n$ ряда $f(n)$ имеет место единообразная оценка вида $s_n\le{I}+f(1)$, что по теореме 1 доказывает справедливость неравенства.
\item Здесь же $s_n\ge\int_1^n{f(x)}\dx$, то $s_n$ неограничена, и ряд расходится.$\blacktriangleright$
\end{enumerate}


\setcounter{object}{0}
\setcounter{approval}{0}
\setcounter{theorem}{0}
\setcounter{example}{0}

\chapter{Бесконечные произведения}
\centerline{ \bf Автор: Мустафазаде Аруз} \vskip 1cm
\begin{object}
\slshape{Рассмотрим числовую последовательность положительных чисел \{$b_n$\}. Формальное бесконечное произведение всех её членов}
\begin{center}
$b_{1}\cdot b_{2} \cdot b_{3} \cdots b_{n}  \cdots $.
\end{center}
\slshape{называется}\/
\upshape\mdseries\rmfamily{\bfseries{бесконечным числовым произведением}}, \slshape{или} 
\upshape\mdseries\rmfamily{\bfseries{бесконечным произведением}},
\slshape{или просто} 
\upshape\mdseries\rmfamily\normalsize{\bfseries{произведением}}.

Бесконечное произведение обозначается так:
\begin{center}
$b_{1} \cdot b_{2} \cdots = \prod\limits_{n=1}^{\infty}b_{n} = \prod\limits b_{n}$.
\end{center}
\end{object}

\begin{object}
\slshape{Конечное произведение} $\prod\nolimits_{n}$ \slshape{вида} $\prod\nolimits_{n} = b_{1} \cdots b_{n}$
\slshape{называется} \upshape\mdseries\rmfamily{\bfseries{n-м частичным произведением}}.
\end{object}

\begin{object}
\slshape{Если последовательность} $\prod\nolimits_{n}$ \slshape{сходится к числу} $\prod \ne{0}$
(\slshape{т.е.} $\prod>0$), \slshape{то бесконечное произведение называется \upshape\mdseries\rmfamily\textbf{сходящимся} (к числу $\prod$)}.
\slshape{Если $\prod = 0$, то это бесконечное произведение называется \upshape\mdseries\rmfamily\textbf{расходящимся к нулю}, а если 
$\prod\to{+\infty}$, то оно называется \textbf{расходящимся к бесконечности}. Если предела нет вообще, то оно называется просто \upshape\mdseries\rmfamily\textbf{расходящимся}}.
\end{object}


\begin{approval}
(необходимый признак сходимости бесконечногo произведения).
\slshape{Если $\prod b_{n}\mbox{ сходится, то }b_{n}\to 1\mbox{ при }n\to \infty$}.
\end{approval}
$\blacktriangleleft$ \upshape\mdseries\rmfamily{Если при} $\prod\nolimits_{n}\to \prod \ne{0}$, 
\upshape\mdseries\rmfamily{то}
\begin{center}
$b_{n} = \dfrac{\prod\nolimits_{n}}{\prod\nolimits_{n-1}}\to \dfrac{\prod}{\prod} = 1$ \upshape\mdseries\rmfamily{при} $n\to \infty$ $\blacktriangleright$.
\end{center}
\upshape\mdseries\rmfamily


\begin{approval}
\slshape{Сходимость бесконечного произведения $\prod b_{n}$ влечёт за собой сходимость ряда $\ln{b_{n}}$, и 
наоборот, причём}
\begin{center}
$\ln{\prod\limits_{n=1}^{\infty}b_{n} = \sum{\ln{b_{n}}}}$.
\end{center}
$\blacktriangleleft$
\upshape\mdseries\rmfamily{Имеем $\ln{\prod\nolimits_{n}} = \sum\limits_{k=1}^{n}\ln{b_{k}}$. Функция $y = \ln{x}$
устанавливает непрерывное взаимно однозначное соответствие между лучом $(0, +\infty)$ и всей вещественной осью
$\mathbb R = (-\infty, +\infty)$. Поэтому в силу положительности $b_{n}$ для всех $n\in \mathbb N$ возможен переход к пределу
в одной части равенства при сходимости другой его части, 
и при этом $\ln{\prod} = \sum\nolimits_{k=1}^{\infty}\ln{b_{k}}$. Сходимость к нулю левой части равенства 
эквивалентна сходимости к $-\infty$ правой его части. $\blacktriangleright$}.
\end{approval}

\textit{Замечание}. Очевидно, что отбрасывание или добавление любого конечного числа ненулевых сомножителей не влияет на сходимость бесконечного произведения. Поэтому можно считать, что конечное число членов этого произведения могут быть и отрицательными.

\begin{object}
\slshape{Бесконечное произведение $\prod\limits_{k=1}^{\infty}b_{k}$ называется} 
\upshape\mdseries\rmfamily\textbf{абсолютно сходящимся},
\slshape{если абсолютно сходится ряд $\sum{\ln{b_{k}}}$. Это означает сходимость ряда \\$\sum|\ln{b_n}|$.
Сходящееся бесконечное произведение $\prod\limits_{n=1}^{\infty}b_{n}$, не являющееся абсолютно сходящимся,
называется \upshape\mdseries\rmfamily\textbf{условно сходящимя}}.

\upshape\mdseries\rmfamily{Из предыдущего утверждения и теоремы о сходимости 
абсолютно сходящегося ряда непосредственно вытекает следующая
теорема}.
\end{object}

\begin{theorem}
\slshape{Абсолютно сходящееся произведение всегда сходится в обычном смысле}.

\upshape\mdseries\rmfamily{Поскольку мы считаем, что $b_{n}>0$ при всех n, числа $b_{n}$ 
обычно представляют в виде $b_n = 1 + a_n$, где $a_n> -1$. Тогда имеем} 
\begin{center}
$\prod\limits_{n=1}^{\infty}b_{n} = \prod\limits_{n=1}^{\infty}(1 + a_{n})$.
\end{center}
\end{theorem}

\begin{theorem}
\label{se}
(критерий абсолютной сходимости бесконечного произведения). \slshape{Бесконечное произведение 
$\prod\limits_{n=1}^{\infty}(1 + a_{n})$ абсолютно сходится тогда и только тогда, когда сходится ряд $\sum|a_n|$.}
\end{theorem}

$\blacktriangleleft$ \upshape{Так как $1 + a_n\to 1\mbox{ при }n\to\infty$, то $a_n\to 0$. Однако}
\begin{center}
$\dfrac{\ln(1 + x)}{x}\to 1\mbox{ при }x\to 0$,
\end{center}
поэтому при $n\to\infty$ имеем
\begin{center}
$\dfrac{\ln(1 + a_n)}{a_n}\to 1$, $\dfrac{|\ln(1 + a_n)|}{|a_n|}\to 1$.
\end{center}
Следовательно, при достаточно большом $n>n_0$ выполнены неравенства 
\begin{center}
$\dfrac12<\dfrac{|\ln(1 + a_n)|}{|a_n|}<\dfrac23$.
\end{center}
Если, например, сходится ряд $\sum|\ln(1 + a_n)|$, то он будет мажорантой для ряда $\sum|a_n|/2$, а если сходится ряд 
$\sum|a_n|$, то он является мажорантой для ряда $\sum 2|\ln(1 + a_n)|/3$. Но это означает, что ряды $\sum|a_n|$ и 
$\sum|\ln(1 + a_n)|$ сходятся и расходятся одновременно.$\blacktriangleright$


Следствием из этой теоремы является следующее утверждение.

\begin{approval}
\slshape{Если при достаточно большом $n>n_0$ все числа $a_n$ имеют один и тот же знак, то сходимость произведения
$\prod(1 + a_n)$ эквивалентна сходимости ряда $\sum a_n$}.
\end{approval}
$\blacktriangleleft$
\upshape\mdseries\rmfamily{Поскольку и сходимость ряда, и сходимость произведения влечёт за собой соотношения
\begin{center}
$a_n\to 0$, $\ln(1 + a_n)\to 0$, $\dfrac{\ln(1 + a_n)}{a_n}\to 1\mbox{ при }n\to\infty$, 
\end{center}
отсюда следует, что при достаточно большом $n>n_0$ величина $\ln(1 + a_n)$ сохраняет знак вместе с $a_n$. Это означает,
что сходимость рядов $\sum a_n$, $\sum\ln(1 + a_n)$ и произведения $\prod(1 + a_n)$ эквивалентна их абсолютной сходимости.
Теперь, применяя Теорему \ref{se}, получаем требуемое утверждение.} 
$\blacktriangleright$

Рассмотрим некоторые примеры бесконечных произведений.


\begin{example}
Гамма-функция Эйлера $\Gamma(s)$. По определению имеем
\begin{center}
$\Gamma(s) = \dfrac{1}{s e^{\gamma s}}\prod\limits_{n=1}^{\infty}\left(1 + \dfrac sn\right)^{-1}e^{s/n}$,
\end{center}
где $s\ne 0, -1, -2, \ldots$ --- любое вещественное число 
(или даже комплексное число, если определение 3
распространить на комплексные числа), $\gamma$ --- постоянная Эйлера,
\begin{center}
$\gamma = \lim\limits_{n\to\infty}(1 + \dfrac 12 + \cdots + \dfrac 1n - \ln n) = 0,577 \ldots$ .
\end{center}

Бесконечное произведение, через которое определяется гамма-функция Эйлера, сходится абсолютно при любом 
$s\ne 0, -1, -2, \ldots$, так как при достаточно большом $n>n_0$ в силу формулы Тейлора с остаточным членом в форме
Лагранжа справедлива оценка
\begin{center}
$|\ln b_n| = \left|\ln\left(\left(1 + \dfrac sn\right)^{-1}e^{s/n}\right)\right|$ = 
$\left|\dfrac sn - \ln\left(1 + \dfrac sn\right)\right| < \dfrac{s^2}{n^2}$,
\end{center}
и сходящийся ряд $\sum{s^2}/{n^2}$ является мажорантой для ряда $\sum|\ln b_n|$.
\end{example}

\begin{approval}
(Формула Эйлера) \slshape{Имеет место следующая формула:}
\begin{center}
$\Gamma(s) = s^{-1}\prod\nolimits_{n=1}^\infty(1 + 1/n)^s(1 + s/n)^{-1}$.
\end{center}
\end{approval}
$\blacktriangleleft$ \upshape\mdseries\rmfamily{Уже доказано, что бесконечное произведение в определении гамма-функции сходится абсолютно в любой точке своей области определения. Поэтому из определения гамма-функции имеем}
\begin{center}
$\Gamma(s) = s^{-1}\lim\limits_{m\to\infty}e^{-s(1 + 1/2 + \cdots + 1/m -\ln m)}\lim\limits_{m\to\infty}$
$\prod\limits_{n=1}^{\infty}\left(1 + \dfrac sn\right)^{-1}e^{s/n} =$\\
$ = s^{-1}\lim\limits_{m\to\infty}m^s$
$\prod\limits_{n=1}^{m}\left(1 + \dfrac sn\right)^{-1} = $
$s^{-1}\lim\limits_{m\to\infty}\prod\limits_{n=1}^{m-1}\left(1 + \dfrac{1}{n}\right)^{s}$
$\prod\limits_{n=1}^{m}\left(1 + \dfrac{s}{n}\right)^{-1} =$\\
$=s^{-1}\lim\limits_{m\to\infty}\left(\prod\limits_{n=1}^{m}\left(1 + \dfrac 1n\right)^{s}\left(1 + \dfrac sn\right)^{-1}\right)$
$\left(1 + \dfrac 1m\right)^{-s}=$
$s^{-1}\prod\limits_{n=1}^{\infty}\left(1 + \dfrac 1n\right)^s\left(1 + \dfrac sn\right)^{-1}$,
\end{center}
что и требовалось доказать. $\blacktriangleright$


\begin{approval}
(Функциональное уравнение для гамма-функции Эйлера $\Gamma(s)$). \slshape{Справедлива следующая формула:}
\begin{center}
$\Gamma(s + 1) = s\Gamma(s)$,  $\Gamma(1) = 1$.
\end{center}
\end{approval}

$\blacktriangleleft$\upshape\mdseries\rmfamily{По формуле Эйлера имеем, что $\Gamma(1) = 1$, а также}
\begin{center}
$\dfrac{\Gamma(s + 1)}{\Gamma(s)} = \dfrac{s}{s + 1}\prod\limits_{n=1}^{\infty}$
$\dfrac{(1 + 1/n)^{s + 1}}{(1 + 1/n)^{s}}\dfrac{1 + s/n}{1 + (s + 1)/n}=$
$\dfrac{s}{s + 1}\prod\limits_{n=1}^{\infty}\dfrac{n + 1}{n}\dfrac{n + s}{n + s + 1}=$\\
$=\dfrac{s}{s + 1}\lim\limits_{m\to\infty}\prod\limits_{n=1}^{m}$
$\dfrac{2\cdot 3\ldots(m + 1)}{1\cdot 2\ldots m}\dfrac{(1 + s)\ldots(1 + m)}{(2 + s)\ldots(m + 1 + s)} =$

$=\dfrac{s}{s + 1}\lim\limits_{m\to\infty}(s + 1)\dfrac{m + 1}{m + 1 + s} = s$.
\end{center}
Отсюда следует, что $\Gamma(s + 1) = s\Gamma(s)$. $\blacktriangleright$

Из утверждения 5 непосредственно получаем такое следствие.


\textbf{Следствие}. \slshape{Для натуральных чисел $n$ имеем $\Gamma(n + 1) = n!$ }
\upshape\mdseries\rmfamily{Далее будет доказано, что при $s>0$ имеет место формула интегрального представления 
для $\Gamma(s)$ вида}
\begin{center}
$\Gamma(s) = \int\limits_{0}^{\infty} x^{s - 1} e^{-x}dx$.
\end{center} 

\begin{example}
При всех вещественных $x$ следующее бесконечное произведение сходится:
\begin{center}
$\prod\limits_{n=1}^{\infty}\left(1 - \dfrac{x^2}{\pi^2 n^2}\right) = \dfrac{\sin x}{x}$.
\end{center}
\end{example}

Это равенство мы докажем позднее, а сходимость вытекает из утверждения 3.

\begin{example}
Бесконечное произведение для дзета-функции Римана. \\ При $s>1$ фукция $\zeta(s)$ определена сходящимся рядом 
$\zeta(s) = \sum\limits_{n=1}^{\infty}\dfrac{1}{n^s}$. Пусть при $p_1 = 2, p_2 = 3, p_3 = 5, \ldots$ --- последовательно
занумерованные простые числа натурального ряда.
\end{example}

\begin{approval}
(формула Эйлера бесконечного произведения дзета-функции Римана $\zeta(s)$). \slshape{При $s>1$ имеет место
следующая формула}:
\begin{center}
$\zeta(s) = \sum\limits_{n=1}^{\infty}\dfrac{1}{n^s} =$ 
$\prod\limits_{k=1}^{\infty}\left(1 - \dfrac{1}{p_k^s}\right)^{-1}$.
\end{center}
\end{approval}

$\blacktriangleleft$ \upshape\mdseries\rmfamily {Имеем } 
\begin{center}
$\prod_k = \prod\limits_{m=1}^{k}\left(1 - \dfrac{1}{p_m^s}\right)^{-1} =$ 
$\prod\limits_{m=1}^{k}\left(1 + \dfrac{1}{p_m^s} + \dfrac{1}{p_m^2s} + \ldots\right)$.
\end{center}
Раскрывая скобки, согласно неравенству $p_k>k$ , справедливому при $k\in\mathbb N$, получим
\begin{center}
$\prod_k>\sum\limits_{n=1}^{k}\dfrac{1}{n^s}$.
\end{center}
С другой стороны, очевидно, что 
\begin{center}
$\prod_k = \sum\limits_{m=1}^{\infty}\dfrac{1}{a_m^s}$,
\end{center}
где $a_m -$ некоторая подпоследовательнось натуральных чисел, которая не содержит повторений в силу однозначности 
разложения натурального числа на простые сомножители. Отсюда имеем неравенства
\begin{center}
$\zeta(s) = \sum\limits_{n=1}^{\infty}\dfrac{1}{n^s}>\sum\limits_{m=1}^{\infty}\dfrac{1}{a_m^s}>\prod_k = $
$\sum\limits_{n=1}^{k}\dfrac{1}{n^s}$.
\end{center}
Переходя здесь к пределу при $k\to\infty$, получаем требуемый результат.$\blacktriangleright$ 

При $s = 1$ справедлива оценка
\begin{center}
$\prod_k = \prod\limits_{m=1}^{k}\left(1 + \dfrac{1}{p_m} + \dfrac{1}{p_m^2} + \ldots\right)>$
$1 + \dfrac 12 + \cdots + \dfrac 1k$,
\end{center}
поэтому произведение $\prod\limits_{k=1}^{\infty}\left(1 - \dfrac{1}{p_k}\right)^{-1}\mbox{ расходится к }+\infty$,
вместе с ним расходятся и ряды $-\sum\limits_{k=1}^{\infty}\ln\left(1 - \dfrac{1}{p_k}\right)\mbox{ и }$
$\sum\limits_{k=1}^{\infty}\dfrac{1}{p_k}$.


\setcounter{object}{0}
\setcounter{approval}{0}
\setcounter{theorem}{0}
\setcounter{example}{0}

\chapter{Сходимость функционального ряда}
\centerline{ \bf Автор: Алиева Лейла} \vskip 1cm

\section{Сходимость функционального ряда}
\ Понятия функциональной последовательности и функционального ряда связаны между собойтак же тесно, как и в обычном числовом случае. С этими понятиями мы, по существу, уже ранее встречались. Примерами могут служить бесконечная геометрическая прогрессия
$$\sum^{\infty}_{n=1}q^n=\frac{q}{1-q}\quad\mbox{при}\quad |q|<1 $$
или дзета-функция Римана 
$$\zeta(s)=\sum^{\infty}_{n=1}\frac{1}{n^s}\quad\mbox{при}\quad s>1.$$
Если в первом случае зафиксировать $q$, а во втором $s$, то мы получим обычные числовые ряды. Но эти же параматры можно рассматривать как аргументы числовых функций, и тогда суммы рядов тоже будут представлять собойнекоторые числовые функции. Подобные соображения приводят нас к следующим определениям.\smallskip

\begin{object} Функциональной последовательностью называется занумерованное множество функций $\{ f_n(x) \}$, имеющих одну и ту же область определения $D\subset \R$. При этом множество $D$ называется областью определения функциональной послдовательности $\{ f_n(x) \}$.\end{object}\smallskip

Здесь термин "занумеровать" значит "поставить во взаимо однозначное соответствие с натуральным рядом $\mathbb{N}$".

\begin{object} Пусть $\{a_n(x)\}$ -- некоторая функциональная последовательность (ф.п.), определенная на множестве $D$. формальная бесконечная сумма вида $$ a_1(x)+a_2(x)+a_3(x)+\ldots=\sum^{\infty}_{n=1}a_n(x),$$ или просто $\Sigma a_n(x)$, называется \textbf{функциональным рядом,} определенным на $D$.\end{object}\smallskip

Фиксируя какое-либо значение $x=x_0\in D$, получаем обычный числовой ряд $\Sigma a_n(x_0)$. Как и в числовом случае, определим понятиечастичной суммы функционального ряда.\smallskip

\begin{object} При всех $n\in N$ функции $A_n(x)=a_1(x)+a_2(x)+a_n(x)=\sum^{n}_{k=1}a_k(x)$ называется ($n$-й) \textbf{частичной суммой} функционального ряда $\Sigma a_n(x)$ и $a_n(x)$ -- его \textbf{общим членом}.\end{object}\smallskip

В дальнейшем пусть $D$ обозначает область определения функционального ряда $\Sigma a_n(x)$, т.е. последовательности $\{ A_n(x)\}$.\smallskip

\begin{object} Если при фиксированном $x=x_0\in D$ сходится числовой ряд $\Sigma a_n(x_0)$, то говорят, что функциональный ряд $\Sigma a_n(x)$ \textbf{сходится в точке} $x=x_0$.\end{object}\smallskip

\begin{object} Множество $D_0\subset D$, состоящее из тех точек $x_0$, в которых ряд $\Sigma a_n(x)$ (или последовательность $A_n(x)$) сходится, называется \textbf{областью сходимости} этого ряда (или этой последовательности).\end{object}\smallskip

\textit{Замечание.} Область сходимости функционального ряда обычно бывает уже, чем область ее определения. Пример -- бесконечная геометрическая прогрессия $\frac{q}{1-q}=\sum^{\infty}_{n=1}q^n$.\smallskip

\begin{object} Пусть $D_0$ -- область сходимости функциональной последовательности $\{ A_n(x)\}$ и пусть $A(x)$ есть предельное значение этой последовательности при фиксированном хначении $x\in D_0$. Тогда множество пар $(x,A(x))$ при всех $x\in D_0$ задает некоторую функцию $y=A(x),$ определенную на всем множестве $D_0$. Эта функция называется \textbf{предельной функцией} функциональной последовательности $\{ A_n(x)\}$. Если при этом $A_n(x)$ -- последовательность частичных сумм ряда $\Sigma a_n(x)$, то функция $A(x)$ называется \textbf{суммой} этого ряда. Итак, сумма функционального ряда -- это некоторая функция, определенная на его области сходимости. При $x\in D_0$ \textbf{остаток ряда} $r_n(x)$ тоже представляет собой некоторую функцию от $x$, $r_n(x)=A(x)-A_n(x),$ причем $r_n(x)\to 0$ при $n\to \infty$ и при любом $x\in D_0$.\end{object}\smallskip

Многие свойства суммы $A(x)$ такие, например, как непрерывность суммы ряда $\Sigma a_n(x)$, связаны с поведением его остатка $r_n(x)$ при $n\to \infty$. Для описания этого поведения далее будет введено важное понятия равномерной сходимости функциональных рядов и функциональных последовательностей на множестве. Для того чтобы подчеркнуть отличие от него введенного выше понятия простой сходимости, последнюю еще называют \textbf{поточечной сходимостью.}

Важные примеры функциональных рядов возникают из разложения различных функциий по формуле Тейлора. Например, разлагая в точке $x_0=0$ функцию $y=\sin x$ при $x \in \R$, имеем $$\sin x=x-\frac{x^3}{3!}+\cdots+(-1)^n-1\frac{x^{2n-1}}{(2n-1)!}+r_n(x),$$ где $r_n(x)$ --остаточный член формулы. Записывая его в форме Лагранжа, получим $$r_n(x)=\frac{x^{2n}}{(2n)!}\sin^{2n} z$$ при некоторой точке $z$ с условием, что она лежит между точками $0$ и $x$. Отсюда $$|r_n(x)|\le\frac{|x|^{2n}}{(2n)!}.$$ Но при $n>x^2$ имеют место следующие неравенства: $$(2n)!>n^{n+1},\quad \frac{x^{2n}}{(2n)!}<\frac{x^2}{n^2}<\frac{1}{n},$$ т.е. $r_n(x)\to 0$ при $n\to \infty$.

Таким образом, полагая $$a_n(x)=\frac{(-1)^{n-1}x^{2n-1}}{(2n-1)!},$$ при всех $x \in \R$ имеем разложение $\sin x=\sum^{\infty}{n=1}a_n(x).$

\begin{object} Степенной ряд $\sum^{\infty}_{n=0}\frac{f^{(n)}(a)}{n!}(x-a)^n$ назывется \textbf{рядом Тейлора} функции $f(x)$ в точке $x=a$, а также \textbf{разложением} функции $f(x)$ в ряд Тейлора в этой точке.\end{object}\smallskip

\textbf{Примеры} рядов Тейлора для некоторых функций:

$1) e^x=\sum^{\infty}_{n=0}\frac{x^n}{n!}(\forall x\in R);$\smallskip

$2)\ln(1+x)=\sum^{\infty}_{n=1}(-1)^{n-1}\frac{x^n}{n}(-1<x\le1);$\smallskip

$3)\sin x=\sum^{\infty}_{n=1}(-1)^{n-1}\frac{x^{2n-1}{(2n-1)!}}(\forall x\in R);$\smallskip

$4)\cos x=\sum^{\infty}_{n=0}(-1)^{n-1}\frac{x^{2n}}{(2n)!}(\forall x\in R);$\smallskip

$5) (1+x)^\alpha=1+\sin x=\sum^{\infty}_{n=1}\frac{\alpha(\alpha-1)\ldots(\alpha-n+1)}{n!}x^n (-1<x\le1);$\smallskip

$6) \arctan x=\sum^{\infty}_{n=1}(-1)^{n-1}\frac{x^{2n-1}}{2n-1}(|x|\le1);$\smallskip

$7) \arcsin x=x+\sum^{\infty}_{n=1}\frac{(2n-1)!!}{(2n)!!}\cdot\frac{x^{2n+1}}{2n+1}(|x|\le1).$


\section{Равномерная сходимость}


\begin{object} Пусть последовательность функций $\{ r_n(x)\}$ сходится к нулю при всех $x \in M$. Тогда говорят, что $r_n(x)$ \textbf{сходится к нулю на множестве $M$}, если для любого $\varepsilon >0$ найдется такой номер $n_0=n_0(\varepsilon)$, что при всех $n>n_0$ и одновременно при всех $x \in M$ выполнено неравенство $|r_n(x)|<\varepsilon$.\smallskip

В этом случае используют обозначение: $r_n(x)\mathop{\rightrightarrows}\limits_{M} 0$ при $n\rightarrow \infty$.\end{object}\smallskip

\textit{Замечание.} Слово "одновременно" в этом определении вообще говоря, является избыточным и его можно опустить, однако оно обращает внимание на главное отличие равномерной сходимости от поточечной, состоящее в том, что в первом случае число $n_0(\varepsilon)$ в определении предела одно и то же для всех точек $x \in M$, а во втором случае оно может зависеть ещё и от $x$, т.е. $n_0(\varepsilon)=n_0(\varepsilon,x)$.\smallskip
\begin{object} Если функция $A(x)=A_n(x)+r_n(x)$, где $r_n\mathop{\rightrightarrows}\limits_{M} 0$ при $n\rightarrow \infty$, то последовательность $A_n(x)$ называют \textbf{равонмерно сходящейся к функции $A(x)$ на множестве $M$} при $n\rightarrow \infty$ и это обзначают так:
\begin{center}$A_n(x)\mathop{\rightrightarrows}\limits_{M} A(x)$ при $n\rightarrow \infty$.\end{center}\smallskip

Символ $M$ здесь можно опустить, если по смыслу понятно, о каком множестве идет речь. Далее, если при этом $A_n(x)$ -- последовательность частичных сумм ряда $\Sigma a_n(x)$, то этот ряд называют \textbf{равномерно сходящейся к $A(x)$ на множестве $M$}.\end{object}

Важность введенного понятия равномерной сходимости видна на примере следующей теоремы.\smallskip


\begin{theorem} Пусть каждая из функций $a_n(x)$ непрерывна в точке $x_0 \in \R$ и ряд $\Sigma a_n(x)$ равномерно сходится к функции $A(x)$ на интервале $I=(x_0-\delta,x_0+\delta)$, где $\delta>0$ -- некоторое фиксированное число. Тогда сумма $A(x)$ является непрерывной функцией в точке $x=x_0$.\end{theorem}

$\blacktriangleleft$ По определению равномерной сходимости имеем
\begin{center}$A(x)=A_n(x)+r_n(x), r_n(x)\mathop{\rightrightarrows}\limits_{I}0 (n\to \infty),$ $A_n(x)= \sum^{n}_{k=1}a_k(x), r_n(x)=\sum^{\infty}_{k=n+1}a_k(x).$
\end{center}
Используя обозначение $\Delta f(x)=f(x)-f(x_0)$, где $f(x)$ -- любая функция, получим
\begin{center}$\Delta A(x)=\Delta A_n(x)+\Delta r_n(x)=\Delta A_n(x)+ r_n(x)-r_n(x_0).$
\end{center}
Отсюда
\begin{center}$|\Delta A(x)|\le |\Delta A_n(x)|+|r_n(x)|+|r_n(x_0)|.$
\end{center}
Поскольку $r_n(x)\mathop{\rightrightarrows}\limits_{I}0 (n\to \infty)$ при любом 
$ \varepsilon_1>0 $
найдется номер $n_0=n_0(\varepsilon_1)$ такой, что для всех $n>n_0$ и для всех $x \in I$ имеем
\begin{center}$|r_n(x)|<\varepsilon_1, |r_n(x_0)|<\varepsilon_1.$
\end{center}
Заметим теперь, что функция $A(x)$ непрерывна в точке $x=x_0$, поэтому для любого $ \varepsilon_1>0 $ найдется $\delta_1=\delta_1(\varepsilon_1)>0$ такое, что при всех $x$ с условием $|x-x_0|<\delta_1$ выполнено неравенство
\begin{center}$|\Delta A_n(x)|=|A_n(x)-A_n(x_0)|<\varepsilon_1.$
\end{center}
Теперь при заданном $\varepsilon>0$ можно взять $\varepsilon_1=\varepsilon/3$, и тогда при всех $x$ с условием $|x-x_0|<\delta(\varepsilon)=\delta_1(\varepsilon_1)$ и при $n=n_0(\varepsilon_1)+1=n_0(\varepsilon)$ получим
\begin{center}$|\Delta A(x)|\le |\Delta A_n(x)|+|r_n(x)|+|r_n(x_0)|<\varepsilon_1+\varepsilon_1+\varepsilon_1=\varepsilon.$
\end{center}
Но это и означает, что функция $A(x)$ непрерывна в точке $x=x_0$. $\blacktriangleright$

Далее рассмотрим некоторые простые свойства равномерно сходящихся функциональных последовательностей.\smallskip

\begin{object} Последовательность функций $\{ A_n(x)\}$ называется \textbf{равномерно ограниченной на множестве $M$}, если существует такое число $C$, что при всех $n\in \mathbb{N}$ и при всех $x \in M$ имеем
\begin{center}$|A_n(x)|<C.$\end{center}\end{object}

\begin{approval}Равномерно сходящаяся на множестве $M$ последовательность $A_n(x)$, состоящая из ограниченных на $M$ функций, является равноиерно ограниченной на $M$.\end{approval}
$\blacktriangleleft$ Пусть $B_m = \mathop{sup}\limits_{x\in M}|A_m(x)|$ для каждого натурального числа $m$. В определении равномерной сходимости возьмем $\varepsilon=1$. Тогда при всех достаточно больших $n>n_0$ и при всех $x\in M$
\begin{center}$|A(x)-A_n(x)|<1,\quad |A(x)|\le |A_n(x)|+1\le B_m+1.$
\end{center}
Это значит, что $A(x)$ ограничена.

Далее, пусть $B_0=\mathop{sup}\limits_{x\in M}|A(x)|,B= \mathop{max}\limits_{0\le k \le n_0}B_k$. Положим $C=B+1$. Тогда при $k \le n_0$ справедлива оценка
\begin{center}$|A_k(x)|\le B<B+1=C,$
\end{center}
а при $k>n_0$ имеем
\begin{center}$|A_k(x)|\le|A(x)-(A(x)-A_k(x))|\le|A(x)|+|A(x)-A_k(x)|\le B_0+1\le B+1=C.$ $\blacktriangleright$
\end{center}
Попутно доказано еще одно утверждение.\smallskip

\begin{approval} Если функция $A(x)$ является ограниченной на множестве $M$ и $A_n(x)\mathop{\rightrightarrows}\limits_{M} A(x)$, то при некотором $n_0\in \mathbb{N}$ функциональная последовательность $B_n(x)=A_{n_0+n}(x)$ равномерно ограничена на $M$.\end{approval}\smallskip

Следующие два утверждения приведем без доказательства, посколькуони доказываются точно так же, как и в аналогичных случаях для числовых рядов.\smallskip

\begin{approval}Пусть при $n\to \infty$ имеем
\begin{center}$a_n(x)\mathop{\rightrightarrows}\limits_{M}a(x), \quad b_n(x)\mathop{\rightrightarrows}\limits_{M}b(x).$
\end{center}
Тогда:

$1^0.a_n(x)+b_n(x)\mathop{\rightrightarrows}\limits_{M}a(x)+b(x)$;

$2^0.$ если $|b(x)|<C$ при некотором $C<0$ и всех $x \in M$, то

\begin{center}$a_n(x)\cdot b_n(x)\mathop{\rightrightarrows}\limits_{M}a(x)\cdot b(x);$
\end{center}

$3^0.\frac{a_n(x)}{b_n(x)}\mathop{\rightrightarrows}\limits_{M}\frac{a(x)}{b(x)},$ если только $1/|b(x)|>C>0$ при всех $x\in M.$\end{approval}

\begin{approval}Если последовательность $d_n(x)$ является равномерно ограниченной и $r_n(x)\mathop{\rightrightarrows}\limits_{M}0$ при $n\to \infty$, то $d_n(x)r_n(x)\mathop{\rightrightarrows}\limits_{M}0$ при $n\to \infty$.\end{approval}


\setcounter{object}{0}
\setcounter{approval}{0}
\setcounter{theorem}{0}
\setcounter{example}{0}

\chapter{Несобственные интегралы второго рода. Интеграл Дирихле.}
\centerline{ \bf Автор: Ромашкина Вероника} \vskip 1cm

Здесь мы сформулируем основные понятия элементарной теории несобственных параметрических интегралов второго рода и приведем формулировки некоторых утверждений, соответствующих доказанным нами теоремам об интегралах первого рода.
\smallskip
Рассмотрим множество $P = X \times Y$, где $ X=(a,b], Y\subset\mathbb R.$ Пусть функция $f(x,y)$ задана на $P$ и не ограничена как функция от $x$ хотя бы при одном фиксированном $y \in Y.$ Далее, пусть при любых $y \in Y$ и $\delta>0, \delta \in (0,b-a) $ функции $f(x,y)$ интегрируема по Риману  на отрезке $ [a+\delta,b]$ как функция от $x$.
\medskip

\begin{object}\itshape Введенное выше формальное выражение вида $ \int\limits_a^b f(x,y)dx$ называется \upshape \bfseries несобственным параметрическим интегралом второго рода \mdseries \itshape с одной особой точкой\upshape {} $x=a$.
\end{object}
\medskip

\begin{object}\itshape Если при любом фиксированном значении $y \in Y$ этот интеграл сходится, то множество $Y$ называется \upshape \bfseries областью сходимости интеграла \mdseries \itshape и его значения $g(y)= \int\limits_a^b f(x,y)dx$ порождают функцию, определенную на множестве $Y$.\upshape\end{object}
\medskip

Подобные определения имеют место и в случае, когда особая точка находится на правом конце промежутка интегрирования $X=[a,b]$, т.е. в точке $b$. В случае когда особая точка $x=x_0$ лежит внутри отрезка $X$ , его можно разбить на две части этой точкой $x_0$ и рассматривать каждую часть отрезка отдельно.\\
Аналогичные рассуждения позволяют рассматривать несобственные интегралы с переменной особой точкой $x_0=x_0(y)$, но здесь мы входить в детали не будем.
\medskip

\bfseries Пример. \mdseries Интеграл $ J=\int\limits_0^1 \frac{dx}{\sqrt{|x|}}$ сходится на $Y=\lbrack0,1\rbrack$ и его можно вычислить.

Действительно, имеем
$$ g(y)=g_1(y)+g_2(y)=\int\limits_0^y \frac{dx}{\sqrt{y-x}} + \int\limits_y^1\frac{dx}{\sqrt{x-y}}=2\sqrt{y}+2\sqrt{1-y}. $$

\begin{object} \itshape Несобственный  интеграл  второго рода
$$ g(y) = \int\limits_a^b f(x,y)dx$$
называется \upshape \bfseries равномерно сходящимся по $y$ на множестве $Y$ \mdseries \itshape, если для функции
$$ g(\delta,y)=\int\limits_{a+\delta}^b f( x,y)dx \quad \delta \to 0+$$
иыполнено условие \upshape
$$ g(\delta,y)\rightrightarrows  g(0,y)= g(y). $$
\end{object}

Исходя из общей формулировки критерия Коши можно сформулировать его для равномерной сходимости несобственного параметрического интеграла второго рода. Но мы ограничимся формулировкой одной сводной теоремы, содержащей утверждения, важные для практических применений.
\medskip

\begin{theorem} \itshape Пусть функция $f(x,y)$ непрерывна на $P = X\times Y,$ где $ X = (a,b], Y = [c,d]$. Пусть $a$ --- особая точка несобственного параметрического интеграла
$$ g(y) = \int\limits_a^b f(x,y)dx.$$
тогда справедливы следущие утверждения:

\bfseries 1. \mdseries Если интеграл $\int\limits_a^b f(x,y)dx$ сходится равномерно на $Y$, то функция $g(y)$ непрерывна при все $y \in Y$.

\bfseries 2. \mdseries В этом случае имеем
$$ \int\limits_c^d g(y)dy=\int\limits_a^b dx\int\limits_c^d f(x,y)dy.$$
\bfseries 3. \mdseries Если интеграл $\int\limits_a^b f(x,y)dx$ сходится,частная производная $f'_y(x,y)$ существует и непрерывна на $P,$ а интеграл $\int\limits_a^b f'_y(x,y)dx$ сходится равномерно на $Y$, то существует $g'(y)$, причем \upshape
$$ g'(y) = \int\limits_a^b f'_y(x,y)dx. $$
\end{theorem}

Если особая точка $x_0$ является внутреней точкой отрезка $X = \lbrack a,b\rbrack,$ то как было отмечено выше, необходимо отрезок $X$ разбить этой точкой на две части и рассматривать каждый из двух получившихся интегралов отдельно. Тот же подход можно применить и в случае, когда бесконечный промежуток интегрирования $X=\lbrack a,+\infty)$ содержит конечное число особых точек $x_1,\ldots,x_n.$ Тогда этот промежуток можно разбить на $2n$ промежутков точками $t_1<t_2<\cdots<t_{2n}$ таким образом, чтобы на каждомотрезке вида $[t_s,t_{s+1}],$ где $s=1,2,\ldots,2n-1,$ лежала бы ровно одна особая точка, а на промежутке $[t_{2n},+\infty)$ особых точек не блыо. В результате получим $2n-1$ несобственных интегралов второго рода и еще один - первого. 
\bigskip

Начнем с вычисления интеграла Дирихле $D(\alpha),$ называемого еще разрывным множителем Дирихле. По определению имеем
$$ D(\alpha)=\int\limits_0^\infty \frac{\sin \alpha x}{x}dx.$$
заметим прежде всего, что точка $x=0$ не является особой, так как подынтегральная
функция ограничена. Очевидно, что $D(0)=0$. Далее, если $\alpha > 0$, то интеграл сходится по признаку Дирихле, поскольку
% tuuuut
$$\left| \int\limits_0^t \sin\alpha xdx\right| =\left|\frac{1-\cos\alpha t}{\alpha}\right|<\frac2\alpha.$$
В этом случае возможна линейная замена переменной интегрирования вида $\alpha x=t$, тогда имеем
$$ D(\alpha) = \int\limits_0^\infty\frac{\sin\alpha x}{\alpha x}dx=\int\limits_0^\infty \frac{\sin t}{t}dt= D(1) = D.$$
Если же $\alpha<0,$ то $\alpha=-|\alpha|, \sin\alpha x=-\sin|\alpha| x,$ откуда
$$ D(\alpha) = \int\limits_0^\infty\frac{\sin\alpha x}{x}dx=-\int\limits_0^\infty \frac{\sin |\alpha| x}{x}dx=-D.$$
Таким образом имеем 
$$ D(\alpha)= 
\left\{\begin{array}{l} D \quad \mbox{при } \alpha>0,\\0 \quad \mbox{при } \alpha=0,\\
-D \quad \mbox{при } \alpha<0
\end{array}\right. $$
Теперь перейдем к вычислению значения $D$.
\medskip

\begin{theorem} \itshape Справедливо равенство $D = \pi/2.$\end{theorem}
\medskip

$\blacktriangleleft$ \upshape Рассмотрим параметрический интеграл $g(y),$ где $y\in Y = \lbrack0,N\rbrack, N\in \mathbb R$ и
$$ g(y) = \int\limits_0^\infty \frac{e^{-yx}\sin x}{x}dx.$$
Подынтегральная функция $f(x,y)=e^{-yx}\sin x/x$ будет непрерывна всюда на $ P = X\times Y,$ где $ X=[0,+\infty), Y=[0,N],$ если положить $f(0,y) = 1.$

Убедимся, что интеграл $g(y)$ сходится равномерно на $Y$. Для этого воспользуемся признаком Абеля. Положим $ \alpha(x,y)=\sin x/x, \beta(x,y)=e^{-yx}.$ Тогда функция $\beta(x,y)$ монотонна и $ 0<\beta(x,y)\le 1,$ а интеграл $\int\limits_0^\infty \alpha(x,y)dx $ сходится равномернона $Y$, поскольку $\alpha(x,y)$ не зависит от $y$.

Возьмем теперь на отрезке $Y$ произвольную точку $y_0\ne 0$ и окружим ее некоторым отрезком $ Y_\delta=\lbrack y_0-\delta,y_0+\delta\rbrack,$ целиком принадлежащим множеству$Y$. На этом отрезке интеграл
$$\int\limits_0^\infty f'_y(x,y)dx=-\int\limits_0^\infty e^{-yx}\sin xdx$$
сходится равномерно. Это следует из признака Вейерштрасса, поскольку $|e^{-xy}\sin x|< e^{-x(y_0-\delta)},$  а интеграл $ \int\limits_0^\infty e^{-x(y_0-\delta)}dx$ сходится. Кроме того, подынтегральная функция $e^{-xy}\sin x$ непрерывна на $ P_\delta = X\times Y_\delta.$ Поэтому по правилу Лейбница для несобственных интегралов имеем 
$$ g'(y) = -\int\limits_0^\infty e^{-yx}\sin xdx.$$
Последний интеграл можно вычислить путем интегрирования по частям. При этом получим
$$ g'(y)=-\frac{1}{1+y^2}.$$
Итак, мы показали, что функция $g(y)$ непрерывна на $ Y = \lbrack 0,N\rbrack,$ а ее производная существует при всех $y\ne 0$. Отсюда по формуле Ньютона-Лейбница при всех $y\in(0,N]$ вытекает равенство
$$ g(y) = g(N) - \int\limits_N^y \frac{dt}{1+t^2}=g(N)+\arctg N+\arctg y.$$
Пользуясь непрерывностью функции $g(y)$ в точке $y=0$, мы получим
$$ g(0)=\lim_{y\to 0+}(g(N)+ \arctg N-\arctg y)=g(N)+\arctg N.$$
Тепреь, устремляя $N$ к $+\infty$ приходим к соотношениям $\arctg N \to \pi/2$,
$$ |g(N)|\le\int\limits_0^\infty e^{-Nx}\frac{|\sin x|}{x}dx\le \int\limits_0^\infty e^{-Nx}dx=\frac1N\to 0.$$
Отсюда следует, что
$$ D=g(0)=\lim_{N\to\infty}(g(N)+\arctg N)=\pi/2. \blacktriangleright$$


\setcounter{object}{0}
\setcounter{approval}{0}
\setcounter{theorem}{0}
\setcounter{example}{0}
\chapter{Равномерная сходимость несобственных параметрических интегралов}
\centerline{\bf Автор: Кеберлинский Эльнур}\vskip 1cm


\vskip 1cm
Дальнейшее развитие теории интегралов,зависящих от параметра,приводит к рассмотрению несобственных интегралов,которые составляют ее наиболее существенную часть.Из двух типов таких интегралов сосредоточим свое внимание главным образом на интегралах первого рода.Интегралов второго рода коснемся лишь вскользь, поскольку их теория не имеет принципиальных отличий от интегралов первого рода.
\vskip 1cm
Рассмотрим функцию $f(x,y)$, заданную на множестве $I \in Y $, где I-промежуток вида $[a,+\infty)$, а Y-некоторое множество вещественных чисел, т.е. $ Y \in \R.$ Допустим,что при любом фиксированном $y \in Y $ функция $f(x,y) $ интегрируема на конечном отрезке вида $ [a,b] $ и существует несобственный интеграл первого рода от этой функции по переменной $ x \in Y=[a,+\infty)$. Тогда этот интеграл сам представляет собой некоторую функцию от $y$, заданную на $ Y$ равенством
$$  g(y)=\int_a^\infty {f(x,y)\dx}.$$

\begin{object}
Функция $g(y)$, представленная в указанном выше виде, называется несобственным интегралом первого рода, зависящим от параметра $y \in Y$.
\end{object}

Замечание. Вместо несобственных интегралов по промежутку вида $[a,+\infty)$ можно,
разумеется,
рассматривать интегралы по промежуткам вида $(-\infty,b]$ или по всей вещественной прямой $\R =(-\infty,+\infty)$. 
Все эти случаи сводятся к рассмотренному точно так же,
как это делалось при изучении обычных несобственных интегралов. 
Например, интеграл
$$\int_{-\infty}^{+\infty} {f(x,y) \dx},y \in Y,$$
достаточно представить в виде суммы интегралов
$$\int_{-\infty}^{+\infty} {f(x,y) \dx}=\int_{-\infty}^0 {f(x,y)\dx} + \int_0^{+\infty} {f(x,y)\dx}$$
и сходимость этой суммы понимать как сходимость каждого из двух ее слагаемых. Первое слагаемое сводится ко второму заменой переменной $x$ на $-x$. Кроме того, можно, конечно, рассматривать и формальные несобственные параметрические интегралы и при этом ставить вопрос об области их сходимости $Y$. Подобного рода вопросы разобраны при рассмотрении функциональных рядов, поэтому мы им много внимания уделять не будем, иногда, однако, будем пользоваться аналогичной терминологией.

\begin{example}
При $y>1$ справедливо равенство 
$$\int_1^\infty {\frac {\dx} {x^y}}=\lim_{t\to +\infty}\int_1^t{\frac {\dx} {x^y}}=\lim_{t\to +\infty} 
\left.\frac {x^{1-y}} {1-y}\right|_1^t=\frac {1} {1-y}$$
\end{example}
\begin{example}
При $y>0$ имеем
$$\int_0^\infty {\frac {\sin xy} {x} \dx}=\int_0^\infty {\frac {\sin xy} {xy}\dxy}=\int_0^\infty {\frac {\sin x} {x}\dx}.$$
\end{example}
\begin{object}
Интеграл $\int_a^\infty {f(x,y)\dx}\quad $ называется равномерно сходящимся по параметру $y$ на множестве $Y,\{y\}=Y$, если
%$$\int_a^t f(x,y)\dx=F(y,t)\stackrel{Y}{\rightrightarrows} g(y) при t\to +\infty. $$
Другими словами, это значит, что для любого $\varepsilon $>0 существует $t=t_0(\varepsilon)$ такое, что при всех $t>t_0(\varepsilon)$ и всех $y\in Y$ имеем

$$ \left|\int_a^t {f(x,y)\dx}-g(y)\right|<\varepsilon,$$

где $g(y)=\int_a^\infty {f(x,y)\dx}$.
\end{object}
Исходя из общей теоремы сформулируем критерий Коши кокретно для равномерной сходимости несобственных интегралов первого рода.
\begin{theorem} 
Необходимое и достаточное условие равномерной сходимости несобственного интеграла первого рода $\int_a^\infty {f(x,y)\dx}$ на множестве $Y$ состоит в том, чтобы для любого $\varepsilon >0$ существовало $T=T(\varepsilon)$ такое, что при всех $t_2>t_1>T$ и любом $y\in Y$ выполнялось бы неравенство

$$\left|\int_{t_1}^{t_2} {f(x,y)\dx}\right| < \varepsilon.$$
\end{theorem}
Приведем также прямую формулировку критерия отсутствия равномерной сходимости несобственного параметрического интеграла.

\begin{theorem}
Равномерная сходимость несобственного интеграла
$$\int_a^\infty {f(x,y)\dx}$$
на множестве $Y$ не имеет места, если найдется $\varepsilon >0$ такое, что для любого $T\in \R$ найдутся числа $t_1$ и $t_2>T$ и $y\in Y$ такие, что

$$\left|\int_{t_1}^{t_2} {f(x,y)\dx}\right|\ge \varepsilon$$ 
\end{theorem}
\begin{object}
Если интеграл $\int_a^\infty {g(x)\dx}$ сходится и при всех $x>a, y\in Y$ имеем $|f(x,y)|\le g(x)$, то функция $g(x)$ называется \textbf{мажорантой} для $f(x,y)$ на $\Pi =I\times Y$.
\end{object}
\begin{theorem}
(признак Вейерштрасса равномерной сходимости несобственных интегралов первого рода). Интеграл $J=\int_a^\infty {f(x,y)\dx}$ сходится равномерно на $Y$, если функция $f(x,y)$ имеет мажоранту $g(x)$ на $\Pi=X\times Y$, где $X=[a,+\infty)$.
\end{theorem}
$\blacktriangleleft$
Воспользуемся критерием Коши. Поскольку интеграл $\int_a^\infty {g(x)\dx}$ сходится, при любом $\varepsilon >0$ найдется число $T=T(\varepsilon)$ такое, что при всех $t_2>t_1>T$ выполнено неравенство
$$\int_{t_1}^{t_2} {g(x)\dx}<\varepsilon.$$
Но тогда при всех $y\in Y$ имеем
$$\left|\int_{t_1}^{t_2}{f(x,y)\dx}\right|\le \int_{t_1}^{t_2}{|f(x,y)|\dx}\le \int_{t_1}^{t_2}{g(x)\dx}<\varepsilon$$
Отсюда согласно критерию Коши заключаем, что интеграл $J$ сходится равномерно на $Y$. Теорема доказана. 
$\blacktriangleright$
\begin{example} 
При $s>s_0>1$ интеграл $\int_1^\infty{x^{-s}\dx}$ сходится равномерно на множестве $s\ge s_0$, поскольку он имеет мажоранту $g(x)=x^{-s_0}$.
\end{example}

\begin{theorem}
(Признаки Абеля и Дирихле для равномерной сходимости параметрических несобственных интегралов первого рода).
Пусть функция $f(x,y)$ определена на множестве $\Pi =X\times Y$,где $X=[a,+\infty), Y=[c,d]$ и $f(x,y)=\alpha(x,y)\beta(x,y)$. Пусть $\beta (x,y)$ монотонна по $x$ при любом фиксированном $y\in Y$.

(А)(признак Абеля). Пусть, кроме того:

1) интеграл $\int_a^\infty {\alpha (x,y)\dx}$ сходится равномерно по $y$ на $Y$;

\parindent=1cm2)функция $\beta (x,y)$ ограничена на $\Pi =X\times Y$,т.е. $|\beta (x,y)|<c$ при некотором вещественном числе $c>0$ и всех $(x,y)\in \Pi$.

Тогда интеграл $J=\int_a^{\infty}{f(x,y)\dx}$ сходится равномерно на $Y$ 

(Д) (признак Дирихле). Пусть вместо условий (А) имеем:

1) при некотором $c>0$ и всех $t>a, y\in Y$ имеет место неравенство
$$\left|\int_a^t {\alpha (x,y)\dx}\right|<c;$$

2)функция $\beta(x,y)$ равномерно на $Y$ сходится к нулю при $x\to 0$ 

Тогда, как и в случае (А), интеграл $J$ сходится равномерно на $Y$.
\end{theorem}
$\blacktriangleleft$
Эта теорема как по своей формулировке, так и по доказательству похожа на соответствующие утверждения из теории рядов. По существу, все отличие сводится к замене использования преобразования Абеля на применение второй теоремы о среднем значении интеграла.

Для доказательства снова воспользуемся критерием Коши. Применяя вторую теорему о среднем, имеем
$$\int_{t_1}^{t_2}{\alpha(x,y)\beta(x,y)\dx}=\beta(t_1,y)\int_{t_1}^{t_3}{\alpha(x,y)\dx}+\beta(t_2,y)\int_{t_3}^{t_2}{\alpha(x,y)\dx}, $$ 
где $t_3$ - некоторая точка отрезка $[t_1,t_2]$.

Теперь в случае (А) в силу равномерной сходимости интеграла $\int_a^\infty {\alpha(x,y)\dx}$ при любом $\varepsilon>0$ и всех достаточно больших $t_2>t_1>t_0(\varepsilon)$ имеем $\left|\int_{t_1}^{t_3}{\alpha(x,y)\dx}\right|<\varepsilon$ и $\left|\int_{t_3}^{t_2}{\alpha(x,y)\dx}\right|<\varepsilon$, откуда

$$\left|\int_{t_1}^{t_2}{\alpha(x,y)\beta(x,y)\dx}\right|\le |\beta(t_1,y)|\left|\int_{t_1}^{t_3}{\alpha(x,y)\dx}\right|+|\beta(t_2,y)|\left|\int_{t_3}^{t_2}{\alpha(x,y)\dx}\right|\le$$ $$\le c\varepsilon+c\varepsilon=2c\varepsilon,$$
поскольку $|\beta(x,y)|<c$ при всех $x$ и $y$.

В силу произвольности числа $\varepsilon>0$ это влечет за собой равномерную сходимость интеграла $J$ и справедливость утверждения (А).

В случае (Д) интегралы от функции $\alpha(x,y)$ ограничены числом $c$ и $\beta(x,y)$ стремится к нулю равномерно по $y$,  поэтому при всяком $\varepsilon>0$ и достаточно больших $t_2>t_1>t_0(\varepsilon)$ выполнено неравенство $|\beta(x,y)|<\varepsilon$, откуда с учетом предыдущей формулы имеем
$$\left|\int_{t_1}^{t_2}{\alpha(x,y)\beta(x,y)\dx}\right|\le c\varepsilon+c\varepsilon=2c\varepsilon,$$
что влечет за собой справедливость утверждения (Д). Теорема доказана.
$\blacktriangleright$

\input{06_Nazrin.tex}
\input{07_Emil.tex}
\input{08_Orxan.tex}
\chapter{Двойной интеграл Римана как предел по базе}
\centerline{ \bf Автор: Фаталиева Сабина} \vskip 1cm

\section{Определение двойного интеграла Римана}

\ \par Двойной интеграл - это интеграл от функции двух пременных, взятый по обеим переменным \textit{одновременно}. Данная фраза не является определением, она указывает на то, как мы намерены вводить общие понятия определенного интеграла на случай функции двух пременных.
\par Для того, чтобы получить такое обобщение, вспомним, как выглядит определение интегала в одномерном случае, то есть в случае функции $y=f(x)$ от одной переменной $x$, определенной на отрезке $I=[a,b]$ и интегрируемой на нем по Риману. Одно из эквивалентных определений данного понятия можно сформулировать так.
\begin{object} 
Интегралом $\int\limits_{a}^{b}f(x)\,dx$ от ограниченной функции $f(x)$ называется число, равное \textbf{алгебраческой сумме} площадей криволинейных трапеций, обраованных кривой $y=f(x)$ при $x\in{[a,b]}$. При этом в данную сумму входят площади криволинейных трапеций, расположенных над осью абсцисс со знаком +, а под ней - со знаком -.
\end{object}
\par Если мы обощим понятие криволинейной трапеции на случай, скажем, фекции двух пременных $z=g(x,y)$, заданной на прямоугольнике $P=I_1\times I_2=[a_1,b_1]\times [a_2,b_2]$, то и получим одно из возможынх определений двойного интеграла от функции $g(x,y)$ по прямоугольнику P.  Этот интеграл обозначается символом $$\iint_P g(x,y)dxdy=\int_{a_1}^{b_1}\int_{a_2}^{b_2} g(x,y)dxdy.$$
\par Допустим сначала, что $g(x,y)\ge 0$ для всех $(x,y)\in P$. Вместо криволтнейной трапеции рассмотрим пространственную финуру $H$, заключенную между поверхностью $z=g(x,y)$ и плоскостью $z=0$ при $(x,y)\in P$. Другими словами, фигура $H$ состоит изо всех тех точек $(x,y,z)$, для которых $x\in I_1,\quad y\in I_2$, а третья координата $z$ удовлетворяет условию $0\le z\le g(x,y)$. 
\begin{object}
Фигуру $H$ будем называть \textbf{цилиндрической криволинейной фигурой}, порожденной поверхностью $z=g(x,y)$.
\end{object}
\par Если окажется, что эта фигура измерима каким-либо образом, то ее меру $\mu (H)$ можно взять в качестве искомого пределения значения двойного интеграла $$I=\iint_P g(x,y)dxdy=\mu (H).$$
\par Заметим, что если $\mu$ есть мера Жордана, то данное выше определение двойного интеграла будет эквивалентным определению двойного интеграла Римана, которое будет сейчас дано. Можно было бы таким же образом разобрать общий случай, когда функци $g(x,y)$ принимает как положительные, так и отрицательные значения, но мы это сделаем  дальнейшем при доказательстве критерия измеримости по Жордану цилиндрической криволинейной фигуры.   
\par Перейдем теперь к построению терии двойного интеграла Римана по прямоугольнику $P$. Cначала определим понятие цилиндрической фигуры в общем случае. 
\begin{object}
Фигура $H\subset \in{\mathbb R}^3$ называется textbf{цилиндрической криволинейной фигурой}, порожденной поверхностью $z=g(x,y)$, заданной на $P$, если $H$ состоит изо всех таких точек $(x,y,z)$, для которых $(x,y)\in P$, а координата $z$ заключена между числами 0 и $g(x)$, то есть при $g(x,y)\ge 0$ имеем $0\le z\le g(x,y)$, а при $g(x,y)<0$ имеем $g(x,y)\le z\le 0$.
\end{object}
\par Разобьем прямоугольник $P$ на меньшие прямоугольники с помощью прямых, параллельнх осям $Ox$ и $Oy$ и проходящих через точки $a_1=x_0<x_1<c\dots <x_m=b_1$ разбиения $T_x$ на оси $Ox$ и $a_2=y_0<y_1<\cdots <y_n=b_2$ разбиения $T_y$ на оси $Oy$.
\par Прямоугольник $P_{k,l}\subset P$, точки $(x,y)$ которого удовлетворяют условиям $$x\in \Delta_k^{(x)}=[x_{k-1},x_k],\quad y\in \Delta_l^{(y)}=[y_{l-1},y_l],$$ где $\Delta_k^{(x)}$ есть $k$-й отрезок разбиения $T_x$ и $\Delta_l^{(y)}$ - $l$-й отрезок разбиения $T_y$, будем называть элементом разбиения $T$ прямоугольника $P$ с индексом $(k,l)$, а множество всех прямоугольников $P_{k,l}$, $k=1,\ldots ,m,\quad l=1\ldots ,n$ - разбиением $T$ прямоугольника $P$.
\par В каждом прямоугольнике $P_{k,l}$ возьмем точку $A_{k,l}$ с координатами $(\xi_{k,l},\theta_{k,l})$. Множество прямоугольников $P_{k,l}$ и точек $A_{k,l}$ будем называть размеченным разбиением прямоугольника $P$ и будем обозначать его через $V$.
\par Очевидно, что каждому размеченному разбиению $V$ однозначно соответствует разбиение $T$ прямоугольника $P$, получаемое из $V$ отбрасыванием точек "разметки" \\$(\xi_{k,l},\theta_{k,l})$. 
Другими словами, $T$ является функцией от $V$, $T=T(V)$.
\begin{object}
Cумма $$\sigma (V)=\sum\limits_{k=1}^m\sum\limits_{{l=1}_n} g(\xi_{k,l},\theta_{k,l})\Delta x_k\Delta y_l$$ назывется \textbf{интегральной суммой Римана} функции $g(x,y)$, соответствующей (отвечающей) размеченному разбиению $V$ стандартного прямоугольника $P$.
\end{object}
\par Длину диагонали $\sqrt{\Delta x_k^2+\Delta y_{l_2}}$ прямоугольника $P_{k,l}$ будем называть его \textbf{диаметром}.
\begin{object} 
\textbf{Диаметром разбиения} (размеченного $V$ и неразмеченного $T$) прямоугольника $P$ будем называть иаксимальное значение диаметров элементов разбиения $P_{k,l}$. Обозначать его будем символом $\Delta_V$ и, соответственно, $\Delta_T$.
\end{object}
\begin{object} 
Число $I$ называется (двойным) \textbf{интегралом Римана} от ограниченной функции $g(x,y)$ по прямоугольнику $P$, если для $\forall$ $\varepsilon >0$ $\exists$ $\delta =\delta (\varepsilon )>0$ такое, что для $\forall$ размеченного разбиения $V$ прямоугольника $P$ с условием $\Delta_{V}<\delta$ справедливо неравенство $$|\sigma (V)-I|<\varepsilon$$.
\end{object}
\par Здесь $\sigma (V)$ - интегральная сумма для функции $g(x,y)$, котрая соответствует размеченному разбиению $V$. Поэтому последнее неравенство можно записать еще и так: $$|\sum\limits_{k=1}^m\sum\limits_{{l=1}_n} g(\xi_{k,l},\theta_{k,l})\Delta x_k\Delta y_l-I|<\varepsilon .$$
\par В это случае будем говорить, что $g(x,y)$ является \textbf{интегрируемой по Риману} на прямоугольнике $P$.
\par Далее рассмотрим следующие вопросы:
\begin{enumerate}
\item убедимся, что интеграл $I$ есть предел по некоторой базе;
\item определим верхние и нижние суммы Дарбу и докажем критерий Римана интегрируемости функции от двух пременных;
\item установим свойства двойного интеграла, аналогичные свойствам однократного интеграла.
\end{enumerate}
\par Начнем с определения базы множеств $B$ и $B'$. Множество всех неразмеченных разбиений прямоугольника обозначим через $A_P$, а размеченных -  $A_P'$. В качестве окончаний $b'_{\delta}$ базы $B'$ возьмем множество \{$V|\Delta_V<\delta$\}, т.е. множество разбиений, состоящее из тех $V\in A\_P$, для которых диаметр $\Delta_V$ меньше, чем $\delta >0$.
\par Так как $\sigma (V)$ определена всюду на $A'_P$, то очевидно, тогда определение двойного интеграла, данное выше, эквивалентно определению предела $\lim_{B'} \sigma (V)$ по базе $B'$. Проверка справедливости этого утверждения состоит в том, что надо формально выписать определение предела по базе и сравнить его с данным выше определением. Далее базу $B'$ будем обозначать символом $\Delta_V \to 0$.
\par Совершенно аналогично определяем базу $\Delta_T \to 0$ для всех неразмеченных разбиений $A_P$.
\par Отметим, что неразмеченное разбиение $T$ прямоугольника $P$ можно определить и как пару $(T_x,T_y)$, состоящую из неразмеченного разбиения $T_x\quad :\quad a_1=x_0<x_1<\cdots <x_m=b_1$ отрезка $[a_1,b_1]$ на оси $Ox$ и неразмеченного разбиения $T_y\quad : \quad a_2=y_0<y_1<\cdots <y_n=b_2$ отрезка $[a_2,b_2]$ на оси $Oy$. Это разбиение $T$ получается проведением $m+1$ вертикальных прямых $x=x_k$,\quad $k=0,\ldots ,m$ и $n+1$ горизонтальных прямых  $y=y_l$,\quad $l=0,\ldots ,n$. Снова заметим, что если у размеченного разбиения $V$ отбросить разметку точками $(\xi_{k,l},\theta_{k,l})\in P_{k,l}$, то, очевидно, возникает неразмеченное разбиение, которое будем обозначать символом $T(V)=T$.
\begin{object} 
Множество всех размеченных разбиений $\{ V\}$, которым отвечает одно и то же неразмеченное разбиение $T_0$, будем называть \textbf{множеством разметок} $T_0$ и обозначать символом $A'_P(T_0)$. Если $V\in A'_P(T_0)$, то будем говорить, чт $V$ является \textbf{разметкой} $T_0$ или, что то же самое, $T(V)=T_0$.
\end{object}

\section{Суммы Дарбу и их свойства}

\par Переходим теперь к построению textit{теории Дарбу} для двойного интеграла Римана по прямоугольнику.
\par Обозначим для некоторого неразмеченного разбиения $T$ прямоугольника $P$ через $M_{k,l}$ и $m_{k,l}$ величины $$M_{k,l}=\quad \sup_{(x,y)\in P_{k,l}} g(x,y), m_{k,l}=\quad \inf_{(x,y)\in P_{k,l}} g(x,y).$$
\par Тогда \textbf{верхней суммой Дарбу} функции $g(x,y)$, соответствующей разбиению $T$, назывыется сумма $S(T)$, где $$S(t)=\sum_{k=1}^m \sum_{l=1}^n M_{k,l}\Delta x_k\Delta y_l,$$ а сумма $$s(T)=\sum_{k=1}^m \sum_{l=1}^n m_{k,l}\Delta x_k\Delta y_l$$ называетя \textbf{нижней суммой Дарбу}.
\textbf{Омега-суммой} $\Omega (T)$, отвечающей разбиению $T$, назовем величину $$\Omega (T)=S(T)-s(T)=\sum_{k=1}^m \sum_{l=1}^n \omega_{k,l}\Delta x_k\Delta y_l,$$ где $\omega_{k,l}=M_{k,l}-m{k,l}$.
\begin{object}
Число $I*=\quad \inf_{T\in A_P} S(T)$ называетя \textbf{верхним интегралом Дарбу} от функции $g(x,y)$ по прямоугольнику $P$, а число $I_{*}=\quad \sup_{T\in A_P} s(T)$ - \textbf{нижним интгралом Дарбу} от функции $g(x,y)$.
\end{object}
\par Нам потребуются следующие свойства сумм Дарбу.
\begin{lemma} 
Для любого размеченного разбиения $V\in A'_P$ имеем $$s(T(V))\le \sigma (V)\le S(T(V)).$$
\end{lemma}
\begin{lemma} 
Зафиксируем некоторе разбиение $T_0\in A_P$. Будем иметь следующие соотношения $$s(T_0)=\quad \inf_{V\in a'_P()T_0} \sigma (V),S(T_0)=\quad \sup_{V\in a'_P()T_0} \sigma V.$$
\end{lemma} 
\begin{lemma} 
Для любых неразмеченных разбиений $T_1$ и $T_2$ имеем $$s(T_1)\le S(T_2).$$
\end{lemma} 
\begin{lemma} 
\label{lemmfour}
Для ограниченной на прямоугольнике $P$ функции верний $I*$ и нижний $I_{*}$ интегралы Дарбу существуют, причем для любого разбиения $T\in A_P$ справедливы неравенства $$s(T)\le I_{*}\le I*\le S(T).$$
\end{lemma} 
\begin{lemma}
Размеченное разбиение $V$ принадлежит окончанию $b'_{\delta}\in B'$ $\leftrightarrow$ $T(V)\in_{\delta}$.
\end{lemma}
$\blacktriangleleft$ Доказательство лемм аналогично доказательству соответствующих утверждений в одномерном случае и не представляет большого труда. Стоит лишь сказать о лемме 3, поскольку там учавствуют два разных разбиения. Здесь, как и в одномерном случае, введем понятие \textbf{измельчения разбиения}.
\begin{object} 
Неразмеченное разбиение $T_2$ называется \textbf{измельчением разбиения} $T_1$, если разбиение $T_2$ получается из $T_1$ добавлением конечного числа новых точек разбиения на оси $Ox$ и по оси $Oy$. Говорят еще, что $T_2$ \textbf{следует за} $T_1$ и пишут $T_2\supset T_1$ или $T_1\subset T_2$. 
\end{object}
\par В частности, любое неразмеченное разбиение $T$ есть измельчение самого себя. Далее, очевидно, что при измельчении разбиения $T$ нижняя сумма Дарбу $s(T)$ не может уменьшиться, а верхняя сумма Дарбу $S(T)$ не может увеличиться. Поэтому для доказательства утверждения леммы 3 надо на каждой оси$Ox$ и $Oy$ взять разбиение $T_3$, объединяющее $T_1$ и $T_2$. Тогда получим $$s(T_1)\le s(T_3)\le S(T_3)\le S(T_2).$$ Отсюда имеем $s(T_1)\le S(T_2)$, что и доказывает утверждение леммы 3.
\par Отметим также, что утверждение леммы 4 по существу вытекает из леммы 3. Действительно, если образуем числовое множество $M_1$, состоящее изо всех значений величин $s(T)$,и множество $M_2$ значений величин $S(T)$, то утверждение леммы 3 означает, что любой элемент $a\in M_2$ есть верхняя грань множества $M_1$, а потому наименьшая верхняя грань множества $M_1$, т.е. $I_{*}$ не превосходит этого элемента $a\in M_1$. Отсюда для любого числа $a\in M_1$ имеем $I_{*}\le a$. Это значит, что $I_{*}$ является нижней гранью множества $M_2$. Но величина $I*$, по своему определению, есть точная нижняя грань множества $M_2$, и потому для любого разбиения $T\in A_P$ имеем $$s(T)\le I_{*}\le I*\le S(T)$$. Лемма 4 доказана.$\blacktriangleright$
\begin{lemma} 
Для $\forall$ $T$ имеем $\Omega (T)\ge I*-I_{*}$. 
Действительно, из леммы~\ref{lemmfour} получим $$\Omega (T)=S(T)-s(T)\ge I*- s(T)\ge I*-I_{*}.$$      
\end{lemma}

\input{10_Mamedguseyn.tex}
\setcounter{theorem}{0}
\setcounter{example}{0}
\setcounter{task}{0}

\chapter{Критерий  равномерной сходимости функциональной последовательности.}
\vskip 7mm
    \centerline{\bf Мамедов Джавид} \vskip 1cm
    Докажем теперь критерий Коши равномерной сходимости функциональной последовательности.
\vskip 5mm

\begin{theorem}
(критерий Коши). Для того чтобы функциональная последовательность  $ A_n(x) $  равномерно сходилась на множестве $M$ , необходимо и достаточно, чтобы для любого $\varepsilon$>0 существовал номер $n_0=n_0(\varepsilon)$
такой, что при всех $m>n_0$ и $n>n_0$ и всех $x\in M$ имело бы место неравенство $|A_n(x)-A_m(x)|<\varepsilon$.
\end{theorem}
\vskip 5mm
$\blacktriangleleft$
{\bfНеобходимость}. В этом случае $A_n(x) \underset{M}{\rightrightarrows} A(x)$.
 Таким образом, для любого $\varepsilon>0$ существует число $n_0=n_0(\varepsilon)$ такое, что для всех $n>n_0$ и для всех $x\in M$ имеем $|A_n(x)-A(x)|<\varepsilon/2$. Но тогда
при $m>n_0$ и $n>n_0$ имеем

\vskip 5mm

$$|A_m(x)-A_n(x)|\le|A_m(x)-A(x)|+|A(x)-A_n(x)|<\varepsilon/2 + \varepsilon/2 =\varepsilon$$,
\vskip 5mm
что и требовалось доказать.
\vskip 5mm
{\bfДостаточность}.При каждом фиксированном $x\in M$ функциональная последовательность $A_n(x)$ превращается в числовую и для нее выполняется критерий Коши. Это значит, что она имеет предел $A(x)$,
 т.е. предельная функция существует на всем множестве $M$. Далее, каково бы ни было число $\varepsilon>0$, по условию найдется номер $n_1=n_1(\varepsilon/2)$ такой, что при всех $m$ и $n>n_1$ имеем
$|A_n(x)-A_m(x)|<\varepsilon/2$.

Снова произвольно зафиксируем $x\in M$ и устремим $m$ к бесконечности. Получим неравенство
\vskip 5mm

$$|A_n(x)-A(x)|\le\varepsilon/2<\varepsilon$$

Но тогда, полагая $n_0=n_0(\varepsilon)=n_1(\varepsilon/2)$, при всех $n>n_0$ и всех $x\in M$ 
будем иметь 

$$|A_n(x)-A(x)|<\varepsilon$$,

 т.е. $A_n(x) \underset{M}{\rightrightarrows} A(x)$. Теорема 1 доказана.
$\blacktriangleright$
Если $A_n(x)$ \texttwelveudash последовательность частичных сумм функционального ряда $\sum a_n(x)$, то теорема 1 дает нам критерий Коши равномерной сходимости этого ряда. Сформулируем его в виде следующей теоремы.
\vskip 5mm
\begin{theorem}
Для равномерной сходимости ряда $\sum a_n(x)$ на множестве $M$ необходимо и достаточно, чтобы для любого $\varepsilon>0$ существовало $n_0=n_0(\varepsilon)$ такое, что для каждого $n>n_0$, и для каждого $p\in \mathbb N$ и для всех $x\in M$ выполнялось бы равенство

$$\Biggl|\sum_{k=n+1}^{n+p}a_k(x)\Biggr|<\varepsilon$$.
\end{theorem}

И наконец, исходя из теоремы 2 сформулируем в прямой форме критерий отсутствия равномерной сходимости ряда $\sum a_n(x)$.
\vskip 5mm

\begin{theorem}
Утверждение о том, что ряд $\sum a_n(x)$ или последовательность $A_n(x)$ не являются равномерно сходящимися на множестве $M$, означает, что существует $\varepsilon>0$ такое, что найдутся две последовательности $\{n_m\}$ и $\{p_m\}\in \mathbb N$, причем $n_{m+1}>n_m$, а также последовательность $\{x_m\}\in M$, для которых имеет место неравенство 

$$\Biggl|\sum_{k=n_m+1}^{n_m+p_m}a_k(x_m)\Biggr|\ge\varepsilon$$.

\end{theorem}
\vskip 5mm

\begin{example}
неравномерно сходящихся рядов и последовательностей.
\vskip 5mm
{\bf1.} Ряд $A(x)=\sum_{n=0}^{\infty}x(1-x)^n$ сходится неравномерно на $[0,2)$.
\vskip 5mm
Действительно, сумма ряда $A(x)$ при $x\ne0$ равна 

$$A(x)=x\sum_{n=0}^{\infty}(1-x)^n=x\frac{1}{1-(1-x)}=\frac{x}{x}=1$$.

и $A(0)=0$. Это значит, что $x=0$ \texttwelveudash точка разрыва функции $A(x)$. Но если бы сходимость была равномерной, то функция $A(x)$ была бы непрерывной в силу теоремы 1\S2, поскольку $a_n(x)=x(1-x)^n$ непрерывна в нуле. Но это не так. Следовательно, равномерной сходимости нет.
\vskip 5mm
{\bf2.}Если $A_n(x)=x^n$, то на множестве $M=(0,1)$ равномерная сходимость не имеет места.\\
Действительно, в теореме 3 положим $\varepsilon=0,1$ и при каждом $m>1$ возьмем $n_m=m$, $x_m=1-\frac{1}{m}$, $p_m=m$. Тогда будем иметь 

$$|A_m(x_m)-A_{2m}(x_m)|=\Biggl|\biggl(1-\frac{1}{m}\biggr)^m-\biggl(1-\frac{1}{m}\biggr)^{2m}\Biggr|=\biggl(1-\frac{1}{m}\biggr)^m \biggl(1-\biggl(1-\frac{1}{m}\biggr)^m\biggr)>$$$$>\frac{1}{3}\cdot\frac{2}{3}
>0,1=\varepsilon.$$
Таким образом, по критерию Коши в форме теоремы 3 последовательность $A_n(x)$ не является равномерно сходящейся.
\end{example}
\vskip 5mm

\begin{task} 
Пусть функции $f_n(x)$ непрерывны на $[0,1]$ при всех $n\in \mathbb N$ и $f_n(x)\to f_0(x)$ при $n\to \infty$. Доказать, что $f_0(x)$ имеет точку непрерывности на $(0,1)$.
\end{task}

%\setcounter{object}{0}
\setcounter{approval}{0}
\setcounter{theorem}{0}
\setcounter{example}{0}

\chapter{Применение формулы Тейлора к некоторым функциям}
\centerline{ \bf Автор: Алиева Эльмира} \vskip 1cm
\quad\upshape{\bfseries{1. Показательная функция: }} $f(x)=e^x$. Имеем
\begin{center}
$f(0)=f'(0)=...=f^{(n)}(0)=1, \quad f^{(n+1)}(x)=e^x$.
\end{center}
Формула Тейлора с остаточным членом в форме Лагранжа принимает вид
\begin{center}
$e^x=1+\dfrac{x}{1!}+\dfrac{x^2}{2!}+...+\dfrac{x^n}{n!}+\dfrac{x^{n+1}}{(n+1)!}e^{\theta x}, \quad 0<\theta <1.$
\end{center}
При любом фиксированном $x$ остаток в ней стремится к нулю, поскольку
\begin{center}
$\lim\limits_{n\to\infty}\dfrac{x^{n+1}}{(n+1)!} = 0$.
\end{center}
\quad\upshape{\bfseries{2. Функция }}$f(x) = \sin x$. Имеем
\begin{center}
$f^{(n)}(x) = \sin\left(x+n \dfrac{\pi}{2}\right)$,
\end{center}
\begin{center}
$f^{(2k+1)}(\theta x)=\sin\left(\theta x+(2k+1) \dfrac{\pi}{2}\right)=(-1)^k \cos\theta x.$
\end{center}
Формула Тейлора с остаточным членом в форме Лагранжа дает
\begin{center}
$\sin x=x-\dfrac{x^3}{3!}+\dfrac{x^5}{5!}-...+(-1)^{k-1}\dfrac{x^{2k-1}}{(2k-1)!}+(-1)^k\dfrac{x^{2k+1}}{(2k+1)!} \cos \theta x$.
\end{center}
\quad\upshape{\bfseries{3. Функция: }} $f(x)=\cos x$. Имеем
\begin{center}
$f^{(n)}(x)=cos\left(x+n \dfrac{\pi}{2}\right)$,
\end{center}
\begin{center}
$f^{(2k)}(\theta x)=\cos\left(\theta x+2k\cdot\dfrac{\pi}{2}\right)=(-1)^k\cos\theta x$.
\end{center}
Тогда
\begin{center}
$\cos x=1-\dfrac{x^2}{2!}+\dfrac{x^4}{4!}-...+(-1)^{k-1}\dfrac{x^{2k-2}}{(2k-2)!}+(-1)^k\dfrac{x^{2k}}{(2k)!}\cos\theta x$.
\end{center}
\quad\upshape{\bfseries{4. Функция: }} $f(x)=\ln(1+x)$. Имеем
\begin{center}
$f'(x)=\dfrac{1}{1+x}, \quad f^{(n)}(x)=(-1)^{n-1}\cdot\dfrac{(n-1)!}{(1+x)^n}$.
\end{center}
Следовательно,
\begin{center}
$\ln(1+x)=x-\dfrac{x^2}{2}+\dfrac{x^3}{3}-...+(-1)^{n-1}\dfrac{x^n}{n}+R_n$,
\end{center}
\begin{center}
$R_n=(-1)^n\dfrac{x^{n+1}}{n+1}\left(\dfrac{1}{1+\theta x}\right)^{n+1}$ 
\end{center}
Заметим, что если $|x|<1$, то $R_n\to{0}$ при $n\to{+\infty}$. Кроме того,\\
\parа) если $0\leq x<1$, то $|R_n|\leq\dfrac{1}{n+1}$;
\parб) если $-1<-r\leq x<0$, то $|R_n|\leq\dfrac{r^{n+1}}{n(1-r)}$, где
\begin{center}
$R_n=\dfrac{(-1)^n}{n}\dfrac{x^{n+1}}{1+\theta x}\left(\dfrac{1-\theta}{1+\theta x}\right)^n$
\end{center}
(остаток в форме Коши).

\upshape{\bfseries{5. Функция: }} $f(x)=(1+x)^\alpha$. Имеем
\begin{center}
$f^{(n)}=\alpha (\alpha -1)...(\alpha -n+1)(1+x)^{\alpha -n}$,
\end{center}
поэтому
\begin{center}
$(1+x)^\alpha =1+\alpha x+\dfrac{\alpha(\alpha -1)}{2} x^2+\dfrac{\alpha(\alpha -1)(\alpha -2)}{3!}x^3+...$\\
$+\dfrac{\alpha(\alpha -1)...(\alpha -n+1)}{n!}x^n+R_n$,
\end{center}
где 
\begin{center}
$R_n=\dfrac{\alpha(\alpha -1)...(\alpha -n)}{(n+1)!}x^{n+1}(1+\theta_1x)^{\alpha-n-1}, \quad  0<\theta_1<1$
\end{center}
(остаток в форме Лагранжа),
\begin{center}
$R_n=\dfrac{\alpha(\alpha -1)...(\alpha -n)}{(n+1)!}x^{n+1}(1+\theta_2x)^{\alpha-1}(\dfrac{1-\theta_2}{1+\theta_2x})^n, \quad 0<\theta_2<1$
\end{center}
(остаток в форме Коши). Если $|x|<1$, то $R_n \to 0$ при $n \to \infty.$
\\ 
\vskip 2mm
Другими словами,
\begin{center}
$\lim\limits_{n\to\infty} f_n(0,x)=f(x) = 0$.
\end{center}
Это предельное выражение символически записываетя так:
\begin{center}
$f(x)=f(a)+\dfrac{f'(a)}{1!}(x-a)+...+\dfrac{f^{(n)}(a)}{n!}(x-a)^n+...$ .
\end{center}
\vskip 2mm
и называется \upshape{\bfseries{рядом Тейлора функции}} $f(x)$  вточке $x=a$.

Заметим, что при всех $n\in N$ для $n$-го члена ряда имеет место равенство
\begin{center}
$\dfrac{f^{(n)}(a)}{n!}(x-a)^n=\dfrac{{\rm d}^nf(x)}{n!}=\left. \dfrac{{\rm d}^nf(x)}{n!}\right|_{\overset{x=a}{\triangle x=x-a}} $.
\end{center}

Поэтому ряд Тейлора можно переписать в следующем виде
\begin{center}
$\Delta f=\dfrac{{\rm d}f}{1!}+\dfrac{{\rm d}^2f}{2!}+...+\dfrac{{\rm d}^nf}{n!}+...$ .
\end{center}
Тем самым определен точный смысл равенства, приведенного ранее в лекции 18, \S4.
\\
\vskip 1mm
\slshape{Замечание.} Ряд Тейлора не всегда сходится к породившей его функции.
\vskip 3mm
\upshape{\bfseries{Пример.}}

\begin{equation*}
f(x) =
 \begin{cases}
   $e$^{-\tfrac{1}{x^2}}, &\text{если $x \ne 0$,}\\
   0, &\text{если $x=0$.}
 \end{cases}
\end{equation*}
Тогда при любом натуральном $k$ имеем
\begin{center}
$f^{(k)}(0)=0.$
\end{center}

Таким образом, мы видим, что рядом Тейлора нулевой, а породившая его функция отлична от тождественного нуля.




























\setcounter{object}{0}
\setcounter{approval}{0}
\setcounter{theorem}{0}
\setcounter{example}{0}


\chapter{Замечательные пределы}
\centerline{ \bf Автор: Гусейн-заде Сакина } \vskip 1cm

\begin{approval}
\slshape{Имеют место соотношения:}

\slshape{a)}
$\lim\limits_{x\to\infty} (1+\frac{1}{x})^{x}=e;$


\slshape{б)}
$\lim\limits_{x\to\ 0}(1+x)^{1/x} =e;$


\slshape{в)}
$\lim\limits_{x\to\ 0}\frac{\ln(1+x)}{x} =1;$


\slshape{г)}
$\lim\limits_{x\to\ 0}\frac{e^{x}-1}{x} =1;$

\end{approval}

$\blacktriangleleft$
 \upshape{a) Рассмотрим сначала случай $\it x  \rightarrow  +\infty.$}  
  \upshape{ В силу свойства монотонности показательной функции справедливы неравенства}
 
 \begin{center}
 $
 \left(1+\dfrac{1}{[x]+1}\right) ^{[x]}<\left(1+\dfrac{1}{x}\right) ^{x}<\left(1+\dfrac{1}{[x]}\right) ^{[x]+1}
 $.
 \end{center}
Но мы знаем, что
 
  \begin{center}
  $
\lim\limits_{n\to\infty}\left(1+\dfrac{1}{n}\right) ^{n}=e.
  $
  \end{center}
Отсюда  
   \begin{center}
   $
 \lim\limits_{n\to\infty}\left(1+\dfrac{1}{n+1}\right)^{n}=e, $   $\lim\limits_{n\to\infty}\left(1+\dfrac{1}{n}\right)^{n+1}=e.
   $
   \end{center}
т.е. справедливы утверждения
   \begin{center}
   $
\forall\: \: \varepsilon\: >\: 0   \:\:  \exists \:\:  N_1(\varepsilon ):\forall\: \:  n>N_1 \: \: \Rightarrow \: \:  \left | \left(1+\dfrac{1}{n+1}\right)^{n}-e \right |<\varepsilon ;
   $
   
   $
\:\:\:\:\:\:\:\:\:\:\:\:\:\:  \exists \:\:  N_2(\varepsilon ):\forall\: \:  n>N_2 \: \: \Rightarrow \: \:  \left | \left(1+\dfrac{1}{n}\right)^{n+1}-e \right |<\varepsilon.
   $
\end{center}

Тогда при $n>\:\:max(N_1, N_2)$ имеем

\begin{center}
$e-\varepsilon <\left(1+\dfrac{1}{n+1} \right )^{n}<e+\varepsilon;$
$e-\varepsilon <\left(1+\dfrac{1}{n} \right )^{n+1}<e+\varepsilon.$
\end{center}   
Если $x>1+max(N_1, N_2)=N$, то $[x]>max(N_1, N_2)=N  -  1$.
 Следовательно, при $x>N$ справедливы неравенства
 
\begin{center}
$
e-\varepsilon <\left(1+\dfrac{1}{[x]+1} \right )^{[x]}\left(1+\dfrac{1}{x} \right )^{x}<\left(1+\dfrac{1}{[x]} \right )^{[x]+1}<e+\varepsilon.
$
\end{center} 
Таким образом, получим
\begin{center}
 $
\forall\: \: \varepsilon\: >\: 0   \:\:  \exists \:\:  N(\varepsilon ):\forall\: \:  x>N \: \: \Rightarrow \: \:  \left | \left(1+\dfrac{1}{x}\right)^{x}-e \right |<\varepsilon ;
   $
\end{center} 
Это значит, что $\left(1+\dfrac{1}{x}\right)^{x} \rightarrow  +\infty $ при $ x \rightarrow  +\infty $. 

\upshape{Рассмотрим теперь случай $x \rightarrow -\infty $. Положим $y=-x$. Тогд,а используя теорему 4 \S\:6 гл. III о пределе сложно функции, будем иметь}
\begin{center}
$
e= \lim\limits_{y\to+\infty} \left(1+\dfrac{1}{y-1}\right)^{y} = \lim\limits_{y\to+\infty} \left(\dfrac{y}{y-1}\right)^{y} = \lim\limits_{y\to+\infty} \left(1-\dfrac{1}{y}\right)^{-y}= \lim\limits_{y\to-\infty} \left(1+\dfrac{1}{x}\right)^{x}.
$
\end{center}
Соединяя вместе случаи $x \rightarrow +\infty $ и $x \rightarrow -\infty $, приходим к соотношению 
\begin{center}
$
 \lim\limits_{x\to\infty} \left(1+\dfrac{1}{x}\right)^{x} = e.
$
\end{center}
Утверждение а) доказано.

\upshape{б) Для доказательства соотношения $\lim\limits_{x\to\ 0} (1+x)^{1/x} = e$ воспользуемся той же теоремой 4 \S\:6 гл. III. Полагая $x=1/y$, получим}
\begin{center}
$
 e=\lim\limits_{y\to\infty} \left(1+\dfrac{1}{y}\right)^{y} =  \lim\limits_{x\to\infty} \left(1+x\right)^{1/x}.
$
\end{center}

\upshape{в) Так как}
\begin{center}
$(1+x)^{1/x}=e^{\frac{ln(1+x)}{x}}\rightarrow e$ при $x\rightarrow 0,$
\end{center}
то из непрерывности и монотонности функции $y=e^{x}$ следует, что

\begin{center}
$ e=\lim\limits_{x\to\ 0} \dfrac{\ln(1+x)}{x} = 1.$
\end{center}


\upshape{г) Вновь воспользуемся теоремой о пределе сложной функци, полагая}
\begin{center}
$ g(x)=e^{x}-1 \rightarrow 0 $ при $x \rightarrow 0,$

$ f(y)=\dfrac{\ln(1+y)}{y} \rightarrow 1 $ при $y \rightarrow 0,$
\end{center}
и, кроме того, $f(0)=1.$

\upshape{Тогда имеем $f(g(x))=\frac{x}{e^{x}-1} \rightarrow 1$ при $x \rightarrow 0$. Отсюда следует утверждение г).}

\upshape{Утверждение 1 полностью доказано.}  $\blacktriangleright$

\begin{approval}
$ \lim\limits_{x\to\ 0} \dfrac{\sin(x)}{x} = 1.$
\end{approval}
$\blacktriangleleft$ При $0<x<\pi /2$ рассмотрим сектор единичного круга, отвечающего душе длины $х$, и два треугольника, один из которых вписан в сектор, в второй, прямоугольный, содержит его, имея с ним общий угол и сторону на оси абсцисс. Сравнивая площади этих фигур, имеем
\begin{center}
$\dfrac {\sin{x}}{2}<\dfrac {x}{2}<\dfrac {\tg{x}}{2}$
\end{center}
Отсюда получим
\begin{center}
$\cos{x}<\dfrac {\sin{x}}{2}<1.$
\end{center}

\upshape{Последние неравенства связывают четные функции, поэтому они имеют место при $0<|x|<\pi/2$. Так как cos $x$ - непрерывная функция, то по теореме о переходе к пределу в неравенствах имеем}

\begin{center}
$\lim\limits_{x\to\ 0} \dfrac{\sin(x)}{x} = 1.$
\end{center}
Доказательство закончено.  $\blacktriangleright$

\bf{Примеры вычисления пределов.}

\bf{1.} \rm $\lim\limits_{x\to\ 0} \frac{(1+x)^{\alpha}-1}{x} = \alpha.$

\begin{center}
$ \dfrac{(1+x)^{\alpha}-1}{x}  =  \dfrac{e^{\alpha\ln(1+x)}-1}{x} = \dfrac{e^{\alpha x+o(x)}-1}{x}=$

 $= \dfrac{1+\alpha x + o(x)-1}{x} = \alpha+o(1) \rightarrow \alpha$ при $x \rightarrow 0.$

\end{center}
Этот прием называется заменой бесконечно малой функции на эквивалентную ей.


\bf{2.} \rm $\lim\limits_{x\to\ 0} \frac{1-\cos x}{x^2} = \frac{1}{2}.$

\begin{center}
$ \dfrac{1-\cos x}{x^2}  =  \dfrac{2\sin^2 \frac{x}{2}}{x^2} = \dfrac{2(\frac{x}{2}+0(x^2))^2}{x^2}= = \dfrac{\frac{x^2}{2}+o(x^2)}{x^2} = \dfrac{1}{2}+o(1). $
\end{center}
Таким образом:

1) $(1+x)^\alpha=1+\alpha x + o(x)$ при $x \rightarrow  0;$

1) $\cos x=1- \frac{x^2}{2} + o(x^2)$ при $x \rightarrow  0;$

1) $\lim\limits_{n\to+\infty} \left(1+\frac{x}{n}\right)^{n} = e^{x}.$ Положим $x_n=\frac{x}{n} \rightarrow 0$ при $n \rightarrow  \infty.$ Тогда по теореме о пределе сложной функции имеем
\begin{center}
$ \lim\limits_{n\to\infty} \left(1+\frac{x}{n}\right)^{n} = \lim\limits_{n\to\infty} \left((1+x_n)^{1/x_n}\right)^{x} = e^{x\lim_{n\to\infty} \dfrac{\ln(1+x_n)}{x_n}}= e^{x}  $
\end{center}


\setcounter{chapter}{12}
\chapter{Истоки алгебры}
\section{Алгебра вкратце}
\section{Некоторые }
\setcounter{section}{2}
\section{Определители небольших порядков.}
{\vskip 4mm
\centerline{\bf Гайыбов Ниджат} \vskip 1cm
 Излагая метод Гаусса, мы не слишком заботились о значениях коэффициентов при главных неизвестных. Важно было лишь то, что эти коэффициенты отличны от нуля. Проведем теперь более аккуратно процесс исключения неизвестных хотя бы в случае квадратных линейных систем небольших размеров. Это даст пищу для размышлений и исходный материал для построения общей теории определителей в гл. 3.


Как и в \S3, рассмотрим систему двух уравнений с двумя неизвестными 
\begin{equation}
\label{f1}
\begin{matrix}
a_{11}x_{1}+a_{12}x_{2}=b_{1}, \\
a_{21}x_{1}+a_{22}x_{2}=b_{2} 
\end{matrix}
\end{equation}
  и постараемся найти общие формулы для компонент $x_1^0,x_2^0$ ее решения.


Назовем {\sl определителем} матрицы $ \begin{Vmatrix} a_{11}& a_{12}\\a_{21}& a_{22}\end{Vmatrix}$ выражение $a_{11} a_{22}-a_{21} a_{12}$ и обозначим его 
$\begin{vmatrix}a_{11}& a_{12}\\a_{21}& a_{22}\end{vmatrix}.$\\
Тем самым квадратной матрице сопоставляется число
\begin{equation}
\label{f2}
\begin{vmatrix}a_{12}&a_{12}\\ a_{21}&a_{22}\end{vmatrix}=a_{11}a_{22}-a_{21}a_{12}.
\end{equation}
Если мы попытаемся исключить $x_{2}$ из системы ($\ref{f1}$), умножив первое уравнение на $a_{22}$ и прибавив к нему второе,умноженное на -$a_{12}$, то получим 
$$\begin{vmatrix}
a_{11}&a_{12}\\
a_{21}&a_{22}
\end{vmatrix}
x_{1}
=b_{1}a_{22}-b_{2}a_{12}.
$$ 
Правую часть также можно рассматривать как определитель матрицы $\begin{Vmatrix}b_{1}&a_{12}\\ a_{21}&a_{22}\end{Vmatrix}.$ Предположим, что $\begin{vmatrix}a_{11}&a_{12}\\ a_{21}&a_{22}\end{vmatrix}\ne0.$ Тогда мы имеем 
\begin{equation}
\label{f3}
x_1= \frac{\begin{Vmatrix}b_1 & a_{12}\\ b_2 & a_{22}\end{Vmatrix}}{\begin{vmatrix} a_{11} & a_{12} \\ a_{21} & a_{22} \end{vmatrix}}, \qquad
x_2=\frac{\begin{vmatrix}a_{11} & b_{1}\\ a_{21} & b_2\end{vmatrix}}{\begin{vmatrix} a_{11} & a_{12} \\ a_{21} & a_{22} \end{vmatrix}}
\end{equation}


Имея формулы для решения системы двух линейных уравнений с двумя неизвестными,мы можем решать и некоторые другие системы(решать системы-значить находить их решения).Рассмотрим,например систему двух однородных уравнений с тремя неизвестными 
\begin{equation}
\label{f4}
\begin{matrix}
a_{11}x_{1}+a_{12}x_{2}+a_{13}x_{3}=0, \\
a_{21}x_{1}+a_{22}x_{2}+a_{23}x_{3}=0.
\end{matrix}
\end{equation}
Нас интересует ненулевое решение этой системы, так что хотя бы одно из $x_{i}$ не равно нулю.Пусть, например, $x_{3}\ne0.$ Разделив обе части на $-x_{3}$ и положив
$y_{1}=-x_{1}/x_{3}$, $y_{2}=-x_{2}/x_{3}$, запишем систему ($\ref{f4}$) в том же виде 
$$
\begin{matrix}
a_{11}y_{1}+a_{12}y_{2}=a_{13}, \\
a_{21}y_{1}+a_{22}y_{2}=a_{23}.
\end{matrix}
$$
что и (1). При предположении $\begin{vmatrix}a_{11}&a_{12}\\a_{21}&a_{22}\end{vmatrix}\ne0$ формулы ($\ref{f3}$) дают 
$$
y_{1}=-\frac{x_1}{x_3}=\frac{\begin{vmatrix}a_{13}&a_{12}\\a_{23}&a_{22}\end{vmatrix}}{\begin{vmatrix}a_{11}&a_{12}\\a_{21}&a_{22}\end{vmatrix}},\qquad
y_{2}=-\frac{x_2}{x_3}=\frac{\begin{vmatrix}a_{11}&a_{12}\\a_{21}&a_{23}\end{vmatrix}}{\begin{vmatrix}a_{11}&a_{12}\\a_{21}&a_{22}\end{vmatrix}}
$$


Неудивительно, что мы нашли из системы (4) не сами $x_1$, $x_2$, $x_3$, а только их отношения: из однородности системы легко следует, что если $(x_1^0,x_2^0,x_3^0)$---решение и $c$---любое число, то $(cx_1^0,cx_2^0,cx_3^0)$ тоже будет решением. Поэтому мы можем положить 
\begin{equation}
\label{f5}
x_{1}=-\begin{vmatrix}a_{13}&a_{12}\\a_{23}&a_{22}\end{vmatrix},\qquad
x_{2}=-\begin{vmatrix}a_{11}&a_{13}\\a_{21}&a_{23}\end{vmatrix},\qquad
x_{3}=\begin{vmatrix}a_{11}&a_{12}\\a_{21}&a_{22}\end{vmatrix}.
\end{equation}
и сказать, что любое решение получается из указанного умножением всех $x_{i}$ на некоторое число $c$. Чтобы придать ответу несколько более симметричный вид, заметим, что всегда $$ \begin{vmatrix}a&b\\c&d\end{vmatrix}=-\begin{vmatrix}b&a\\d&c\end{vmatrix},$$ как это непосредственно видно из формулы ($\ref{f2}$). Поэтому ($\ref{f5}$) можно записать в виде 
\begin{equation}
\label{f6}
x_{1}=\begin{vmatrix}a_{13}&a_{12}\\a_{23}&a_{22}\end{vmatrix},\qquad
x_{2}=-\begin{vmatrix}a_{11}&a_{13}\\a_{21}&a_{23}\end{vmatrix},\qquad
x_{3}=\begin{vmatrix}a_{11}&a_{12}\\a_{21}&a_{22}\end{vmatrix}.
\end{equation}
Эти формулы выведены в предположении, что $\begin{vmatrix}a_{11}&a_{12}\\ a_{21}&a_{22}\end{vmatrix}\ne0.$ Нетрудно проверить, что доказанное утверждение верно, если хоть один из выражения ($\ref{f6}$) определителей отличен от нуля. Если же все три определителя равны нулю, то, конечно, формулы (\ref{f6}) дают
решение (а именно нулевое), но мы не можем утверждать, что все решения получаются из него умножением на число. 


\setcounter{chapter}{13}
\chapter{}
\setcounter{object}{0}
\setcounter{approval}{0}
\setcounter{example}{0}
\begin{center}

\vskip 6mm
{\S3. ПРЕДЕЛ ПОСЛЕДОВАТЕЛЬНОСТИ}
\\Автор: Керимов Руслан
\end{center} \vskip 2mm
\begin{object}
\slshape Последовательность {\upshape\{$a_n$\}} называется {\bfseries\upshape сходящейся},
если существует число {\upshape$l\in\R$} такое, что последовательность {\upshape$a_n=~a_n-~l$}
является бесконечно малой последовательностью.
\end{object}

В этом случае говорят, что $\{a_n\}$ сходится или что $\{a_n\}$ имеет предел
и этот предел равен $l$. Записывают это так:
$$
\lim_{n\to\infty}a_n=l\text{ или }a_n\to l\text{ при }n\to\infty.
$$

Это определение на ``$\varepsilon$-языке'' можно записать следующим образом:
$$
\forall\varepsilon>0\enskip\exists n_0(\varepsilon),\text{ такое, что }
\forall n>n_0\text{ имеем }|a_n-l|<\varepsilon.
$$

Будем говорить также, что последовательность $\{a_n\}$ расходится к ``плюс
бесконечности'', если для любого $c>0$ лишь для конечного числа членов ее
выполняется неравенство: 
$$
a_n<c.
$$
Обозначается это так:
$$
\lim_{n\to\infty}a_n=+\infty\text{ или }a_n\to+\infty\text{ при }n\to\infty.
$$
Последовательность $\{a_n\}$ расходится к ``минус бесконечности'', если для
любого $b<0$ лишь для конечного числа членов ее выполняется неравенство
$$
a_n>b.
$$
Обозначается это так:
$$
\lim_{n\to\infty}a_n=-\infty\text{ или }a_n\to-\infty\text{ при }n\to\infty.
$$
И, наконец, последовательность $\{a_n\}$ расходится к ``бесконечности'',
если для любого $c>0$ лишь для конечного числа членов ее выполняется неравенство
$$
|a_n|<c.
$$
Обозначается это так:
$$
\lim_{n\to\infty}a_n=\infty\text{ или }a_n\to\infty\text{ при }n\to\infty.
$$

\begin{approval}.Если $\{a_n\}$ сходится, то она имеет единственный предел.
\end{approval}
$\blacktriangleleft$
Пусть это не так. Тогда существуют числа $l_1\ne l_2$ такие, что
последовательности $\alpha_n=a_n-l_1$ и $\beta_n=a_n-l_2$
обе являются бесконечно малыми последовательностями. Отсюда
$\alpha_n+l_1=a_n=\beta_n+l_2$, поэтому $l_2-l_2=\beta_n-\alpha_n$
есть бесконечная малая последовательность. Но тогда по теореме 5 \S2
имеем $l_1-l_2=0$, т.е. $l_1=l_2$.
$\blacktriangleright$

\begin{approval}Если $\{a_n\}$ --- бесконечно малая последовательность, то\\ 
$\lim\limits_{n\to\infty}a_n=0$.
\end{approval}
$\blacktriangleleft$
Действительно, при $l=0$ имеем $\alpha_n-0=a_n$ есть бесконечно малая
последовательность, т.е. предел $\{a_n\}$ при $n\to\infty$ равен $0$.
$\blacktriangleright$

\begin{approval}Если $\{a_n\}$ сходится, то она ограничена.
\end{approval}
$\blacktriangleleft$
Если $\{a_n\}$ сходится, то найдется число $l$ такое, что $\alpha_n=a_n-l$
--- бесконечно малая последовательность. Значит, существует $c>0$ такое,
что при всех натуральных $n$ имеем $|\alpha_n|<c$. Но $a_n=l+\alpha_n$, откуда
$$
|a_n|=|l+\alpha_n|\le|l|+|\alpha_n|\le|l|+c=c_1,
$$
т.е. $\{a_n\}$ --- ограниченная последовательность, что и требовалось доказать.
$\blacktriangleright$

\begin{approval}Если $\lim\limits_{n\to\infty}a_n=l$  и $a_n\ne0$, то существует
$a_0\in\N$, такое, что при всех $n>n_0$ имеем $|a_n|>|l|/2$ (или, что тоже самое)
$1/|a_n|<2/|l|)$.

Это означает, что последовательность $1/a_n$, составленная из обратных величин,
ограничена.
\end{approval}

$\blacktriangleleft$
В силу условия имеем, что $\alpha_n=a_n-l$ --- бесконечно малая последовательность.
Тогда вне $|l|/2$-окресности нуля лежит только конечное число членов последовательности
$\{a_n\}$. Пусть $n_0$ --- самое большое значение номера таких членов; тогда при всех
$n>n_0$ имеем $|\alpha_n|<|l|/2$. Отсюда при этих $n$ получим ($l=a_n-\alpha_n$)
$$
|l|=|a_n-\alpha_n|\le|a_n|+|-\alpha_n|=|a_n|+|\alpha_n|.
$$
Следовательно,
$$
|a_n|\ge|l|-|\alpha_n|>|l|-\cfrac{|l|}{2}=\cfrac{|l|}{2},
$$
что и требовалось доказать.
$\blacktriangleright$

\begin{approval}Если $a_n\to l_1$, $b_n\to l_2$ при $n\to\infty$, то 
$c_n=a_n\pm b_n\to l_1\pm l_2$ при $n\to\infty$. Другими словами, 
для сходящихся последовательностей предел их суммы равен сумме их пределов.
\end{approval}
$\blacktriangleleft$
Из условия имеем $\alpha_n=a_n-l_1$, $\beta_n=b_n-l_2$ --- бесконечно малые 
последовательности. Следовательно,
$$
c_n-(l_1\pm l_2)=(a_n\pm b_n)-(l_1\pm l_2)=\alpha_n\pm \beta_n=\gamma_n
$$
--- бесконечно малая последовательность. Значит, из определения предела имеем
$$
\lim_{n\to\infty}c_n=l_1\pm l_2,
$$
что и требовалось доказать.
$\blacktriangleright$

\begin{approval}Если $a_n\to l_1$, $b_n\to l_2$ при $n\to\infty$, то 
$c_n=a_nb_n=l_1l_2$ при $n\to\infty$ (предел произведения равен 
произведению пределов).
\end{approval}
$\blacktriangleleft$
Имеем $a_n=l_1+\alpha_n$, $b_n=l_2+\beta_n$, $c_n=a_nb_n=l_1l_2+\alpha_n l_2+
\beta_n l_1+\alpha_n\beta_n=l_1l_2+\gamma_n$. Но $\gamma_n$ --- бесконечно
 малая последовательность, так как она есть сумма трех последовательностей,
 каждая из которых есть бесконечно малая последовательность. Отсюда:
 $$
 \lim_{n\to\infty}c_n=l_1l_2.
 $$
 Доказательство закончено.
$\blacktriangleright$

\begin{approval}Пусть $\lim\limits_{n\to\infty}a_n=l_1$, $\lim\limits_{n\to\infty}
b_n=l_2$, $l_2\ne0$. Тогда $\lim\limits_{n\to\infty}\cfrac{a_n}{b_n}=
\cfrac{l_1}{l_2}$, т.е. если предел знаменателя не равен нулю, то предел
отношения равен отношению пределов.
\end{approval}
$\blacktriangleleft$
Рассмотрим последовательности $c_n=\cfrac{a_n}{b_n}$ и $\gamma_n=
c_n-\cfrac{l_1}{l_2}=\cfrac{a_n}{b_n}-\cfrac{l_1}{l_2}=\cfrac{a_nl_2-b_n
l_1}{b_nl_2}$, $a_n=l_1+\alpha_n=a_n-l_1$, $\beta_n=b_n-l_2$.
Из условия вытекает, что $\alpha_n$, $\beta_n$ есть бесконечно малая 
последовательность. Нам достаточно доказать, что тоже является бесконечно
малая последовательность. Для этого запишем $\gamma_n$ в виде
$$
\gamma_n=\cfrac{(l_1+\alpha_n)l_2-(l_2+\beta_n)l_1}{b_nl_2}=
\cfrac{\alpha_nl_2-\beta_nl_1}{l_2}\cdot\cfrac{1}{b_n}.
$$
Теперь заметим, что последовательность $\cfrac{\alpha_nl_2-\beta_nl_1}{l_2}$
является бесконечно малой в силу утверждений $5$ и $6$, а 
последовательность $1/b_n$ ограничена в силу утверждения $4$. Но тогда по
теореме $4$ \S2 последовательность $\gamma_n$ является бесконечно малой.
Таким образом, $\lim\limits_{n\to\infty}c_n=l_1/l_2$, что и требовалось
доказать.
$\blacktriangleright$

\vskip 3mm
\begin{example} \itshape Сумма членов бесконечной убывающей 
геометрической прогрессии. \linebreak[4] Пусть $s_n=a+aq+\ldots+aq^{n-1}$. Тогда
$$
qs_n=aq+\ldots+aq^{n-1}+aq^n,\;s_n=\cfrac{a-aq^n}{1-q}.
$$
Так как при $|q|<1$ имеем $\{q^n\}$ --- бесконечно малая последовательность, то
$$
s=\lim_{n\to\infty}s_n=\cfrac{a}{1-q}.
$$

Заметим, что величину $s$ можно представить в виде
$$
s=\lim_{n\to\infty}s_n=\frac{a}{1-q}.
$$
где $s_n=\sum_{k=1}^naq^{k-1}$ называется $n$-{\bfseries й частичной 
суммой ряда,} а величина $r_n=s-s_n$ --- {\bfseries остатком ряда.}
\end{example}

\chapter{Теорема о длине дуги кривой}
\centerline{Алиса Зинина}
\vskip 5mm
\begin{theorem} Пусть функции $\varphi_1(t),\ldots,\varphi_m(t)$, задающие простую кривую $L$, имеют непрерывные производные на отрезке $[a,b]$. Тогда кривая $L$ $-$ спрямляема и ее длина $|L|$ выражается формулой
\vskip 5mm
$$|L|=\int_a^b{\sqrt{(\varphi_1'(t))^2+\cdots+(\varphi_m'(t))^2}dt}.$$
\end{theorem}
\vskip 5mm
$\blacktriangleleft$ Покажем сначала, что длина любой ломаной не превосходит величины
\vskip 5mm
$$A=\int_a^b\sqrt{(\varphi_1'(t))^2+\cdots+(\varphi_m'(t))^2}dt.$$
\vskip 5mm
Пусть узлы ломаной $l$ соответствуют точкам $t_0,t_1,\cdots,t_n$ разбиения $T$ отрезка $[a,b]:a=t_0<t_1<\cdots<t_n=b. $ Тогда имеем $$|l|=\sum_{s=1}^n\sqrt{(\varphi_1(t_s)-\varphi_1(t_{s-1}))^2+\cdots+(\varphi_m(t_s)-\varphi_m(t_{s-1}))^2}=$$
$$=\sum_{s=1}^n\sqrt{\left(\int_{t_{s-1}}^{t_s}{\varphi_1'(t)dt}\right)^2+\cdots+\left(\int_{t_{s-1}}^{t_s}{\varphi_m'(t)dt}\right)^2}.$$
Далее, в силу неравенства (см. гл. VIII, $\S6$, теорема 3)
$$\sqrt{\left(\int_a^b{f_1(t)dt}\right)^2+\cdots+\left(\int_a^b{f_m(t)dt}\right)^2}\le\int_a^b{\sqrt{f_1^2(t)+\cdots+f_m^2(t)}dt}$$
получим
$$|l|\le\sum_{s=1}^n\int_{t_{s-1}}^{t_s}{\sqrt{(\varphi_1'(t))^2+\cdots+(\varphi_m'(t))^2}dt}=A$$$ 
\blacktriangleright$

Таким образом мы доказали, что $A$ --- верхняя грань длин всех ломаных $l$, вписанных в кривую $L$, т.е. кривая $L$ является спрямляемой.

Покажем, что $A$ есть {\itshape точная} верхняя грань длин таких ломаных, т.е. длина кривой $L$ равна $A$.

Поскольку функции $\varphi_k'(t)$, $k=1,\ldots,m$, непрерывны на отрезкен $[a,b]$, но по теореме Гейне-Кантора они являются равномерно непрерывными на этом отрезке. Следовательно, при всех $k=1,\ldots,m$ для всякого $\varepsilon>0$ существует число $\delta=\delta(\varepsilon)>0$ такое, что для всех $t',t'' \in [a,b]:|t'-t''|<\delta$ выполняется неравенство
$$|\varphi_k'(t')-\varphi_k'(t'')|<\dfrac{\varepsilon}{\sqrt{m}(b-a)}=\varepsilon_1.$$

Возьмем любое разбиение $T$ отрезка $[a,b]$ с диаметром $\Delta_T<\delta, T: a=t_0<t_1\ldots<t_n=b$, и пусть ломаная $l$ соответствует этому разбиению $T$.

Оценим теперь сверху разность $A-|l|\ge0.$ Имеем
$$A-|l|=\int_a^b{\sqrt{(\varphi_1'(t))^2+\cdots+(\varphi_m'(t))^2}dt}-\sum_{s=1}^n\sqrt{\sum_{k=1}^m(\varphi_k(t_s))-\varphi_k(t_{s-1}))^2}=$$
$$=\sum_{s=1}^n\int_{t_{s-1}}^{t_s}{\left(\sqrt{\sum_{k=1}^m(\varphi_k'(t))^2}-\sqrt{\sum_{k=1}^m\left(\dfrac{\Delta\varphi_k(t_{s-1})}{\Delta t_{s-1}}\right)^2}\right)dt},$$

где $\Delta\varphi_k(t_{s-1})=\varphi_k(t_s)-\varphi_k(t_{s-1}), \Delta t_{s-1}=t_s-t_{s-1}.$
Далее применим неравенство треугольника в следующем виде. Пусть заданы вершины $O(0,\ldots,0), A(a_1,\ldots,a_m),\\ B(b_1,\ldots,b_m)$ треугольника $OAB$. Тогда имеет место неравенство треугольника (неравенство Минковского при $p=2$):
$$\left|{\sqrt{\sum_{k=1}^m a_k^2}-\sqrt{\sum_{k=1}^m b_k^2}}\right| \le \sqrt{\sum_{k=1}^m(a_k-b_k)^2}.$$
Следовательно, получим
%$$A-|l|\le\sum_{s=1}^n\int_{t_{s-1}}^{t_s}{\sqrt{\sum_{k=1}^m\left(\varphi_k'(t)-\dfrac{\Delta\varphi_k(t_{s-1})}{\Delta t_{s-1}}\right)^2 }dt$$
$$A-|l|\le\sum_{s=1}^n\int_{t_{s-1}}^{t_s} {\sqrt{\sum_{k=1}^m{\left(\varphi_k'(t) -\dfrac{\Delta \varphi_k(t_{s-1})}{\Delta t_{s-1}}\right)^2}}dt}=$$
$$=\sum_{s=1}^n\int_{t_{s-1}}^{t_s}\sqrt{\sum_{k=1}^m{\left(\varphi_k'(t) -\varphi_k'(\eta_{k,s})\right)^2}}dt<$$
$$<\sum_{s=1}^n\int_{t_{s-1}}^{t_s}{\varepsilon_1 \sqrt{m} dt }= \varepsilon_1 \sqrt{m} (b-a)=\varepsilon,$$
где $\eta_{k,s} -$ некоторая точка отрезка $[t_{s-1},t_s]$. Теорема доказана.$\blacktriangleright$

\begin{sledstvie} Пусть $s=s(u)-$ длина дуги кривой $L$, задаваемой уравнениями $x_1=x_1(t),\ldots,x_m=x_m(t), a\le t\le u$. Тогда для дифференциала длины дуги кривой $ds$ справедлива формула
$$(ds)^2=(dx_1)^2+\cdots+(dx_m)^2.$$
\end{sledstvie}
$\blacktriangleleft$ Из теоремы имеем
$$s(u)=\int_a^u{\sqrt{(x_1'(t))^2+\cdots+(x_m'(t))^2}}dt.$$
Дифференциируя это выражение, найдем
$ds(u)=\sqrt{(x_1'(u))^2+\cdots+(x_m'(u))^2}du$, или $ds=\sqrt{(dx_1)^2+\cdots+(dx_m)^2}$.
Следствие доказано.$\blacktriangleright$

Отсюда имеем, что квадрат дифференциала длины дуги плоской кривой $(m=2)$ в полярных координатах $(r,\varphi)$ равен
$$ds^2=dr^2+r^2 d\varphi^2,$$
где координаты $r$ и $\varphi$ определяются по формулам:
$$x=r\cos\varphi,\quad y=r\sin\varphi,\quad r\ge0,\quad 0\le\varphi<2\pi.$$
Действительно,
$$dx=\cos\varphi dr - r \sin\varphi d \varphi, \quad dy=\sin\varphi dr + r\cos\varphi d\varphi.$$
\vskip 4mm
Следовательно, получим
\vskip 4mm
$$ds^2 = dx^2 + dy^2 = dr^2 +r^2d\varphi^2.$$
\vskip 4mm
В частности, если уравнение эллипса задано в параметрической форме 
\vskip 3mm
$$\overline{r}=\overline{r}(\varphi)=(a\cos\varphi, b\sin\varphi),\quad a\ge b >0, $$
\vskip 3mm
и угол $\varphi$ между осью $Ox$ и радиус-вектором $\overline{r}$ изменяется от нуля до $2\pi$, то для дифференциала длины дуги эллипса имеем
\vskip 3mm
$$ds = \sqrt{a^2\sin^2\varphi + b^2\cos^2\varphi}d\varphi$$
\vskip 3mm
\begin{example} Найти длину дуги кривой (циклоиды)
\begin{equation}
\left\{
\begin{array}{l}
\ds x= \quad R(t-\sin t), \\
\ds y= \quad R(1- \cos t),
\end{array}
\right.
\end{equation}\
\vskip 3mm
где $0 \le t \le \theta \le 2\pi$. Имеем: $ds^2=dx^2 + dy^2, \quad ds^2=4R^2sin^2\dfrac{t}{2}(dt)^2, \quad ds= \\
2Rsin(\dfrac{t}{2})dt.$ Следовательно,
\vskip 4mm
$$s=s(\theta)=4R(1-cos\dfrac{\theta}{2}).$$
\vskip 4mm
Заметим, что эта кривая задает траекторию движения точки на ободе катящегося колеса радиуса $R$.
\end{example}















\chapter{Формула Тейлора с остаточным членом в общей форме} 
\centerline{ Лекция Виктора Башарова}
\vskip 10mm

 Согласно формуле Тейлора с остаточным членом в форме Пеано 
в окрестности точки можно записать приближенное равенство

\vskip 7mm

\centerline{$f(x) \thickapprox f_{n}(a,x).$}

\vskip 7mm
                  
Оказывается, что многочлен $f_{n}(a,x)$ может хорошо приближать $f(x)$ 
и в некоторой, иногда весьма большой, окрестности точки $a$. Более 
того, знание всех чисел $f^{(n)}(a)$, соответствующих только одной точке 
$a$ часто позволяет вычислить $f(x)$ при любом $x$ с любой требуемой 
степенью точности. Этот факт важен не столько для вычислений, 
сколько для построения теории. Выражаясь более точно, мы сейчас 
докажем одну из важнейших теорем анализа, центральную теорему
курса в этом семестре, а именно: формулу Тейлора с остаточным 
членом в общей форме (или форме Шлемильха--Роша). 

%\vskip 6mm

\begin{theorem} (формула Тейлора). Пусть $f(x) - (n+1)$ раз диференциируемая функция на интервале $(x_{0},x_{1})$. Пусть $a<b$  -- любые две точки из этого интервала.\\
 Тогда для любого положительного $\alpha > 0$ существует точка $c$, лежащая между $a$ и $b$ такая, что 
%\vskip 8mm

$r_{n}(b) = f(b) - f_{n}(a,b) =  \dfrac{n+1}{\alpha} \left(\dfrac{b-a}{b-c}\right)^{\alpha} \cdot (b-c)^{n+1} \cdot \dfrac{f^{(n+1)}(c)}{(n+1)!}$.\\
%\vskip 8mm
Напомним, что 
%\vskip 6mm
$$f_{n}(a,b) = f(a)+\dfrac{f'(a)}{1!}(b-a)+ \cdots +\dfrac{f^{n}(a)}{n!}(b-a)^n.$$
\end{theorem}
$\blacktriangleleft$ Определим число $H$ равенством
%\vskip 6mm

\centerline{$H = \dfrac{f(b)-f_{n}(a,b)}{(b-a)^{\alpha}}$.}

%\vskip 6mm

По существу, нам надо доказать, что на интервале $(a,b)$ найдется точка $c$ такая, что 

%\vskip 4mm
$$H = \dfrac{n+1}{\alpha}(b-c)^{n+1- \alpha} \cdot \dfrac{f^{n+1}(c)}{(n+1)!}$$

%\vskip 14mm
Докажем это, опираясь на теорему Ролля. Равенство, определяющее число $H$, можно записать так:
%\vskip 4mm
$$f(b)-f_{n}(a,b)-H(b-a)^{\alpha}=0.$$
%\vskip 4mm
Рассмотрим функцию $\varphi(t)$, определенную на $[a,b]$ соотношением \\
%\vskip 3mm
\centerline{$\varphi(t)=f(b)-f_{n}(t,b)-H(b-t)^{\alpha}.$}
%\vskip 3mm
Тогда, очевидно, $\varphi(a)=0$. Кроме того, имеем, что $\varphi(t)=0$ дифференциируема на $(a,b)$ и непрерывна на $[a,b]$. Далее, так как справедливо равенство $f_{n}(b,b)=f(b)$, то
%\vskip 3mm

\centerline{$\varphi(b)=f(b)-f(b)-H(b-b)^{\alpha}=0.$}
%\vskip 3mm
Следовательно, по теореме Ролля на интервале $(a,b)$ производная $\varphi'(a)$ обращается в нуль в некоторой точке $c$, т.е.
$$\varphi'(t)=0,\quad \text{ при } t=c,\quad c\in(a,b).$$
%\vskip 3mm
Запишем $\varphi'(t)$ в развернутой форме:
$$\varphi'(t)= -f_{n}(t,b)+\alpha H(b-t)^{\alpha - 1}=$$
%\vskip 2mm
$$\left(f(t)+\dfrac{f'(t)}{1!}(b-t)+ \cdots + \dfrac{f^{(n)}}{n!}{(b-t)}^{n} \right)_{t}' + \alpha H(b-t)^{\alpha-1}.$$
%\vskip 3mm
Так как при $s=1, \ldots ,n$ имеем
%\vskip 3mm
$$\left(\dfrac{f^{s}(t)}{s!}(b-t)^s \right)'_{t} = \dfrac{f^{s+1}(t)}{s!}(b-t)^s - \dfrac{f^{s}(t)}{(s-1)!}(b-t)^{s-1},$$
%\vskip 3mm
то
%\vskip 3mm
$$\varphi'(t)= \alpha H(b-t)^{\alpha-1} - \dfrac{f^{(n+1)}(t)}{n!}(b-t)^{n}$$ 
%\vskip 3mm
Отсюда при $t=c$ получаем 

$$\varphi'(c)= \alpha H(b-c)^{\alpha-1} - \dfrac{f^{(n+1)}(c)}{n!}(b-c)^{n}=0.$$
%\vskip 3mm
Следовательно,
%\vskip 3mm

$$H = \dfrac{n+1}{\alpha}(b-c)^{n+1- \alpha} \cdot \dfrac{f^{n+1}(c)}{(n+1)!}. \blacktriangleright$$
\begin{sledstvie} Формула Тейлора с остаточным членом в форме Шлемильха -- Роша верна и при $a \ge b.$
\end{sledstvie}
$\blacktriangleleft$ 1. Если $b=a$, то $f_{n}(a,b)=f(a), r_{n}(b)=0$ и формула имеет место.

2. Если $b<a$, то применим теорему к  функции $g(x)=f(2a-x), b_{1}=2a-b.$ Тогда $b_{1}>a$ и справедливо равенство 
%\vskip 4mm
$$g(b_{1})=g_{n}(a,b_{1})+R_{n}(b_{1}).$$
%\vskip 4mm
Но легко убедиться в том, что $g(b_{1})=f(b), g_{n}(a,b_{1})=f_{n}(a,b)$,
$R_{n}(b_{1})=r_{n}(b)$. Действительно, имеем
%\vskip 3mm
$$(b_{1}-a)^s=(a-b)^s=(-1)^s(b-a)^s;$$  
%\vskip 3mm
$$g_{n}(a,b_{1})=g(a)+ \dfrac{g(a)}{1!}(b_{1}-a)+ \ldots +\dfrac{g^{(n)}(a)}{n!}(b_{1}-a)^n=$$
%\vskip 3mm
$$=f(a)+ \dfrac{f(a)}{1!}(b-a)+ \ldots +\dfrac{f^{(n)}(a)}{n!}(b-a)^n=f_{n}(a,b).$$
%\vskip 4mm
Далее при некотором $c_{1},$ $a<c_{1}<b$, справедливо равенство
%\vskip 4mm
$$R_{n}(b_{1})=\dfrac{n+1}{\alpha} \left(\dfrac{b_{1}-a}{b_{1}-c_{1}} \right)^{\alpha} (b_{1}-c_{1})^{n+1} \dfrac{g^{(n+1)}(c_{1})}{(n+1)!}.$$
%\vskip 4mm
Положим $c=2a-c_{1}$. Тогда $b<c<a$,
%\vskip 4mm
$$R_{n}(b_{1})=\dfrac{n+1}{\alpha} \left(\dfrac{b-a}{b-c} \right)^{\alpha} \dfrac{f^{(n+1)}(c_{1})}{(n+1)!}(b-c)^{n+1}=r_{n}(b).$$
%\vskip 4mm
Таким образом, формула Тейлора с остаточным членом в форме\\
 Шлемильха--Роша в случае $a \ge b$ имеет тот же вид, что и при $a<b$.
$\blacktriangleright$
%\vskip 3mm
{\bfseries Частные случаи формулы Тейлора.}
1. Остаточный член в форме Лагранжа $(\alpha=n+1)$. В этом случае
%\vskip 4mm
$$r_{n}(b)=\dfrac{n+1}{n+1} \left(\dfrac{b-a}{b-c} \right)^{n+1} (b-c)^{n+1} \dfrac{f^{(n+1)}(c)}{(n+1)!}=\dfrac{f^{(n+1)}(c)}{(n+1)!}(b-a)^{n+1}.$$
%\vskip 3mm
2. Остаток в форме Коши ($\alpha=1$):
$$r_n(b)=\dfrac{n+1}{1}\dfrac{b-a}{b-c}(b-c)^{n+1}\dfrac{f^{(n+1)}(c)}{(n+1)!},$$

$$c=a+\theta(b-a),\quad 0<\theta<1,\quad 1-\theta= \dfrac{b-c}{b-a},$$
%\vskip 3mm
$$r_{n}(b)=(b-a)^{n+1}(1-\theta)^n \dfrac{f^{(n+1)}(a+\theta(b-a))}{n!}.$$ 
%\vskip 5mm
{\itshape Замечания}. 1.\quad Формула Тейлора (с остаточным членом в любой форме) в частном случае $a=0$ называется {\itshape формулой Маклорена}.
\quad Сравнивая формулы Тейлора с остаточными членами в общей форме и в форме Пеано, видим, что в в первом случае имеем более точный результат, однако достигается это за счет более жестких требований к функции. В самом деле, в первом случае в окрестности точки, в которой рассматривается разложение, требуется существование $(n+1)$-й производной данной функции, а во втором случае -- только $(n-1)$-й производной, то есть на две производные меньше.

























\begin{thebibliography}{9}
\bibitem{arxip} Архипов Г.И., Садовничий В.А., Чубариков В.Н. Лекции по математическому анализу.
\bibitem{demid} Демидович Б.П. Сборник задач и упражнений по математическому анализу.
\end{thebibliography}



\tableofcontents
\end{document}
