\setcounter{theorem}{0}
\setcounter{example}{0}
\setcounter{task}{0}

\chapter{Критерий  равномерной сходимости функциональной последовательности.}
\vskip 7mm
    \centerline{\bf Мамедов Джавид} \vskip 1cm
    Докажем теперь критерий Коши равномерной сходимости функциональной последовательности.
\vskip 5mm

\begin{theorem}
(критерий Коши). Для того чтобы функциональная последовательность  $ A_n(x) $  равномерно сходилась на множестве $M$ , необходимо и достаточно, чтобы для любого $\varepsilon$>0 существовал номер $n_0=n_0(\varepsilon)$
такой, что при всех $m>n_0$ и $n>n_0$ и всех $x\in M$ имело бы место неравенство $|A_n(x)-A_m(x)|<\varepsilon$.
\end{theorem}
\vskip 5mm
$\blacktriangleleft$
{\bfНеобходимость}. В этом случае $A_n(x) \underset{M}{\rightrightarrows} A(x)$.
 Таким образом, для любого $\varepsilon>0$ существует число $n_0=n_0(\varepsilon)$ такое, что для всех $n>n_0$ и для всех $x\in M$ имеем $|A_n(x)-A(x)|<\varepsilon/2$. Но тогда
при $m>n_0$ и $n>n_0$ имеем

\vskip 5mm

$$|A_m(x)-A_n(x)|\le|A_m(x)-A(x)|+|A(x)-A_n(x)|<\varepsilon/2 + \varepsilon/2 =\varepsilon$$,
\vskip 5mm
что и требовалось доказать.
\vskip 5mm
{\bfДостаточность}.При каждом фиксированном $x\in M$ функциональная последовательность $A_n(x)$ превращается в числовую и для нее выполняется критерий Коши. Это значит, что она имеет предел $A(x)$,
 т.е. предельная функция существует на всем множестве $M$. Далее, каково бы ни было число $\varepsilon>0$, по условию найдется номер $n_1=n_1(\varepsilon/2)$ такой, что при всех $m$ и $n>n_1$ имеем
$|A_n(x)-A_m(x)|<\varepsilon/2$.

Снова произвольно зафиксируем $x\in M$ и устремим $m$ к бесконечности. Получим неравенство
\vskip 5mm

$$|A_n(x)-A(x)|\le\varepsilon/2<\varepsilon$$

Но тогда, полагая $n_0=n_0(\varepsilon)=n_1(\varepsilon/2)$, при всех $n>n_0$ и всех $x\in M$ 
будем иметь 

$$|A_n(x)-A(x)|<\varepsilon$$,

 т.е. $A_n(x) \underset{M}{\rightrightarrows} A(x)$. Теорема 1 доказана.
$\blacktriangleright$
Если $A_n(x)$ \texttwelveudash последовательность частичных сумм функционального ряда $\sum a_n(x)$, то теорема 1 дает нам критерий Коши равномерной сходимости этого ряда. Сформулируем его в виде следующей теоремы.
\vskip 5mm
\begin{theorem}
Для равномерной сходимости ряда $\sum a_n(x)$ на множестве $M$ необходимо и достаточно, чтобы для любого $\varepsilon>0$ существовало $n_0=n_0(\varepsilon)$ такое, что для каждого $n>n_0$, и для каждого $p\in \mathbb N$ и для всех $x\in M$ выполнялось бы равенство

$$\Biggl|\sum_{k=n+1}^{n+p}a_k(x)\Biggr|<\varepsilon$$.
\end{theorem}

И наконец, исходя из теоремы 2 сформулируем в прямой форме критерий отсутствия равномерной сходимости ряда $\sum a_n(x)$.
\vskip 5mm

\begin{theorem}
Утверждение о том, что ряд $\sum a_n(x)$ или последовательность $A_n(x)$ не являются равномерно сходящимися на множестве $M$, означает, что существует $\varepsilon>0$ такое, что найдутся две последовательности $\{n_m\}$ и $\{p_m\}\in \mathbb N$, причем $n_{m+1}>n_m$, а также последовательность $\{x_m\}\in M$, для которых имеет место неравенство 

$$\Biggl|\sum_{k=n_m+1}^{n_m+p_m}a_k(x_m)\Biggr|\ge\varepsilon$$.

\end{theorem}
\vskip 5mm

\begin{example}
неравномерно сходящихся рядов и последовательностей.
\vskip 5mm
{\bf1.} Ряд $A(x)=\sum_{n=0}^{\infty}x(1-x)^n$ сходится неравномерно на $[0,2)$.
\vskip 5mm
Действительно, сумма ряда $A(x)$ при $x\ne0$ равна 

$$A(x)=x\sum_{n=0}^{\infty}(1-x)^n=x\frac{1}{1-(1-x)}=\frac{x}{x}=1$$.

и $A(0)=0$. Это значит, что $x=0$ \texttwelveudash точка разрыва функции $A(x)$. Но если бы сходимость была равномерной, то функция $A(x)$ была бы непрерывной в силу теоремы 1\S2, поскольку $a_n(x)=x(1-x)^n$ непрерывна в нуле. Но это не так. Следовательно, равномерной сходимости нет.
\vskip 5mm
{\bf2.}Если $A_n(x)=x^n$, то на множестве $M=(0,1)$ равномерная сходимость не имеет места.\\
Действительно, в теореме 3 положим $\varepsilon=0,1$ и при каждом $m>1$ возьмем $n_m=m$, $x_m=1-\frac{1}{m}$, $p_m=m$. Тогда будем иметь 

$$|A_m(x_m)-A_{2m}(x_m)|=\Biggl|\biggl(1-\frac{1}{m}\biggr)^m-\biggl(1-\frac{1}{m}\biggr)^{2m}\Biggr|=\biggl(1-\frac{1}{m}\biggr)^m \biggl(1-\biggl(1-\frac{1}{m}\biggr)^m\biggr)>$$$$>\frac{1}{3}\cdot\frac{2}{3}
>0,1=\varepsilon.$$
Таким образом, по критерию Коши в форме теоремы 3 последовательность $A_n(x)$ не является равномерно сходящейся.
\end{example}
\vskip 5mm

\begin{task} 
Пусть функции $f_n(x)$ непрерывны на $[0,1]$ при всех $n\in \mathbb N$ и $f_n(x)\to f_0(x)$ при $n\to \infty$. Доказать, что $f_0(x)$ имеет точку непрерывности на $(0,1)$.
\end{task}
