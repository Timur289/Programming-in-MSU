\documentclass[12pt,titlepage]{report}
\usepackage[russian]{babel}
\usepackage{amsmath}
\usepackage{amssymb}
\usepackage{amsthm}
\usepackage{textcomp}
\usepackage[utf8]{inputenc}
\usepackage[dvips]{graphics,epsfig}
\usepackage{latexsym} 
\pagestyle{empty}
\frenchspacing
\textwidth=16cm
\textheight=25cm
\topmargin=-0.5in
\oddsidemargin=0mm
\newcommand\ds{\displaystyle}
\newtheorem{object}{Определение}
\newtheorem{approval}{Утверждение}
\newtheorem{theorem}{Теорема}
\newtheorem{example}{Пример}
\newtheorem{lemma}{Лемма}
\newtheorem{task}{Задача}
\newtheorem{sledstvie}{Следствие}
\def\R{\mathbb R}
\def\N{\mathbb N}
\def\dx{{\rm d}x}
\def\dxy{{\rm d}xy}


%\setcounter{object}{0}
%\setcounter{approval}{0}
%\setcounter{theorem}{0}
%\setcounter{example}{0}
\begin{document}
\chapter{Сходимость числовых рядов}
\centerline{ \bf Автор:Гаджиева Нармин и Тугушева Рената} \vskip 1cm
\section{Абсолютная и условная сходимость рядов. Ряды Лейбница }


Мы снова возвращаемся к рассмотрению числовых рядов общего вида.
\begin{object}Ряд $\sum a_{n}$ называется \textbf{абсолютно сходящимся}, если сходится ряд $\sum|a_{n}|$.\end{object}
Ясно, что всякий сходящийся ряд с неортрицательными членами абсолютно сходится. В то же время легко построить сходящийся ряд,
который не является абсолютно сходящимся. В качестве примера можно привести следующий ряд: $$-1+1-\dfrac12+\dfrac12-\dfrac13+\dfrac13-\dfrac14+\dfrac14-...$$ Его сумма равна нулю, и в то же время ряд, составленный из модулей его членов, расходится в силу расходимости гармонического ряда.
\begin{object}
Сходящийся ряд $\sum a_{n}$ называется \textbf{условно сходящимся}, если ряд $\sum|a_{n}|$ расходится.
\end{object}
Согласно этому определению, рассмотренный выше ряд является условно сходящимся. Заметим, что про абсолютно (или условно) сходящийся ряд говорят еще, что ряд сходится абсолютно (или условно). Целесообразность введенных понятий подкрепляется следующей теоремой.
\begin{theorem}Если ряд $\sum a_{n}$ абсолютно сходится, то он является сходящимся.\end{theorem}
$\blacktriangleleft$ По критерию Коши из сходимости ряда $\sum |a_{n}|$ следует, что при любом $\epsilon>0$ найдется номер $n_{0}(\epsilon)$ такой, что при всех p$\ge$1 и n>n($\epsilon$) имеем $$\sum_{m=n+1}^{n+p} a_{m}<\epsilon,$$ откуда $$|\sum_{m=n+1}^{n+p} a_{m}|\le\sum_{m=n+1}^{n+p} |a_{m}|<\epsilon.$$ Но это означает выполнение критерия Коши. $\blacktriangleright$
\begin{object}Числовой ряд $\sum a_{n}$ называется \textbf{знакочередующимся}, если все его соседние члены имеют разные знаки.\end{object}
\begin{object}Знакочередующийся ряд вида $\sum a_{n}$ называется \textbf{рядом Лейбница}, если модуль его общего члена монотонно стремится к нулю.\end{object}
\begin{theorem}\label{st}Всякий ряд Лейбница $\sum a_{n}$ сходится.\end{theorem}
$\blacktriangleleft$ Покажем начала, что всякий отрезок этого ряда мажорируется модулем его первого члена.Пусть $\sum_{m=n+1}^{n+p} a_{m}$\\
- некоторый отрезок ряда. Мы хотим доказать неравенство вида
$$|\sum_{m=n+1}^{n+p} a_{m}|\le|a_{n}|.$$
Положим $b_{k}=|a_{k}|$ при всех k$\in$N. Тогда
$$|a_{k}+a_{k+1}|=|a_{k}|-|a_{k+1}|=b_{k}-b_{k+1}<b_{k}$$.
Кроме того, при всех к числа $(a_{k}+a_{k+1})$ имеют один и тот же знак. Следовательно, при четном p=2r имеем
$$0\le|a_{n+1}+a_{n+2}+...+a_{n+2r-1}+a_{n+2r}|=$$
$$=(b_{1}-b_{2})+...+(b_{2r-1}-b_{2r})=$$
$$=b_{1}-(b_{2}-b_{3})-...-(b_{2r-2}-b_{2r-1})-b_{2r}\le b_{1}=|a_{n+1}|.$$
Если же p=2r+1 нечетное, то
$$0\le|a_{n+1}+...+a_{n+2r+1}|=(b_{1}-b_{2})+...+(b_{2r-1}-b_{2r})+b_{2r+1}=$$
$$=b_{1}-(b_{2}-b_{3})-...-(b_{2r}-b_{2r+1})\le b_{1}=|a_{n+1}|.$$
Таким образом, в обоих случаях действительно
$$T_{n,p}=|\sum_{m=n+1}^{n+p} a_{m}|\le|a_{n+1}|$$
Но теперь, так как $b_{n+1}\to$0, мы при любом наперед заданном $\epsilon>0$ и достаточно большом n имеем
$$T_{n,p}\le b_{n+1}<\epsilon$$
откуда в силу произвольности p, исходя из критерия Коши, заключаем, что ряд $\sum a_{n}$ сходится. $\blacktriangleright$ 

\begin{theorem}(оценка остатка ряда Лейбница). Для остатка $r_{n}$ ряда Лейбница $\sum a_{n}$ справедлива оценка $|r_{n}|\le|a_{n+1}|$.\end{theorem}

$\blacktriangleleft$ Согласно теореме \ref{st} ряд $\sum a_{n}$ сходится, поэтому
$$|r_{n}|=\Biggl|\sum_{m=n+1}^\infty a_{n}\Biggr|.$$
Заметим, что при доказательстве теоремы \ref{st} для любого натурального $p$ нами получена оценка
$$\Biggl|\sum_{m=n+1}^{n+p} a_{n}\Biggr|\le|a_{n+1}|$$
Устремляя $p\to\infty$, получаем требуемое неравенство. $\blacktriangleright$


\section{Признаки Абеля и Дирихле}

Признаки Абеля и Дирихле применяются при доказательстве достаточно широкого класса числовых рядов общего вида. Доказательство обоих признаков основано на формуле дискретного преобразования Абеля, которую мы сейчас докажем.
\begin{theorem}Пусть $A_{k}=\sum_{m=N+1}^k a_{m}$. Тогда при M>N имеют место формулы:
\begin{equation}\label{nfirst}\sum_{k=N+1}^M a_{k}b_{k}=A_{M}b_{M}+\sum_{k=N+1}^{M-1}A_{k}(b_{k}-b_{k+1})\end{equation}
\begin{equation}\label{second}\sum_{k=N+1}^M a_{k}b_{k}=A_{M}b_{M+1}+\sum_{k=N+1}^M A_{k}(b_{k}-b_{k+1})\end{equation}
\end{theorem}

$\blacktriangleleft$ Заметим прежеде всего, что правые части формул равны между собой, так как, вычитая правую часть формулы (\ref{second}) из правой части формулы (\ref{nfirst}), получаем
$$A_{M}b_{M}-A_{M}b_{M+1}-a_{m}(b_{M}-b_{M+1})=0$$
Следовательно, достаточно доказать только формулу (\ref{nfirst}). Преобразуя ее правую часть, имеем
$$A_{M}b_{M}+\sum_{k=N+1}^{M-1}A_{k}(b_{k}-b_{k+1})=A_{M}b_{M}+\sum_{k=N+1}^{M-1}A_{k}b_{k+1}=$$
$$=\sum_{k=N+1}^M A_{k}b_{k}-\sum_{l=N+2}^M A_{l}b_{l}=A_{N+1}b_{N+1}+\sum_{k=N+2}^{M}(A_k-A_{k-1})b_k=$$
$$=a_{N+1}b_{N+1}+\sum_{k=N+2}^{M}a_kb_k=\sum_{k=N+1}^{M}a_kb_k\ \blacktriangleright$$

Признаки Абеля и Дирихле применяются к рядам вида $\sum a_nb_n.$

\begin{theorem}
Справедливы следующие утверждения:\\
(А)(Признак Абеля). Если последовательность $b_n$ монотонна и ограничена, а ряд $\sum a_n$ сходится, то ряд $\sum a_nb_n$ также сходится.\\
(Д)(Признак Дирихле).Если последовательность $b_n$ монотонна и $b_n\to0$ при $n\to \infty$, а последовательность $s_n$ частичных сумм ряда $\sum a_n$ ограничена, то ряд $\sum a_nb_n$ сходится.
\end{theorem}

$\blacktriangleleft$ Ограничимся рассмотрением случая, когда $b_n\ge0$ и $b_n$ убывает. Все прочие случаи легко сводятся к данному следующим образом. Если $b_n\le0$, то надо изменить знаки у всех $a_n$ и $b_n$. Если же $b_n$ возрастает, то $b_n$ надо представить в виде $b_n=b_0-d_n$, где $b_0=\lim_{n\to\infty}b_n$, и свести теорему к исследованию ряда $\sum a_nb_n$. Здесь уже $d_n$ убывает.
Доказательство теоремы проведем, применяя критерий Коши к ряду $\sum a_nb_n$. Для этого применим формулу (\ref{nfirst}) преобразования Абеля к отрезку этого ряда $T_{n,p}$. Используя обозначение $A_k=s_k-s_n$ и учитывая, что $b_k-b_{k+1}\ge=0$, получим
$$\bigl|T_{n,p}|\bigr|=\Biggl|\sum_{k=n+1}^{n+p}a_kb_k\Biggr|=\Biggl|A_{n+p}b_{n+p}+\sum_{k=n+1}^{n+p-1}A_k(b_k-b_{k+1}\Biggr|\le$$
$$\le \bigl|A_{n+p}\bigr|b_{n+p}+\max_{n<k<n+p}\bigl|A_k\bigr|\sum_{k=n+1}^{n+p-1}(b_k-b_{k+1})\le$$
$$\le\max_{n<k\le n+p}\bigl|a_k\bigr|b_{n+1}$$

Рассмотрим теперь случай (А). Поскольку $b_n$ ограничена, при некотором c>0 для всех n имеем $|b_{n+1}|<c$. Далее, так как ряд $\sum a_n$ сходится, то для любого $\varepsilon>0$ найдется номер $n_0(\varepsilon)$ такой, что при всех $n>n_0(\varepsilon)$ и k>n имеем
$$|A_k|=\Bigl|\sum_{m=n+1}^k a_m\Bigr|=|s_k-s_n|\le|s_k|+|s_n|<\varepsilon$$
Тогда при указанных n и любом p для величины $T_{n,p}$ справедлива оценка
$$|T_{n,p}|=\Bigl|\sum_{k=n+1}^{n+p} a_kb_k.\Bigr|\le c\varepsilon$$
Но так как $\varepsilon$ произвольно, а c фиксировно, то последнее неравенство означает, что ряд $\sum a_nb_n$ удовлетворяет критерию Коши и поэтому сходится. Тем самым признак Абеля доказан.
В случае (Д) ограничены частные суммы $a_k$ ряда $\sum a_n$, и поэтому существует c, для которого $|A_k|<c$ при всех k. Кроме того, $b_n\to 0$. Следовательно, при произвольном $\varepsilon>0$, достаточно большом $n>n_0(\varepsilon)$ произвольном $p\ge1$ имеем оценку $$|T_{n,p}|=\Bigl|\sum_{k=n+1}^{n+p}a_kb_k\Bigr|<c\varepsilon$$ Отсюда, как и в случае (А), заключаем, что по критерию Коши ряд $\sum a_nb_n$ сходится. $\blacktriangleright$
\end{document}
