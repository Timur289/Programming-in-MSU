\setcounter{approval}{0}
\setcounter{theorem}{0}
\setcounter{example}{0}
\chapter{ПЕРЕСТАНОВКИ ЧЛЕНОВ РЯДОВ} \vskip -1cm
\centerline{\bf Кязимзаде Мамедгусейн} \vskip 1cm
\par\textbf{Определение 1.} Пусть  	$\sigma$ : N $\rightarrow$ N - взаимно однозначное отображение натурального ряда на самого себя. Тогда ряд $\sum$ $b_n$, где $b_n$ = $a_\sigma(n)$ называется \textbf{перестановкой ряда} $\sum$ $a_n$

\begin{theorem}
Любая перестановка $\sum$ $b_n$ абсолютно сходящегося ряда $\sum$ $a_n$ = A абсолютно сходиться к той же сумме A.
\end{theorem}

$\blacktriangleleft$
Положим
$$A = \sum_{n=1}^\infty a_n, A_n = \sum_{k=1}^n a_k, A' = \sum_{k=1}^\infty |a_k|, A'_n = \sum_{k=1}^n |a_k|$$
Зафиксируем некоторое $\epsilon$ > 0. Пусть $n_1$ настолько велико,что $A'$ - $A'_n1$ < $\epsilon$. Тогда при любом n > $n_1$ для остатка $r_n$ = $A$ - $A_n$ ряда $\sum$ $a_n$ имеем оценку
$$|r_n| = |\sum_{k=n+1}^\infty |a_k| \leq \sum_{k=n+1}^\infty |a_k| = A' - A'_n1 < \epsilon $$
Пусть теперь $n_2$ таково, что среди чисел $\sigma(1)$,...,$\sigma(n_2) $ содержатся все целые числа отрезка [1,$n_1$]. Положим m= max($\sigma(1)$,...,$\sigma(n_2))$. Тогда при всех n > $n_2$ имеем 
$$B_n = \sum_{k=1}^n b_k = \sum_{k=1}^n a_\sigma(k) = \sum_{k=1}^n a_k + \sum_{k=n_1+1}^m a_k$$
Штрих в последней сумме означает, что некоторые слагаемые в ней опущены. Для этой суммы, очевидно, имеют место оценки
$$ |B_n - A_n| = |\sum_{k=n_1+1}^m a_k| \leq \sum_{k=n_1+1}^m |a_k| = A'_m - A'_n  \leq A' - A'_n(1) < \epsilon    $$
Отсюда следует, что при всех n > $n_2$ 
$$|A-B_n| \leq |A-A_{n_1}| + |B_n - A_{n1}| \leq (A' - A_{n1}) + (A' - A_{n1}) \leq 2\epsilon $$
В силу произвольности выбора $\epsilon$ > 0 это означает,что $B_n\rightarrow A$ при $n \rightarrow \infty$ , т.е ряд $\sum b_n$ сходится и $\sum b_n = \sum a_n$. Отсюда, в частности,следует, что и ряд $\sum |b_k|$ сходится к сумме $A'$, т.е. ряд $\sum |b_n|$ сходится абсолютно. Теорема 1 доказана полностью.
$\blacktriangleright$

\par Основное отличие в свойствах абсолютно и условно сходящихся рядов обнаруживается при перестановке их членов. Как показывает теорема , бесконечная сумма абсолютно сходящегося ряда в этом случае ведет себя точно так же, как и конечная сумма, т.е. при перестановке слагаемых сумма не изменяется. Гораздо более сложная ситуация имеет место для условно сходящегося ряда. Тем не менее, она достаточно полно описывается следующей теоремой.
\begin{theorem} 
(теорема Римана о перестановке членов условно сходящегося ряда) Каково бы не было вещественное число A, найдется сходящаяся перестановка $\sum b_n$ условно сходящегося ряда $\sum a_n$ такая, что $\sum b_n$ = A.
\end{theorem}
$\blacktriangleleft$
Для простоты будем считать, что $a_n$*0 при всех n. Сначала в ряде $\sum a_n$ выделяем все положительные слагаемые $p_k$ и отрицательные слагаемые $-q_l$, нумеруя их индексами k и l в порядке следования в ряде $\sum a_n$. Затем составляем перестановку $\sum b_n$ ряда $\sum a_n$: в качестве $b_1$ берем $p_1$, если A $\ge$ 0, и $-q_l$, если A $\leq$ 0. Подчеркнем, что все $p_k$ и $-q_l$ положительны.
\parДалее мы добавляем в общую сумму $\sum_{m=1}^n b_m$ очередные слагаемые по следующему правилу: если сумма не превышает A, то добавляем очередное положительное слагаемое $b_{n+1}$ = $p_{k1}$. В результате этого сумма все время колеблется вокруг значения A, причем размах колебаний постепенно убывает до нуля, и в пределе для суммы ряда $\sum b_n$ мы получаем требуемое значение A.
\parДля того чтобы доказательство теоремы было полным, достаточно в приведенной схеме обосновать только некоторые его моменты.
\parДокажем, что оба ряда $\sum p_k$ и $\sum-q_l$ расходятся. Действительно если бы оба они расходились, то исходный ряд $\sum a_n$ сходился бы абсолютно, а если бы один ряд расходился, а другой сходился, то частичная сумма ряда $\sum a_n$, составленная из двух частичных сумм ряда $\sum a_n$, составленная из двух частичных сумм рядов $\sum p_k$ и $\sum-q_l$ соответсвенно, тоже расходилась, что неверно. Далее заметим, что поскольку ${p_k}$ и ${-q_l}$ являются подпоследовательностями для ${a_n}$, $p_k$ $\rightarrow$ 0 при k$\rightarrow$ $\infty$
Для определенности будем считать, что A $\ge$ 0. Тогла по построению ряд $b_n$ имеет такую структуру:
$\sum b_n$= $\underbrace{p_1+...+p_{k}}_{\text{$P_1$}}-\underbrace{q_1-...-q_{l1}}_{\text{$Q_1$}}+ \underbrace{p_{k1+1}+...+p_{k2}}_{\text{$P_2$}}-\underbrace{q_{l1+1}-...-p_{l2}}_{\text{$Q_1$}}+...$
\par Здесь числа $P_1$,$P_2$,...,$Q_1$,$Q_2$,... обозначают суммы подряд идущих слагаемых одного знака в ряде $\sum b_n$. Количество групп слагаемых одинакового знака в этой сумме бесконечно, так как в противном случае ряд $\sum b_n$ отличался бы от $\sum p_k$ или $\sum-q_l$ лишь конечным числом членов,и тогда он расходился бы к $+\infty$ или к $-\infty$, соответствено. Но это не имеет места, так как по построению величина частичной суммы $s_n$ ряда на каждом шаге изменяется в направлении приближения к числу A, если только $s_n$ $\ne$ A. В силу этого, в сумму $\sum b_n$ войдут все числа $p_k$ и $-q_l$, а следовательно, и все $a_n$, т.е. $\sum b_n$-действительно перестановка ряда $\sum a_n$.
\par Теперь оценим разность $r_n=s_n$-A. При всяком n член ряда $\sum b_n$ в зависимости от своего знака попадает в одну из сумм $P_m$ или $Q_m$. Следовательно, мы имеем одно из равенств: $b_n$=$p_k$ или $b_n$=$-q_l$.
\par По построению ряда $\sum b_n$ величина $r_n$ меняет знак в том случае, если $b_n$=$p_{km}$ или $b_n$=$-q_l$. Тогда в обоих случаях имеет место оценка
$$ |r_n| = |s_n - A| \leq |b_n| $$
\par Для всех прочих n при добавлении очередного слагаемого $|r_n|$ значения частичной суммы $|s_n|$ от числа A убывает, поэтому тогда справедливо неравенство $|r_n|$ < $|r_n-1|$. Следовательно, всегда имеем
$$ |s_n-A| < p_{km} + q_{lm} + q_{lm-1}. $$
\par Здесь номер m можно рассматривать как монотонно стремящуюся к бесконечноси функцию от n, и поэтому для последовательности $d_n$, где $d_n$ = $p_{km}$ + $q_{lm}$ + $q_{lm-1}$, в силу того, что $p_k$ и $q_l$ $\rightarrow$ 0 при k и l $\rightarrow$ $\infty$. Отсюда при n $\rightarrow$ $\infty$ оконсательно получим $r_n$ = $s_n$ - A $\rightarrow$ 0 и $s_n$ $\rightarrow$ A. Теорема 2 доказана.$\blacktriangleright$
