\chapter{Формула Тейлора с остаточным членом в общей форме} 
\centerline{ Лекция Виктора Башарова}
\vskip 10mm

 Согласно формуле Тейлора с остаточным членом в форме Пеано 
в окрестности точки можно записать приближенное равенство

\vskip 7mm

\centerline{$f(x) \thickapprox f_{n}(a,x).$}

\vskip 7mm
                  
Оказывается, что многочлен $f_{n}(a,x)$ может хорошо приближать $f(x)$ 
и в некоторой, иногда весьма большой, окрестности точки $a$. Более 
того, знание всех чисел $f^{(n)}(a)$, соответствующих только одной точке 
$a$ часто позволяет вычислить $f(x)$ при любом $x$ с любой требуемой 
степенью точности. Этот факт важен не столько для вычислений, 
сколько для построения теории. Выражаясь более точно, мы сейчас 
докажем одну из важнейших теорем анализа, центральную теорему
курса в этом семестре, а именно: формулу Тейлора с остаточным 
членом в общей форме (или форме Шлемильха--Роша). 

%\vskip 6mm

\begin{theorem} (формула Тейлора). Пусть $f(x) - (n+1)$ раз диференциируемая функция на интервале $(x_{0},x_{1})$. Пусть $a<b$  -- любые две точки из этого интервала.\\
 Тогда для любого положительного $\alpha > 0$ существует точка $c$, лежащая между $a$ и $b$ такая, что 
%\vskip 8mm

$r_{n}(b) = f(b) - f_{n}(a,b) =  \dfrac{n+1}{\alpha} \left(\dfrac{b-a}{b-c}\right)^{\alpha} \cdot (b-c)^{n+1} \cdot \dfrac{f^{(n+1)}(c)}{(n+1)!}$.\\
%\vskip 8mm
Напомним, что 
%\vskip 6mm
$$f_{n}(a,b) = f(a)+\dfrac{f'(a)}{1!}(b-a)+ \cdots +\dfrac{f^{n}(a)}{n!}(b-a)^n.$$
\end{theorem}
$\blacktriangleleft$ Определим число $H$ равенством
%\vskip 6mm

\centerline{$H = \dfrac{f(b)-f_{n}(a,b)}{(b-a)^{\alpha}}$.}

%\vskip 6mm

По существу, нам надо доказать, что на интервале $(a,b)$ найдется точка $c$ такая, что 

%\vskip 4mm
$$H = \dfrac{n+1}{\alpha}(b-c)^{n+1- \alpha} \cdot \dfrac{f^{n+1}(c)}{(n+1)!}$$

%\vskip 14mm
Докажем это, опираясь на теорему Ролля. Равенство, определяющее число $H$, можно записать так:
%\vskip 4mm
$$f(b)-f_{n}(a,b)-H(b-a)^{\alpha}=0.$$
%\vskip 4mm
Рассмотрим функцию $\varphi(t)$, определенную на $[a,b]$ соотношением \\
%\vskip 3mm
\centerline{$\varphi(t)=f(b)-f_{n}(t,b)-H(b-t)^{\alpha}.$}
%\vskip 3mm
Тогда, очевидно, $\varphi(a)=0$. Кроме того, имеем, что $\varphi(t)=0$ дифференциируема на $(a,b)$ и непрерывна на $[a,b]$. Далее, так как справедливо равенство $f_{n}(b,b)=f(b)$, то
%\vskip 3mm

\centerline{$\varphi(b)=f(b)-f(b)-H(b-b)^{\alpha}=0.$}
%\vskip 3mm
Следовательно, по теореме Ролля на интервале $(a,b)$ производная $\varphi'(a)$ обращается в нуль в некоторой точке $c$, т.е.
$$\varphi'(t)=0,\quad \text{ при } t=c,\quad c\in(a,b).$$
%\vskip 3mm
Запишем $\varphi'(t)$ в развернутой форме:
$$\varphi'(t)= -f_{n}(t,b)+\alpha H(b-t)^{\alpha - 1}=$$
%\vskip 2mm
$$\left(f(t)+\dfrac{f'(t)}{1!}(b-t)+ \cdots + \dfrac{f^{(n)}}{n!}{(b-t)}^{n} \right)_{t}' + \alpha H(b-t)^{\alpha-1}.$$
%\vskip 3mm
Так как при $s=1, \ldots ,n$ имеем
%\vskip 3mm
$$\left(\dfrac{f^{s}(t)}{s!}(b-t)^s \right)'_{t} = \dfrac{f^{s+1}(t)}{s!}(b-t)^s - \dfrac{f^{s}(t)}{(s-1)!}(b-t)^{s-1},$$
%\vskip 3mm
то
%\vskip 3mm
$$\varphi'(t)= \alpha H(b-t)^{\alpha-1} - \dfrac{f^{(n+1)}(t)}{n!}(b-t)^{n}$$ 
%\vskip 3mm
Отсюда при $t=c$ получаем 

$$\varphi'(c)= \alpha H(b-c)^{\alpha-1} - \dfrac{f^{(n+1)}(c)}{n!}(b-c)^{n}=0.$$
%\vskip 3mm
Следовательно,
%\vskip 3mm

$$H = \dfrac{n+1}{\alpha}(b-c)^{n+1- \alpha} \cdot \dfrac{f^{n+1}(c)}{(n+1)!}. \blacktriangleright$$
\begin{sledstvie} Формула Тейлора с остаточным членом в форме Шлемильха -- Роша верна и при $a \ge b.$
\end{sledstvie}
$\blacktriangleleft$ 1. Если $b=a$, то $f_{n}(a,b)=f(a), r_{n}(b)=0$ и формула имеет место.

2. Если $b<a$, то применим теорему к  функции $g(x)=f(2a-x), b_{1}=2a-b.$ Тогда $b_{1}>a$ и справедливо равенство 
%\vskip 4mm
$$g(b_{1})=g_{n}(a,b_{1})+R_{n}(b_{1}).$$
%\vskip 4mm
Но легко убедиться в том, что $g(b_{1})=f(b), g_{n}(a,b_{1})=f_{n}(a,b)$,
$R_{n}(b_{1})=r_{n}(b)$. Действительно, имеем
%\vskip 3mm
$$(b_{1}-a)^s=(a-b)^s=(-1)^s(b-a)^s;$$  
%\vskip 3mm
$$g_{n}(a,b_{1})=g(a)+ \dfrac{g(a)}{1!}(b_{1}-a)+ \ldots +\dfrac{g^{(n)}(a)}{n!}(b_{1}-a)^n=$$
%\vskip 3mm
$$=f(a)+ \dfrac{f(a)}{1!}(b-a)+ \ldots +\dfrac{f^{(n)}(a)}{n!}(b-a)^n=f_{n}(a,b).$$
%\vskip 4mm
Далее при некотором $c_{1},$ $a<c_{1}<b$, справедливо равенство
%\vskip 4mm
$$R_{n}(b_{1})=\dfrac{n+1}{\alpha} \left(\dfrac{b_{1}-a}{b_{1}-c_{1}} \right)^{\alpha} (b_{1}-c_{1})^{n+1} \dfrac{g^{(n+1)}(c_{1})}{(n+1)!}.$$
%\vskip 4mm
Положим $c=2a-c_{1}$. Тогда $b<c<a$,
%\vskip 4mm
$$R_{n}(b_{1})=\dfrac{n+1}{\alpha} \left(\dfrac{b-a}{b-c} \right)^{\alpha} \dfrac{f^{(n+1)}(c_{1})}{(n+1)!}(b-c)^{n+1}=r_{n}(b).$$
%\vskip 4mm
Таким образом, формула Тейлора с остаточным членом в форме\\
 Шлемильха--Роша в случае $a \ge b$ имеет тот же вид, что и при $a<b$.
$\blacktriangleright$
%\vskip 3mm
{\bfseries Частные случаи формулы Тейлора.}
1. Остаточный член в форме Лагранжа $(\alpha=n+1)$. В этом случае
%\vskip 4mm
$$r_{n}(b)=\dfrac{n+1}{n+1} \left(\dfrac{b-a}{b-c} \right)^{n+1} (b-c)^{n+1} \dfrac{f^{(n+1)}(c)}{(n+1)!}=\dfrac{f^{(n+1)}(c)}{(n+1)!}(b-a)^{n+1}.$$
%\vskip 3mm
2. Остаток в форме Коши ($\alpha=1$):
$$r_n(b)=\dfrac{n+1}{1}\dfrac{b-a}{b-c}(b-c)^{n+1}\dfrac{f^{(n+1)}(c)}{(n+1)!},$$

$$c=a+\theta(b-a),\quad 0<\theta<1,\quad 1-\theta= \dfrac{b-c}{b-a},$$
%\vskip 3mm
$$r_{n}(b)=(b-a)^{n+1}(1-\theta)^n \dfrac{f^{(n+1)}(a+\theta(b-a))}{n!}.$$ 
%\vskip 5mm
{\itshape Замечания}. 1.\quad Формула Тейлора (с остаточным членом в любой форме) в частном случае $a=0$ называется {\itshape формулой Маклорена}.
\quad Сравнивая формулы Тейлора с остаточными членами в общей форме и в форме Пеано, видим, что в в первом случае имеем более точный результат, однако достигается это за счет более жестких требований к функции. В самом деле, в первом случае в окрестности точки, в которой рассматривается разложение, требуется существование $(n+1)$-й производной данной функции, а во втором случае -- только $(n-1)$-й производной, то есть на две производные меньше.
























