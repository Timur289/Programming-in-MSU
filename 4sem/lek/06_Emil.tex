\setcounter{object}{0}
\setcounter{approval}{0}
\setcounter{theorem}{0}
\setcounter{example}{0}
\chapter{Частные производные высшых порядков}
\vskip-1cm
\centerline{\bf Автор: Султанов Эмиль}\vskip 1cm


\parindent=1cm
Пусть $f(\bar{x})$ имеет в некоторой $\varepsilon$ - окрестности $O(\bar{a},\varepsilon)$ все первые частные производные $\frac{\partial f(\bar{x})}{\partial x_s} ,\ s=1,\ldots,n$. Эти частные производные сами являются функциями от $n$ переменных и могут иметь частные производные, т.е. можно определить следующие велечины 
 $$\frac{\partial}{\partial x_r} \left(\frac{\partial f}{\partial x_s}\right) = \frac{\partial^2 f}{\partial x_r \partial x_s}=f''_{{x_s}{x_r}}=\left(f'_{x_s}\right)'_{x_r}, \ \ s,r=1,\ldots,n.$$
Эти велечины называются  частными производными второго прядка. Если $s\neq r$, то они называются смешанными производными.

\parindent=1cm
Имеют место следующие теоремы о равенстве смешанных производных второго порядка.

\begin{theorem}
(теорема Шварца).  Пусть функция $f(\bar{x})$ в некоторой окрестности точки $\bar{x}=\bar{a}$ имеет смешанные частные производные второго порядка $\frac{\partial^2 f}{\partial x_1 \partial x_2}$ и $\frac{\partial^2 f}{\partial x_2 \partial x_1}$, причем они непрерывны в точке $\bar{x}=\bar{a}$. \ Тогда в точке $\bar{x}=\bar{a}$ эти производные равны между собой, т.е.
 $$f''_{{x_2}{x_1}}(\bar{a})=f''_{{x_1}{x_2}}(\bar{a})$$
\end{theorem}
$\blacktriangleleft$
Без ограничения общности можно считать, что $n=2$ и $f(\bar{x})=f(x_1,x_2)$. Положим

$$\bigtriangleup^2 f=f(a_1+h_1,a_2+h_2)-f(a_1+h_1,a_2)-(a_1,a_2+h_2)+f(a_1,a_2),$$
$$ \varphi (x)=f(x,a_2+h_2)-f(x,a_2).$$
Применяя дважды формулу Лагранжа конечных приращений, получим
$$\varphi (a_1+h_1)-\varphi (a_1)=h_1\varphi' (a_1+\theta_1 h_1,a_2)=$$
$$=h_1\left(f'_{x_1}(a_1+\theta_1 h_1,a_2+h_2)-f'_{x_1}(a_1+\theta_1 h_1,a_2) \right)=$$
$$=h_1 h_2 f''_{{x_1}{x_2}}(a_1+\theta_1 h_1,a_2+\theta_2 h_2).$$
В силу непрерывности функции $f''_{{x_1}{x_2}}(x_1,x_2)$ в точке $\bar{x}=\bar{a}$ имеем
$$\varphi (a_1+h_1)- \varphi (a_1)=h_1 h_2 (f''{{x_1}{x_2}}(a_1,a_2)+o(1)).$$
С другой стороны,
$$\varphi (a_1 + h_1)- \varphi (a_1)=\psi (a_2+h_2)-\psi (a_2),$$
где $\psi (x)=f(a_1+h_1,x)-f(a_1,x).$

\parindent=1cm
Вновь применяя теорему Лагранжа, находим
$$\psi (a_2+h_2)-\psi (a_2)=h_2(f'_{x_2}(a_1+h_1,a_2+\theta'_2)-f'_{x_2}(a_1,a_2+\theta'_2))=$$
$$=h_1 h_2 f''_{{x_2}{x_1}}(a_1+\theta'_1 h_1,a_2+\theta'_2 h_2)=h_1 h_2(f''_{{x_2}{x_1}}(a_1,a_2)+o(1)).$$

Отсюда
$$h_1 h_2(f''_{{x_1}{x_2}}(a_1, a_2)+o(1))=h_1 h_2(f''_{{x_2}{x_1}}(a_1, a_2)+o(1))$$
т.е. получаем справедливость равенства
$$f''_{{x_1}{x_2}}(a_1, a_2)=f''_{{x_2}{x_1}}(a_1, a_2)$$
Теорема 1 доказана.
$\blacktriangleright$

\begin{theorem}
(теорема Юнга). Пусть функции $f'(x_1, x_2)$ и $f'(x_1, x_2)$ определены в некоторой окрестности точки $\bar{x}=\bar{a}=(a_1, a_2)$
и дифференцируемы в точке $\bar{a}$.Тогда $$f''_{{x_1}{x_2}}(a_1, a_2)=f''_{{x_2}{x_1}}(a_1, a_2)$$.
\end{theorem}
$\blacktriangleleft$
Рассмотрим функцию
$$\bigtriangleup^2 f=f(a_1+h_1,a_2+h_2)-f(a_1+h_1,a_2)-(a_1,a_2+h_2)+f(a_1,a_2),$$
$$ \varphi (x)=f(x,a_2+h_2)-f(x,a_2).$$
Имеем
$$\bigtriangleup^2 f=\varphi (a_1+h)- \varphi (a_1)$$.
Из теоремы Лагранжа следует, что
$$\bigtriangleup^2 f=h \varphi' (a_1+\theta_1 h)=h(f'_{x_1}(a_1+\theta_1 h, a_2+h)-f'_{x_1}(a_1+\theta_1 h, a_2)).$$
В силу того что функция $f'_{x_1}(x_1, x_2)$ дифференцируема в точке $\bar{x}=\bar{a}$, 
$$f'_{x_1}(a_1+\theta_1 h, a_2+h)-f'_{x_1}(a_1, a_2)=\theta_1 h f''_{{x_1}{x_1}}(\bar{a})+h f_{{x_1}{x_2}}(\bar{a})+o(h),$$
$$ f'_{x_1}(a_1+\theta_1 h, a_2)-f'_{x_1}(a_1, a_2)=\theta_1 h f''_{{x_1}{x_1}}(\bar{a})+o(h).$$
Следовательно,
$$\bigtriangleup^2 f=h^2 f''_{{x_2}{x_1}}(\bar{a})+o(h^2).$$
С другой стороны,
$$\bigtriangleup^2 f=\psi (a_2+h)-\psi (a_2), $$
где $\psi(y)=f(a_1+h, y)-f(a_1, y).$ Аналогично предыдущему получим
$$\bigtriangleup^2 f=h^2 f''_{{x_2}{x_1}}(\bar{a})+o(h^2).$$
Таким образом,
$$f''_{{x_1}{x_2}}(a_1, a_2)=f''_{{x_2}{x_1}}(a_1, a_2).$$
Теорема 2 доказана.
$\blacktriangleright$
\begin{theorem}
Теорема Юнга и Шварца имеют место при $n>2$
\end{theorem}
$\blacktriangleleft$
Надо зафиксировать все переменные, кроме $x_r, x_s, $ и применить доказанные теоремы к получившимся функциям.
$\blacktriangleright$
\begin{object}
Функция $f(\bar{a})$ называется дважды дифференцируемой в точке, если все первые производные дифференцируемы. Вообще, функция $f(\bar{a})$
называется $n$ раз дифференцируема, если все частнык производные $(n-1)$-го порядка являются дифференцируемыми функциями.
\end{object}
\begin{theorem}
(достаточное условие дифференцируемости). Для того чтобы функция $f(\bar{a})$ была $n$ раз дифференцируема в точке, достаточно, чтобы
все частные производные порядка $n$ были непрерывны в этой точке.
\end{theorem}
$\blacktriangleleft$
проводится по индукции.
$\blacktriangleright$
\begin{theorem}
(из теоремы Юнга). Если функция $f(\bar{x})$ является $n$ раз дифференцируемой, то смешанные частные производные до порядка $n$ не зависят от порядка, 
в котором производится дифференцирование.
\end{theorem}
$\blacktriangleleft$
получается индукцией из теоремы Юнга.
$\blacktriangleright$

