\setcounter{object}{0}
\setcounter{approval}{0}
\setcounter{example}{0}
\begin{center}

\vskip 6mm
{\S3. ПРЕДЕЛ ПОСЛЕДОВАТЕЛЬНОСТИ}
\\Автор: Керимов Руслан
\end{center} \vskip 2mm
\begin{object}
\slshape Последовательность {\upshape\{$a_n$\}} называется {\bfseries\upshape сходящейся},
если существует число {\upshape$l\in\R$} такое, что последовательность {\upshape$a_n=~a_n-~l$}
является бесконечно малой последовательностью.
\end{object}

В этом случае говорят, что $\{a_n\}$ сходится или что $\{a_n\}$ имеет предел
и этот предел равен $l$. Записывают это так:
$$
\lim_{n\to\infty}a_n=l\text{ или }a_n\to l\text{ при }n\to\infty.
$$

Это определение на ``$\varepsilon$-языке'' можно записать следующим образом:
$$
\forall\varepsilon>0\enskip\exists n_0(\varepsilon),\text{ такое, что }
\forall n>n_0\text{ имеем }|a_n-l|<\varepsilon.
$$

Будем говорить также, что последовательность $\{a_n\}$ расходится к ``плюс
бесконечности'', если для любого $c>0$ лишь для конечного числа членов ее
выполняется неравенство: 
$$
a_n<c.
$$
Обозначается это так:
$$
\lim_{n\to\infty}a_n=+\infty\text{ или }a_n\to+\infty\text{ при }n\to\infty.
$$
Последовательность $\{a_n\}$ расходится к ``минус бесконечности'', если для
любого $b<0$ лишь для конечного числа членов ее выполняется неравенство
$$
a_n>b.
$$
Обозначается это так:
$$
\lim_{n\to\infty}a_n=-\infty\text{ или }a_n\to-\infty\text{ при }n\to\infty.
$$
И, наконец, последовательность $\{a_n\}$ расходится к ``бесконечности'',
если для любого $c>0$ лишь для конечного числа членов ее выполняется неравенство
$$
|a_n|<c.
$$
Обозначается это так:
$$
\lim_{n\to\infty}a_n=\infty\text{ или }a_n\to\infty\text{ при }n\to\infty.
$$

\begin{approval}.Если $\{a_n\}$ сходится, то она имеет единственный предел.
\end{approval}
$\blacktriangleleft$
Пусть это не так. Тогда существуют числа $l_1\ne l_2$ такие, что
последовательности $\alpha_n=a_n-l_1$ и $\beta_n=a_n-l_2$
обе являются бесконечно малыми последовательностями. Отсюда
$\alpha_n+l_1=a_n=\beta_n+l_2$, поэтому $l_2-l_2=\beta_n-\alpha_n$
есть бесконечная малая последовательность. Но тогда по теореме 5 \S2
имеем $l_1-l_2=0$, т.е. $l_1=l_2$.
$\blacktriangleright$

\begin{approval}Если $\{a_n\}$ --- бесконечно малая последовательность, то\\ 
$\lim\limits_{n\to\infty}a_n=0$.
\end{approval}
$\blacktriangleleft$
Действительно, при $l=0$ имеем $\alpha_n-0=a_n$ есть бесконечно малая
последовательность, т.е. предел $\{a_n\}$ при $n\to\infty$ равен $0$.
$\blacktriangleright$

\begin{approval}Если $\{a_n\}$ сходится, то она ограничена.
\end{approval}
$\blacktriangleleft$
Если $\{a_n\}$ сходится, то найдется число $l$ такое, что $\alpha_n=a_n-l$
--- бесконечно малая последовательность. Значит, существует $c>0$ такое,
что при всех натуральных $n$ имеем $|\alpha_n|<c$. Но $a_n=l+\alpha_n$, откуда
$$
|a_n|=|l+\alpha_n|\le|l|+|\alpha_n|\le|l|+c=c_1,
$$
т.е. $\{a_n\}$ --- ограниченная последовательность, что и требовалось доказать.
$\blacktriangleright$

\begin{approval}Если $\lim\limits_{n\to\infty}a_n=l$  и $a_n\ne0$, то существует
$a_0\in\N$, такое, что при всех $n>n_0$ имеем $|a_n|>|l|/2$ (или, что тоже самое)
$1/|a_n|<2/|l|)$.

Это означает, что последовательность $1/a_n$, составленная из обратных величин,
ограничена.
\end{approval}

$\blacktriangleleft$
В силу условия имеем, что $\alpha_n=a_n-l$ --- бесконечно малая последовательность.
Тогда вне $|l|/2$-окресности нуля лежит только конечное число членов последовательности
$\{a_n\}$. Пусть $n_0$ --- самое большое значение номера таких членов; тогда при всех
$n>n_0$ имеем $|\alpha_n|<|l|/2$. Отсюда при этих $n$ получим ($l=a_n-\alpha_n$)
$$
|l|=|a_n-\alpha_n|\le|a_n|+|-\alpha_n|=|a_n|+|\alpha_n|.
$$
Следовательно,
$$
|a_n|\ge|l|-|\alpha_n|>|l|-\cfrac{|l|}{2}=\cfrac{|l|}{2},
$$
что и требовалось доказать.
$\blacktriangleright$

\begin{approval}Если $a_n\to l_1$, $b_n\to l_2$ при $n\to\infty$, то 
$c_n=a_n\pm b_n\to l_1\pm l_2$ при $n\to\infty$. Другими словами, 
для сходящихся последовательностей предел их суммы равен сумме их пределов.
\end{approval}
$\blacktriangleleft$
Из условия имеем $\alpha_n=a_n-l_1$, $\beta_n=b_n-l_2$ --- бесконечно малые 
последовательности. Следовательно,
$$
c_n-(l_1\pm l_2)=(a_n\pm b_n)-(l_1\pm l_2)=\alpha_n\pm \beta_n=\gamma_n
$$
--- бесконечно малая последовательность. Значит, из определения предела имеем
$$
\lim_{n\to\infty}c_n=l_1\pm l_2,
$$
что и требовалось доказать.
$\blacktriangleright$

\begin{approval}Если $a_n\to l_1$, $b_n\to l_2$ при $n\to\infty$, то 
$c_n=a_nb_n=l_1l_2$ при $n\to\infty$ (предел произведения равен 
произведению пределов).
\end{approval}
$\blacktriangleleft$
Имеем $a_n=l_1+\alpha_n$, $b_n=l_2+\beta_n$, $c_n=a_nb_n=l_1l_2+\alpha_n l_2+
\beta_n l_1+\alpha_n\beta_n=l_1l_2+\gamma_n$. Но $\gamma_n$ --- бесконечно
 малая последовательность, так как она есть сумма трех последовательностей,
 каждая из которых есть бесконечно малая последовательность. Отсюда:
 $$
 \lim_{n\to\infty}c_n=l_1l_2.
 $$
 Доказательство закончено.
$\blacktriangleright$

\begin{approval}Пусть $\lim\limits_{n\to\infty}a_n=l_1$, $\lim\limits_{n\to\infty}
b_n=l_2$, $l_2\ne0$. Тогда $\lim\limits_{n\to\infty}\cfrac{a_n}{b_n}=
\cfrac{l_1}{l_2}$, т.е. если предел знаменателя не равен нулю, то предел
отношения равен отношению пределов.
\end{approval}
$\blacktriangleleft$
Рассмотрим последовательности $c_n=\cfrac{a_n}{b_n}$ и $\gamma_n=
c_n-\cfrac{l_1}{l_2}=\cfrac{a_n}{b_n}-\cfrac{l_1}{l_2}=\cfrac{a_nl_2-b_n
l_1}{b_nl_2}$, $a_n=l_1+\alpha_n=a_n-l_1$, $\beta_n=b_n-l_2$.
Из условия вытекает, что $\alpha_n$, $\beta_n$ есть бесконечно малая 
последовательность. Нам достаточно доказать, что тоже является бесконечно
малая последовательность. Для этого запишем $\gamma_n$ в виде
$$
\gamma_n=\cfrac{(l_1+\alpha_n)l_2-(l_2+\beta_n)l_1}{b_nl_2}=
\cfrac{\alpha_nl_2-\beta_nl_1}{l_2}\cdot\cfrac{1}{b_n}.
$$
Теперь заметим, что последовательность $\cfrac{\alpha_nl_2-\beta_nl_1}{l_2}$
является бесконечно малой в силу утверждений $5$ и $6$, а 
последовательность $1/b_n$ ограничена в силу утверждения $4$. Но тогда по
теореме $4$ \S2 последовательность $\gamma_n$ является бесконечно малой.
Таким образом, $\lim\limits_{n\to\infty}c_n=l_1/l_2$, что и требовалось
доказать.
$\blacktriangleright$

\vskip 3mm
\begin{example} \itshape Сумма членов бесконечной убывающей 
геометрической прогрессии. \linebreak[4] Пусть $s_n=a+aq+\ldots+aq^{n-1}$. Тогда
$$
qs_n=aq+\ldots+aq^{n-1}+aq^n,\;s_n=\cfrac{a-aq^n}{1-q}.
$$
Так как при $|q|<1$ имеем $\{q^n\}$ --- бесконечно малая последовательность, то
$$
s=\lim_{n\to\infty}s_n=\cfrac{a}{1-q}.
$$

Заметим, что величину $s$ можно представить в виде
$$
s=\lim_{n\to\infty}s_n=\frac{a}{1-q}.
$$
где $s_n=\sum_{k=1}^naq^{k-1}$ называется $n$-{\bfseries й частичной 
суммой ряда,} а величина $r_n=s-s_n$ --- {\bfseries остатком ряда.}
\end{example}
