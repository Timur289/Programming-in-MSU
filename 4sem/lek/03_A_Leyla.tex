\setcounter{object}{0}
\setcounter{approval}{0}
\setcounter{theorem}{0}
\setcounter{example}{0}

\chapter{Сходимость функционального ряда}
\centerline{ \bf Автор: Алиева Лейла} \vskip 1cm

\section{Сходимость функционального ряда}
\ Понятия функциональной последовательности и функционального ряда связаны между собойтак же тесно, как и в обычном числовом случае. С этими понятиями мы, по существу, уже ранее встречались. Примерами могут служить бесконечная геометрическая прогрессия
$$\sum^{\infty}_{n=1}q^n=\frac{q}{1-q}\quad\mbox{при}\quad |q|<1 $$
или дзета-функция Римана 
$$\zeta(s)=\sum^{\infty}_{n=1}\frac{1}{n^s}\quad\mbox{при}\quad s>1.$$
Если в первом случае зафиксировать $q$, а во втором $s$, то мы получим обычные числовые ряды. Но эти же параматры можно рассматривать как аргументы числовых функций, и тогда суммы рядов тоже будут представлять собойнекоторые числовые функции. Подобные соображения приводят нас к следующим определениям.\smallskip

\begin{object} Функциональной последовательностью называется занумерованное множество функций $\{ f_n(x) \}$, имеющих одну и ту же область определения $D\subset \R$. При этом множество $D$ называется областью определения функциональной послдовательности $\{ f_n(x) \}$.\end{object}\smallskip

Здесь термин "занумеровать" значит "поставить во взаимо однозначное соответствие с натуральным рядом $\mathbb{N}$".

\begin{object} Пусть $\{a_n(x)\}$ -- некоторая функциональная последовательность (ф.п.), определенная на множестве $D$. формальная бесконечная сумма вида $$ a_1(x)+a_2(x)+a_3(x)+\ldots=\sum^{\infty}_{n=1}a_n(x),$$ или просто $\Sigma a_n(x)$, называется \textbf{функциональным рядом,} определенным на $D$.\end{object}\smallskip

Фиксируя какое-либо значение $x=x_0\in D$, получаем обычный числовой ряд $\Sigma a_n(x_0)$. Как и в числовом случае, определим понятиечастичной суммы функционального ряда.\smallskip

\begin{object} При всех $n\in N$ функции $A_n(x)=a_1(x)+a_2(x)+a_n(x)=\sum^{n}_{k=1}a_k(x)$ называется ($n$-й) \textbf{частичной суммой} функционального ряда $\Sigma a_n(x)$ и $a_n(x)$ -- его \textbf{общим членом}.\end{object}\smallskip

В дальнейшем пусть $D$ обозначает область определения функционального ряда $\Sigma a_n(x)$, т.е. последовательности $\{ A_n(x)\}$.\smallskip

\begin{object} Если при фиксированном $x=x_0\in D$ сходится числовой ряд $\Sigma a_n(x_0)$, то говорят, что функциональный ряд $\Sigma a_n(x)$ \textbf{сходится в точке} $x=x_0$.\end{object}\smallskip

\begin{object} Множество $D_0\subset D$, состоящее из тех точек $x_0$, в которых ряд $\Sigma a_n(x)$ (или последовательность $A_n(x)$) сходится, называется \textbf{областью сходимости} этого ряда (или этой последовательности).\end{object}\smallskip

\textit{Замечание.} Область сходимости функционального ряда обычно бывает уже, чем область ее определения. Пример -- бесконечная геометрическая прогрессия $\frac{q}{1-q}=\sum^{\infty}_{n=1}q^n$.\smallskip

\begin{object} Пусть $D_0$ -- область сходимости функциональной последовательности $\{ A_n(x)\}$ и пусть $A(x)$ есть предельное значение этой последовательности при фиксированном хначении $x\in D_0$. Тогда множество пар $(x,A(x))$ при всех $x\in D_0$ задает некоторую функцию $y=A(x),$ определенную на всем множестве $D_0$. Эта функция называется \textbf{предельной функцией} функциональной последовательности $\{ A_n(x)\}$. Если при этом $A_n(x)$ -- последовательность частичных сумм ряда $\Sigma a_n(x)$, то функция $A(x)$ называется \textbf{суммой} этого ряда. Итак, сумма функционального ряда -- это некоторая функция, определенная на его области сходимости. При $x\in D_0$ \textbf{остаток ряда} $r_n(x)$ тоже представляет собой некоторую функцию от $x$, $r_n(x)=A(x)-A_n(x),$ причем $r_n(x)\to 0$ при $n\to \infty$ и при любом $x\in D_0$.\end{object}\smallskip

Многие свойства суммы $A(x)$ такие, например, как непрерывность суммы ряда $\Sigma a_n(x)$, связаны с поведением его остатка $r_n(x)$ при $n\to \infty$. Для описания этого поведения далее будет введено важное понятия равномерной сходимости функциональных рядов и функциональных последовательностей на множестве. Для того чтобы подчеркнуть отличие от него введенного выше понятия простой сходимости, последнюю еще называют \textbf{поточечной сходимостью.}

Важные примеры функциональных рядов возникают из разложения различных функциий по формуле Тейлора. Например, разлагая в точке $x_0=0$ функцию $y=\sin x$ при $x \in \R$, имеем $$\sin x=x-\frac{x^3}{3!}+\cdots+(-1)^n-1\frac{x^{2n-1}}{(2n-1)!}+r_n(x),$$ где $r_n(x)$ --остаточный член формулы. Записывая его в форме Лагранжа, получим $$r_n(x)=\frac{x^{2n}}{(2n)!}\sin^{2n} z$$ при некоторой точке $z$ с условием, что она лежит между точками $0$ и $x$. Отсюда $$|r_n(x)|\le\frac{|x|^{2n}}{(2n)!}.$$ Но при $n>x^2$ имеют место следующие неравенства: $$(2n)!>n^{n+1},\quad \frac{x^{2n}}{(2n)!}<\frac{x^2}{n^2}<\frac{1}{n},$$ т.е. $r_n(x)\to 0$ при $n\to \infty$.

Таким образом, полагая $$a_n(x)=\frac{(-1)^{n-1}x^{2n-1}}{(2n-1)!},$$ при всех $x \in \R$ имеем разложение $\sin x=\sum^{\infty}{n=1}a_n(x).$

\begin{object} Степенной ряд $\sum^{\infty}_{n=0}\frac{f^{(n)}(a)}{n!}(x-a)^n$ назывется \textbf{рядом Тейлора} функции $f(x)$ в точке $x=a$, а также \textbf{разложением} функции $f(x)$ в ряд Тейлора в этой точке.\end{object}\smallskip

\textbf{Примеры} рядов Тейлора для некоторых функций:

$1) e^x=\sum^{\infty}_{n=0}\frac{x^n}{n!}(\forall x\in R);$\smallskip

$2)\ln(1+x)=\sum^{\infty}_{n=1}(-1)^{n-1}\frac{x^n}{n}(-1<x\le1);$\smallskip

$3)\sin x=\sum^{\infty}_{n=1}(-1)^{n-1}\frac{x^{2n-1}{(2n-1)!}}(\forall x\in R);$\smallskip

$4)\cos x=\sum^{\infty}_{n=0}(-1)^{n-1}\frac{x^{2n}}{(2n)!}(\forall x\in R);$\smallskip

$5) (1+x)^\alpha=1+\sin x=\sum^{\infty}_{n=1}\frac{\alpha(\alpha-1)\ldots(\alpha-n+1)}{n!}x^n (-1<x\le1);$\smallskip

$6) \arctan x=\sum^{\infty}_{n=1}(-1)^{n-1}\frac{x^{2n-1}}{2n-1}(|x|\le1);$\smallskip

$7) \arcsin x=x+\sum^{\infty}_{n=1}\frac{(2n-1)!!}{(2n)!!}\cdot\frac{x^{2n+1}}{2n+1}(|x|\le1).$


\section{Равномерная сходимость}


\begin{object} Пусть последовательность функций $\{ r_n(x)\}$ сходится к нулю при всех $x \in M$. Тогда говорят, что $r_n(x)$ \textbf{сходится к нулю на множестве $M$}, если для любого $\varepsilon >0$ найдется такой номер $n_0=n_0(\varepsilon)$, что при всех $n>n_0$ и одновременно при всех $x \in M$ выполнено неравенство $|r_n(x)|<\varepsilon$.\smallskip

В этом случае используют обозначение: $r_n(x)\mathop{\rightrightarrows}\limits_{M} 0$ при $n\rightarrow \infty$.\end{object}\smallskip

\textit{Замечание.} Слово "одновременно" в этом определении вообще говоря, является избыточным и его можно опустить, однако оно обращает внимание на главное отличие равномерной сходимости от поточечной, состоящее в том, что в первом случае число $n_0(\varepsilon)$ в определении предела одно и то же для всех точек $x \in M$, а во втором случае оно может зависеть ещё и от $x$, т.е. $n_0(\varepsilon)=n_0(\varepsilon,x)$.\smallskip
\begin{object} Если функция $A(x)=A_n(x)+r_n(x)$, где $r_n\mathop{\rightrightarrows}\limits_{M} 0$ при $n\rightarrow \infty$, то последовательность $A_n(x)$ называют \textbf{равонмерно сходящейся к функции $A(x)$ на множестве $M$} при $n\rightarrow \infty$ и это обзначают так:
\begin{center}$A_n(x)\mathop{\rightrightarrows}\limits_{M} A(x)$ при $n\rightarrow \infty$.\end{center}\smallskip

Символ $M$ здесь можно опустить, если по смыслу понятно, о каком множестве идет речь. Далее, если при этом $A_n(x)$ -- последовательность частичных сумм ряда $\Sigma a_n(x)$, то этот ряд называют \textbf{равномерно сходящейся к $A(x)$ на множестве $M$}.\end{object}

Важность введенного понятия равномерной сходимости видна на примере следующей теоремы.\smallskip


\begin{theorem} Пусть каждая из функций $a_n(x)$ непрерывна в точке $x_0 \in \R$ и ряд $\Sigma a_n(x)$ равномерно сходится к функции $A(x)$ на интервале $I=(x_0-\delta,x_0+\delta)$, где $\delta>0$ -- некоторое фиксированное число. Тогда сумма $A(x)$ является непрерывной функцией в точке $x=x_0$.\end{theorem}

$\blacktriangleleft$ По определению равномерной сходимости имеем
\begin{center}$A(x)=A_n(x)+r_n(x), r_n(x)\mathop{\rightrightarrows}\limits_{I}0 (n\to \infty),$ $A_n(x)= \sum^{n}_{k=1}a_k(x), r_n(x)=\sum^{\infty}_{k=n+1}a_k(x).$
\end{center}
Используя обозначение $\Delta f(x)=f(x)-f(x_0)$, где $f(x)$ -- любая функция, получим
\begin{center}$\Delta A(x)=\Delta A_n(x)+\Delta r_n(x)=\Delta A_n(x)+ r_n(x)-r_n(x_0).$
\end{center}
Отсюда
\begin{center}$|\Delta A(x)|\le |\Delta A_n(x)|+|r_n(x)|+|r_n(x_0)|.$
\end{center}
Поскольку $r_n(x)\mathop{\rightrightarrows}\limits_{I}0 (n\to \infty)$ при любом 
$ \varepsilon_1>0 $
найдется номер $n_0=n_0(\varepsilon_1)$ такой, что для всех $n>n_0$ и для всех $x \in I$ имеем
\begin{center}$|r_n(x)|<\varepsilon_1, |r_n(x_0)|<\varepsilon_1.$
\end{center}
Заметим теперь, что функция $A(x)$ непрерывна в точке $x=x_0$, поэтому для любого $ \varepsilon_1>0 $ найдется $\delta_1=\delta_1(\varepsilon_1)>0$ такое, что при всех $x$ с условием $|x-x_0|<\delta_1$ выполнено неравенство
\begin{center}$|\Delta A_n(x)|=|A_n(x)-A_n(x_0)|<\varepsilon_1.$
\end{center}
Теперь при заданном $\varepsilon>0$ можно взять $\varepsilon_1=\varepsilon/3$, и тогда при всех $x$ с условием $|x-x_0|<\delta(\varepsilon)=\delta_1(\varepsilon_1)$ и при $n=n_0(\varepsilon_1)+1=n_0(\varepsilon)$ получим
\begin{center}$|\Delta A(x)|\le |\Delta A_n(x)|+|r_n(x)|+|r_n(x_0)|<\varepsilon_1+\varepsilon_1+\varepsilon_1=\varepsilon.$
\end{center}
Но это и означает, что функция $A(x)$ непрерывна в точке $x=x_0$. $\blacktriangleright$

Далее рассмотрим некоторые простые свойства равномерно сходящихся функциональных последовательностей.\smallskip

\begin{object} Последовательность функций $\{ A_n(x)\}$ называется \textbf{равномерно ограниченной на множестве $M$}, если существует такое число $C$, что при всех $n\in \mathbb{N}$ и при всех $x \in M$ имеем
\begin{center}$|A_n(x)|<C.$\end{center}\end{object}

\begin{approval}Равномерно сходящаяся на множестве $M$ последовательность $A_n(x)$, состоящая из ограниченных на $M$ функций, является равноиерно ограниченной на $M$.\end{approval}
$\blacktriangleleft$ Пусть $B_m = \mathop{sup}\limits_{x\in M}|A_m(x)|$ для каждого натурального числа $m$. В определении равномерной сходимости возьмем $\varepsilon=1$. Тогда при всех достаточно больших $n>n_0$ и при всех $x\in M$
\begin{center}$|A(x)-A_n(x)|<1,\quad |A(x)|\le |A_n(x)|+1\le B_m+1.$
\end{center}
Это значит, что $A(x)$ ограничена.

Далее, пусть $B_0=\mathop{sup}\limits_{x\in M}|A(x)|,B= \mathop{max}\limits_{0\le k \le n_0}B_k$. Положим $C=B+1$. Тогда при $k \le n_0$ справедлива оценка
\begin{center}$|A_k(x)|\le B<B+1=C,$
\end{center}
а при $k>n_0$ имеем
\begin{center}$|A_k(x)|\le|A(x)-(A(x)-A_k(x))|\le|A(x)|+|A(x)-A_k(x)|\le B_0+1\le B+1=C.$ $\blacktriangleright$
\end{center}
Попутно доказано еще одно утверждение.\smallskip

\begin{approval} Если функция $A(x)$ является ограниченной на множестве $M$ и $A_n(x)\mathop{\rightrightarrows}\limits_{M} A(x)$, то при некотором $n_0\in \mathbb{N}$ функциональная последовательность $B_n(x)=A_{n_0+n}(x)$ равномерно ограничена на $M$.\end{approval}\smallskip

Следующие два утверждения приведем без доказательства, посколькуони доказываются точно так же, как и в аналогичных случаях для числовых рядов.\smallskip

\begin{approval}Пусть при $n\to \infty$ имеем
\begin{center}$a_n(x)\mathop{\rightrightarrows}\limits_{M}a(x), \quad b_n(x)\mathop{\rightrightarrows}\limits_{M}b(x).$
\end{center}
Тогда:

$1^0.a_n(x)+b_n(x)\mathop{\rightrightarrows}\limits_{M}a(x)+b(x)$;

$2^0.$ если $|b(x)|<C$ при некотором $C<0$ и всех $x \in M$, то

\begin{center}$a_n(x)\cdot b_n(x)\mathop{\rightrightarrows}\limits_{M}a(x)\cdot b(x);$
\end{center}

$3^0.\frac{a_n(x)}{b_n(x)}\mathop{\rightrightarrows}\limits_{M}\frac{a(x)}{b(x)},$ если только $1/|b(x)|>C>0$ при всех $x\in M.$\end{approval}

\begin{approval}Если последовательность $d_n(x)$ является равномерно ограниченной и $r_n(x)\mathop{\rightrightarrows}\limits_{M}0$ при $n\to \infty$, то $d_n(x)r_n(x)\mathop{\rightrightarrows}\limits_{M}0$ при $n\to \infty$.\end{approval}

