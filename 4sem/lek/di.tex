\documentclass[12pt,titlepage]{report}
\usepackage[russian]{babel}
\usepackage{amsmath}
\usepackage{amssymb}
\usepackage{amsthm}
\usepackage{textcomp}
\usepackage[utf8]{inputenc}
%\usepackage[koi8-r]{inputenc}
\usepackage[dvips]{graphics,epsfig}
%\usepackage{weird,queer,latexsym}
\usepackage{latexsym} 
\pagestyle{empty}
\frenchspacing
\textwidth=16cm
\textheight=25cm
\topmargin=-0.5in
\oddsidemargin=0mm
\newcommand\ds{\displaystyle}
\newtheorem{object}{Определение}
\newtheorem{approval}{Утверждение}
\newtheorem{theorem}{Теорема}
\newtheorem{example}{Пример}
\newtheorem{lemma}{Лемма}
\newtheorem{task}{Задача}
\newtheorem{sledstvie}{Следствие}
\addto\captionsrussian{\renewcommand\chaptername{Лекция}}
\def\R{\mathbb R}
\def\N{\mathbb N}
\def\dx{{\rm d}x}
\def\dxy{{\rm d}xy}

\setcounter{object}{0}
\setcounter{approval}{0}
\setcounter{theorem}{0}
\setcounter{example}{0}
\begin{document}
\chapter{Криволинейные интегралы}
\centerline{\bf Автор: Максимова Диана}
\vskip 3mm
\ \par Криволинейные интегралы --- это интегралы по кривой $L$ в $n$--мерном пространстве. Мы рассмотрим два вида
таких интегралов : интегралы первого и второго рода. 
\par Сделаем некоторые допущения. Пространство, в котором задана кривая L, будем для простоты считать двумерным. 
Саму кривую L будем считать кусочно-гладкой, т.е. ее можно разбить на конечное число гладких кусков (участков).
 Будем рассматривать только один такой кусок.
\par Как известно, кривая $L$ является образом некоторого отрезка $[a,b]$ при отображении $\overline r=
\overline r (t), t\in[a,b]$, где $\overline r (t)=(x(t),y(t))$, причем $x(t)$ и $y(t)$ --- гладкие функции
на отрезке $[a,b]$. Кроме того, внутренние точки отрезка переходят во ``внутренние'' точки кривой, а концы
отрезка --- точки $a$ и $b$ --- переходят в граничные точки кривой $A$ и $B$, т.е. $\overline r(a)=A,\overline r(b)=B$.
 Будем предполагать, что кривая $L$ невырожденна, т.е. не содержит особых точек. Другими словами, для любого 
$t\in[a,b]$ вектор $\overline r (t)$ отличен от нуля.
\par Пусть $T$ --- размеченное разбиение отрезка $[a,b] : a = t_0 < t_1 < \ldots < t_m = b, \xi_1,\ldots,\xi_m$ 
--- точки разметки, $x_k = x(\xi_k),y_k = y(\xi_k), k = 1,\ldots,m;\Delta l_k$ --- длина части кривой $L$, 
которая является образом отрезка $\Delta_k = [t_{k-1},t_k]$. Для рассматриваемой кривой $L$ длина кривой выражается
 по формуле
$$
\Delta l_k = \int\limits_{t_{k-1}}^{t_k}\sqrt {(x'(t))^2+(y'(t))^2}dt.
$$
\par Пусть функция $g(x,y)$ определена на кривой $L$.
\vskip 1mm
\begin{object}
Если существует предел при $\Delta T\to0$ интегральных сумм
$$
\sigma_1(T)=\sum\limits_{k=1}^m g(x_k,y_k)\Delta l_k,
$$
то он называется \textbf{интегралом первого рода} от функции $g(x,y)$ по кривой $g(x,y)$. 
\end{object}
\par Этот интеграл обозначается так:
$$
I_1=\int\limits_L g(x,y)dl.
$$
\par Рассмотрим интегральную сумму $\sigma_2(T)$, где
$$
\sigma_2(T)=\sum\limits_{k=1}^m g(x_k,y_k)\Delta x_k,\Delta x_k=x(t_k) - x(t_{k-1}).
$$
\begin{object}
Если существует предел $I_2$ интегральных сумм $\sigma_2(T)$ при $\Delta_T\to0$, то он называется \textbf{интегралом второго рода} от функции $g(x,y)$ в направлении от $A$ до $B$.
\end{object}
\par Обозначается этот интеграл символом
$$
I_2=\int\limits_{AB} g(x,y)dx.
$$
Аналогично определяется еще один интеграл второго рода
$$
I_3=\int\limits_{AB}g(x,y)dy.
$$
Для интегралов $I_2$ и $I_3$ обычно употребляются обозначения
$$
I_2=\int\limits_{AB} P(x,y)dx,\quad I_3=\int\limits_AB Q(x,y)dy.
$$
Выражение
$$
I=\int\limits_{AB} P(x,y)dx+Q(x,y)dy
$$
называется \textbf{общим криволинейным интегралом второго рода} по кривой $L=AB$ от линейной дифференциальной
формы $Pdx+Qdy$ (здесь кривая обозначается своими начальной и конечной точками).
\end{document}
