\setcounter{object}{0}
\setcounter{approval}{0}
\setcounter{theorem}{0}
\setcounter{example}{0}
\chapter{Равномерная сходимость несобственных параметрических интегралов}
\centerline{\bf Автор: Кеберлинский Эльнур}\vskip 1cm


\vskip 1cm
Дальнейшее развитие теории интегралов,зависящих от параметра,приводит к рассмотрению несобственных интегралов,которые составляют ее наиболее существенную часть.Из двух типов таких интегралов сосредоточим свое внимание главным образом на интегралах первого рода.Интегралов второго рода коснемся лишь вскользь, поскольку их теория не имеет принципиальных отличий от интегралов первого рода.
\vskip 1cm
Рассмотрим функцию $f(x,y)$, заданную на множестве $I \in Y $, где I-промежуток вида $[a,+\infty)$, а Y-некоторое множество вещественных чисел, т.е. $ Y \in \R.$ Допустим,что при любом фиксированном $y \in Y $ функция $f(x,y) $ интегрируема на конечном отрезке вида $ [a,b] $ и существует несобственный интеграл первого рода от этой функции по переменной $ x \in Y=[a,+\infty)$. Тогда этот интеграл сам представляет собой некоторую функцию от $y$, заданную на $ Y$ равенством
$$  g(y)=\int_a^\infty {f(x,y)\dx}.$$

\begin{object}
Функция $g(y)$, представленная в указанном выше виде, называется несобственным интегралом первого рода, зависящим от параметра $y \in Y$.
\end{object}

Замечание. Вместо несобственных интегралов по промежутку вида $[a,+\infty)$ можно,
разумеется,
рассматривать интегралы по промежуткам вида $(-\infty,b]$ или по всей вещественной прямой $\R =(-\infty,+\infty)$. 
Все эти случаи сводятся к рассмотренному точно так же,
как это делалось при изучении обычных несобственных интегралов. 
Например, интеграл
$$\int_{-\infty}^{+\infty} {f(x,y) \dx},y \in Y,$$
достаточно представить в виде суммы интегралов
$$\int_{-\infty}^{+\infty} {f(x,y) \dx}=\int_{-\infty}^0 {f(x,y)\dx} + \int_0^{+\infty} {f(x,y)\dx}$$
и сходимость этой суммы понимать как сходимость каждого из двух ее слагаемых. Первое слагаемое сводится ко второму заменой переменной $x$ на $-x$. Кроме того, можно, конечно, рассматривать и формальные несобственные параметрические интегралы и при этом ставить вопрос об области их сходимости $Y$. Подобного рода вопросы разобраны при рассмотрении функциональных рядов, поэтому мы им много внимания уделять не будем, иногда, однако, будем пользоваться аналогичной терминологией.

\begin{example}
При $y>1$ справедливо равенство 
$$\int_1^\infty {\frac {\dx} {x^y}}=\lim_{t\to +\infty}\int_1^t{\frac {\dx} {x^y}}=\lim_{t\to +\infty} 
\left.\frac {x^{1-y}} {1-y}\right|_1^t=\frac {1} {1-y}$$
\end{example}
\begin{example}
При $y>0$ имеем
$$\int_0^\infty {\frac {\sin xy} {x} \dx}=\int_0^\infty {\frac {\sin xy} {xy}\dxy}=\int_0^\infty {\frac {\sin x} {x}\dx}.$$
\end{example}
\begin{object}
Интеграл $\int_a^\infty {f(x,y)\dx}\quad $ называется равномерно сходящимся по параметру $y$ на множестве $Y,\{y\}=Y$, если
%$$\int_a^t f(x,y)\dx=F(y,t)\stackrel{Y}{\rightrightarrows} g(y) при t\to +\infty. $$
Другими словами, это значит, что для любого $\varepsilon $>0 существует $t=t_0(\varepsilon)$ такое, что при всех $t>t_0(\varepsilon)$ и всех $y\in Y$ имеем

$$ \left|\int_a^t {f(x,y)\dx}-g(y)\right|<\varepsilon,$$

где $g(y)=\int_a^\infty {f(x,y)\dx}$.
\end{object}
Исходя из общей теоремы сформулируем критерий Коши кокретно для равномерной сходимости несобственных интегралов первого рода.
\begin{theorem} 
Необходимое и достаточное условие равномерной сходимости несобственного интеграла первого рода $\int_a^\infty {f(x,y)\dx}$ на множестве $Y$ состоит в том, чтобы для любого $\varepsilon >0$ существовало $T=T(\varepsilon)$ такое, что при всех $t_2>t_1>T$ и любом $y\in Y$ выполнялось бы неравенство

$$\left|\int_{t_1}^{t_2} {f(x,y)\dx}\right| < \varepsilon.$$
\end{theorem}
Приведем также прямую формулировку критерия отсутствия равномерной сходимости несобственного параметрического интеграла.

\begin{theorem}
Равномерная сходимость несобственного интеграла
$$\int_a^\infty {f(x,y)\dx}$$
на множестве $Y$ не имеет места, если найдется $\varepsilon >0$ такое, что для любого $T\in \R$ найдутся числа $t_1$ и $t_2>T$ и $y\in Y$ такие, что

$$\left|\int_{t_1}^{t_2} {f(x,y)\dx}\right|\ge \varepsilon$$ 
\end{theorem}
\begin{object}
Если интеграл $\int_a^\infty {g(x)\dx}$ сходится и при всех $x>a, y\in Y$ имеем $|f(x,y)|\le g(x)$, то функция $g(x)$ называется \textbf{мажорантой} для $f(x,y)$ на $\Pi =I\times Y$.
\end{object}
\begin{theorem}
(признак Вейерштрасса равномерной сходимости несобственных интегралов первого рода). Интеграл $J=\int_a^\infty {f(x,y)\dx}$ сходится равномерно на $Y$, если функция $f(x,y)$ имеет мажоранту $g(x)$ на $\Pi=X\times Y$, где $X=[a,+\infty)$.
\end{theorem}
$\blacktriangleleft$
Воспользуемся критерием Коши. Поскольку интеграл $\int_a^\infty {g(x)\dx}$ сходится, при любом $\varepsilon >0$ найдется число $T=T(\varepsilon)$ такое, что при всех $t_2>t_1>T$ выполнено неравенство
$$\int_{t_1}^{t_2} {g(x)\dx}<\varepsilon.$$
Но тогда при всех $y\in Y$ имеем
$$\left|\int_{t_1}^{t_2}{f(x,y)\dx}\right|\le \int_{t_1}^{t_2}{|f(x,y)|\dx}\le \int_{t_1}^{t_2}{g(x)\dx}<\varepsilon$$
Отсюда согласно критерию Коши заключаем, что интеграл $J$ сходится равномерно на $Y$. Теорема доказана. 
$\blacktriangleright$
\begin{example} 
При $s>s_0>1$ интеграл $\int_1^\infty{x^{-s}\dx}$ сходится равномерно на множестве $s\ge s_0$, поскольку он имеет мажоранту $g(x)=x^{-s_0}$.
\end{example}

\begin{theorem}
(Признаки Абеля и Дирихле для равномерной сходимости параметрических несобственных интегралов первого рода).
Пусть функция $f(x,y)$ определена на множестве $\Pi =X\times Y$,где $X=[a,+\infty), Y=[c,d]$ и $f(x,y)=\alpha(x,y)\beta(x,y)$. Пусть $\beta (x,y)$ монотонна по $x$ при любом фиксированном $y\in Y$.

(А)(признак Абеля). Пусть, кроме того:

1) интеграл $\int_a^\infty {\alpha (x,y)\dx}$ сходится равномерно по $y$ на $Y$;

\parindent=1cm2)функция $\beta (x,y)$ ограничена на $\Pi =X\times Y$,т.е. $|\beta (x,y)|<c$ при некотором вещественном числе $c>0$ и всех $(x,y)\in \Pi$.

Тогда интеграл $J=\int_a^{\infty}{f(x,y)\dx}$ сходится равномерно на $Y$ 

(Д) (признак Дирихле). Пусть вместо условий (А) имеем:

1) при некотором $c>0$ и всех $t>a, y\in Y$ имеет место неравенство
$$\left|\int_a^t {\alpha (x,y)\dx}\right|<c;$$

2)функция $\beta(x,y)$ равномерно на $Y$ сходится к нулю при $x\to 0$ 

Тогда, как и в случае (А), интеграл $J$ сходится равномерно на $Y$.
\end{theorem}
$\blacktriangleleft$
Эта теорема как по своей формулировке, так и по доказательству похожа на соответствующие утверждения из теории рядов. По существу, все отличие сводится к замене использования преобразования Абеля на применение второй теоремы о среднем значении интеграла.

Для доказательства снова воспользуемся критерием Коши. Применяя вторую теорему о среднем, имеем
$$\int_{t_1}^{t_2}{\alpha(x,y)\beta(x,y)\dx}=\beta(t_1,y)\int_{t_1}^{t_3}{\alpha(x,y)\dx}+\beta(t_2,y)\int_{t_3}^{t_2}{\alpha(x,y)\dx}, $$ 
где $t_3$ - некоторая точка отрезка $[t_1,t_2]$.

Теперь в случае (А) в силу равномерной сходимости интеграла $\int_a^\infty {\alpha(x,y)\dx}$ при любом $\varepsilon>0$ и всех достаточно больших $t_2>t_1>t_0(\varepsilon)$ имеем $\left|\int_{t_1}^{t_3}{\alpha(x,y)\dx}\right|<\varepsilon$ и $\left|\int_{t_3}^{t_2}{\alpha(x,y)\dx}\right|<\varepsilon$, откуда

$$\left|\int_{t_1}^{t_2}{\alpha(x,y)\beta(x,y)\dx}\right|\le |\beta(t_1,y)|\left|\int_{t_1}^{t_3}{\alpha(x,y)\dx}\right|+|\beta(t_2,y)|\left|\int_{t_3}^{t_2}{\alpha(x,y)\dx}\right|\le$$ $$\le c\varepsilon+c\varepsilon=2c\varepsilon,$$
поскольку $|\beta(x,y)|<c$ при всех $x$ и $y$.

В силу произвольности числа $\varepsilon>0$ это влечет за собой равномерную сходимость интеграла $J$ и справедливость утверждения (А).

В случае (Д) интегралы от функции $\alpha(x,y)$ ограничены числом $c$ и $\beta(x,y)$ стремится к нулю равномерно по $y$,  поэтому при всяком $\varepsilon>0$ и достаточно больших $t_2>t_1>t_0(\varepsilon)$ выполнено неравенство $|\beta(x,y)|<\varepsilon$, откуда с учетом предыдущей формулы имеем
$$\left|\int_{t_1}^{t_2}{\alpha(x,y)\beta(x,y)\dx}\right|\le c\varepsilon+c\varepsilon=2c\varepsilon,$$
что влечет за собой справедливость утверждения (Д). Теорема доказана.
$\blacktriangleright$
