\section{Определители небольших порядков.}
{\vskip 4mm
\centerline{\bf Гайыбов Ниджат} \vskip 1cm
 Излагая метод Гаусса, мы не слишком заботились о значениях коэффициентов при главных неизвестных. Важно было лишь то, что эти коэффициенты отличны от нуля. Проведем теперь более аккуратно процесс исключения неизвестных хотя бы в случае квадратных линейных систем небольших размеров. Это даст пищу для размышлений и исходный материал для построения общей теории определителей в гл. 3.


Как и в \S3, рассмотрим систему двух уравнений с двумя неизвестными 
\begin{equation}
\label{f1}
\begin{matrix}
a_{11}x_{1}+a_{12}x_{2}=b_{1}, \\
a_{21}x_{1}+a_{22}x_{2}=b_{2} 
\end{matrix}
\end{equation}
  и постараемся найти общие формулы для компонент $x_1^0,x_2^0$ ее решения.


Назовем {\sl определителем} матрицы $ \begin{Vmatrix} a_{11}& a_{12}\\a_{21}& a_{22}\end{Vmatrix}$ выражение $a_{11} a_{22}-a_{21} a_{12}$ и обозначим его 
$\begin{vmatrix}a_{11}& a_{12}\\a_{21}& a_{22}\end{vmatrix}.$\\
Тем самым квадратной матрице сопоставляется число
\begin{equation}
\label{f2}
\begin{vmatrix}a_{12}&a_{12}\\ a_{21}&a_{22}\end{vmatrix}=a_{11}a_{22}-a_{21}a_{12}.
\end{equation}
Если мы попытаемся исключить $x_{2}$ из системы ($\ref{f1}$), умножив первое уравнение на $a_{22}$ и прибавив к нему второе,умноженное на -$a_{12}$, то получим 
$$\begin{vmatrix}
a_{11}&a_{12}\\
a_{21}&a_{22}
\end{vmatrix}
x_{1}
=b_{1}a_{22}-b_{2}a_{12}.
$$ 
Правую часть также можно рассматривать как определитель матрицы $\begin{Vmatrix}b_{1}&a_{12}\\ a_{21}&a_{22}\end{Vmatrix}.$ Предположим, что $\begin{vmatrix}a_{11}&a_{12}\\ a_{21}&a_{22}\end{vmatrix}\ne0.$ Тогда мы имеем 
\begin{equation}
\label{f3}
x_1= \frac{\begin{Vmatrix}b_1 & a_{12}\\ b_2 & a_{22}\end{Vmatrix}}{\begin{vmatrix} a_{11} & a_{12} \\ a_{21} & a_{22} \end{vmatrix}}, \qquad
x_2=\frac{\begin{vmatrix}a_{11} & b_{1}\\ a_{21} & b_2\end{vmatrix}}{\begin{vmatrix} a_{11} & a_{12} \\ a_{21} & a_{22} \end{vmatrix}}
\end{equation}


Имея формулы для решения системы двух линейных уравнений с двумя неизвестными,мы можем решать и некоторые другие системы(решать системы-значить находить их решения).Рассмотрим,например систему двух однородных уравнений с тремя неизвестными 
\begin{equation}
\label{f4}
\begin{matrix}
a_{11}x_{1}+a_{12}x_{2}+a_{13}x_{3}=0, \\
a_{21}x_{1}+a_{22}x_{2}+a_{23}x_{3}=0.
\end{matrix}
\end{equation}
Нас интересует ненулевое решение этой системы, так что хотя бы одно из $x_{i}$ не равно нулю.Пусть, например, $x_{3}\ne0.$ Разделив обе части на $-x_{3}$ и положив
$y_{1}=-x_{1}/x_{3}$, $y_{2}=-x_{2}/x_{3}$, запишем систему ($\ref{f4}$) в том же виде 
$$
\begin{matrix}
a_{11}y_{1}+a_{12}y_{2}=a_{13}, \\
a_{21}y_{1}+a_{22}y_{2}=a_{23}.
\end{matrix}
$$
что и (1). При предположении $\begin{vmatrix}a_{11}&a_{12}\\a_{21}&a_{22}\end{vmatrix}\ne0$ формулы ($\ref{f3}$) дают 
$$
y_{1}=-\frac{x_1}{x_3}=\frac{\begin{vmatrix}a_{13}&a_{12}\\a_{23}&a_{22}\end{vmatrix}}{\begin{vmatrix}a_{11}&a_{12}\\a_{21}&a_{22}\end{vmatrix}},\qquad
y_{2}=-\frac{x_2}{x_3}=\frac{\begin{vmatrix}a_{11}&a_{12}\\a_{21}&a_{23}\end{vmatrix}}{\begin{vmatrix}a_{11}&a_{12}\\a_{21}&a_{22}\end{vmatrix}}
$$


Неудивительно, что мы нашли из системы (4) не сами $x_1$, $x_2$, $x_3$, а только их отношения: из однородности системы легко следует, что если $(x_1^0,x_2^0,x_3^0)$---решение и $c$---любое число, то $(cx_1^0,cx_2^0,cx_3^0)$ тоже будет решением. Поэтому мы можем положить 
\begin{equation}
\label{f5}
x_{1}=-\begin{vmatrix}a_{13}&a_{12}\\a_{23}&a_{22}\end{vmatrix},\qquad
x_{2}=-\begin{vmatrix}a_{11}&a_{13}\\a_{21}&a_{23}\end{vmatrix},\qquad
x_{3}=\begin{vmatrix}a_{11}&a_{12}\\a_{21}&a_{22}\end{vmatrix}.
\end{equation}
и сказать, что любое решение получается из указанного умножением всех $x_{i}$ на некоторое число $c$. Чтобы придать ответу несколько более симметричный вид, заметим, что всегда $$ \begin{vmatrix}a&b\\c&d\end{vmatrix}=-\begin{vmatrix}b&a\\d&c\end{vmatrix},$$ как это непосредственно видно из формулы ($\ref{f2}$). Поэтому ($\ref{f5}$) можно записать в виде 
\begin{equation}
\label{f6}
x_{1}=\begin{vmatrix}a_{13}&a_{12}\\a_{23}&a_{22}\end{vmatrix},\qquad
x_{2}=-\begin{vmatrix}a_{11}&a_{13}\\a_{21}&a_{23}\end{vmatrix},\qquad
x_{3}=\begin{vmatrix}a_{11}&a_{12}\\a_{21}&a_{22}\end{vmatrix}.
\end{equation}
Эти формулы выведены в предположении, что $\begin{vmatrix}a_{11}&a_{12}\\ a_{21}&a_{22}\end{vmatrix}\ne0.$ Нетрудно проверить, что доказанное утверждение верно, если хоть один из выражения ($\ref{f6}$) определителей отличен от нуля. Если же все три определителя равны нулю, то, конечно, формулы (\ref{f6}) дают
решение (а именно нулевое), но мы не можем утверждать, что все решения получаются из него умножением на число. 

