\chapter{Двойной интеграл Римана как предел по базе}
\centerline{ \bf Автор: Фаталиева Сабина} \vskip 1cm

\section{Определение двойного интеграла Римана}

\ \par Двойной интеграл - это интеграл от функции двух пременных, взятый по обеим переменным \textit{одновременно}. Данная фраза не является определением, она указывает на то, как мы намерены вводить общие понятия определенного интеграла на случай функции двух пременных.
\par Для того, чтобы получить такое обобщение, вспомним, как выглядит определение интегала в одномерном случае, то есть в случае функции $y=f(x)$ от одной переменной $x$, определенной на отрезке $I=[a,b]$ и интегрируемой на нем по Риману. Одно из эквивалентных определений данного понятия можно сформулировать так.
\begin{object} 
Интегралом $\int\limits_{a}^{b}f(x)\,dx$ от ограниченной функции $f(x)$ называется число, равное \textbf{алгебраческой сумме} площадей криволинейных трапеций, обраованных кривой $y=f(x)$ при $x\in{[a,b]}$. При этом в данную сумму входят площади криволинейных трапеций, расположенных над осью абсцисс со знаком +, а под ней - со знаком -.
\end{object}
\par Если мы обощим понятие криволинейной трапеции на случай, скажем, фекции двух пременных $z=g(x,y)$, заданной на прямоугольнике $P=I_1\times I_2=[a_1,b_1]\times [a_2,b_2]$, то и получим одно из возможынх определений двойного интеграла от функции $g(x,y)$ по прямоугольнику P.  Этот интеграл обозначается символом $$\iint_P g(x,y)dxdy=\int_{a_1}^{b_1}\int_{a_2}^{b_2} g(x,y)dxdy.$$
\par Допустим сначала, что $g(x,y)\ge 0$ для всех $(x,y)\in P$. Вместо криволтнейной трапеции рассмотрим пространственную финуру $H$, заключенную между поверхностью $z=g(x,y)$ и плоскостью $z=0$ при $(x,y)\in P$. Другими словами, фигура $H$ состоит изо всех тех точек $(x,y,z)$, для которых $x\in I_1,\quad y\in I_2$, а третья координата $z$ удовлетворяет условию $0\le z\le g(x,y)$. 
\begin{object}
Фигуру $H$ будем называть \textbf{цилиндрической криволинейной фигурой}, порожденной поверхностью $z=g(x,y)$.
\end{object}
\par Если окажется, что эта фигура измерима каким-либо образом, то ее меру $\mu (H)$ можно взять в качестве искомого пределения значения двойного интеграла $$I=\iint_P g(x,y)dxdy=\mu (H).$$
\par Заметим, что если $\mu$ есть мера Жордана, то данное выше определение двойного интеграла будет эквивалентным определению двойного интеграла Римана, которое будет сейчас дано. Можно было бы таким же образом разобрать общий случай, когда функци $g(x,y)$ принимает как положительные, так и отрицательные значения, но мы это сделаем  дальнейшем при доказательстве критерия измеримости по Жордану цилиндрической криволинейной фигуры.   
\par Перейдем теперь к построению терии двойного интеграла Римана по прямоугольнику $P$. Cначала определим понятие цилиндрической фигуры в общем случае. 
\begin{object}
Фигура $H\subset \in{\mathbb R}^3$ называется textbf{цилиндрической криволинейной фигурой}, порожденной поверхностью $z=g(x,y)$, заданной на $P$, если $H$ состоит изо всех таких точек $(x,y,z)$, для которых $(x,y)\in P$, а координата $z$ заключена между числами 0 и $g(x)$, то есть при $g(x,y)\ge 0$ имеем $0\le z\le g(x,y)$, а при $g(x,y)<0$ имеем $g(x,y)\le z\le 0$.
\end{object}
\par Разобьем прямоугольник $P$ на меньшие прямоугольники с помощью прямых, параллельнх осям $Ox$ и $Oy$ и проходящих через точки $a_1=x_0<x_1<c\dots <x_m=b_1$ разбиения $T_x$ на оси $Ox$ и $a_2=y_0<y_1<\cdots <y_n=b_2$ разбиения $T_y$ на оси $Oy$.
\par Прямоугольник $P_{k,l}\subset P$, точки $(x,y)$ которого удовлетворяют условиям $$x\in \Delta_k^{(x)}=[x_{k-1},x_k],\quad y\in \Delta_l^{(y)}=[y_{l-1},y_l],$$ где $\Delta_k^{(x)}$ есть $k$-й отрезок разбиения $T_x$ и $\Delta_l^{(y)}$ - $l$-й отрезок разбиения $T_y$, будем называть элементом разбиения $T$ прямоугольника $P$ с индексом $(k,l)$, а множество всех прямоугольников $P_{k,l}$, $k=1,\ldots ,m,\quad l=1\ldots ,n$ - разбиением $T$ прямоугольника $P$.
\par В каждом прямоугольнике $P_{k,l}$ возьмем точку $A_{k,l}$ с координатами $(\xi_{k,l},\theta_{k,l})$. Множество прямоугольников $P_{k,l}$ и точек $A_{k,l}$ будем называть размеченным разбиением прямоугольника $P$ и будем обозначать его через $V$.
\par Очевидно, что каждому размеченному разбиению $V$ однозначно соответствует разбиение $T$ прямоугольника $P$, получаемое из $V$ отбрасыванием точек "разметки" \\$(\xi_{k,l},\theta_{k,l})$. 
Другими словами, $T$ является функцией от $V$, $T=T(V)$.
\begin{object}
Cумма $$\sigma (V)=\sum\limits_{k=1}^m\sum\limits_{{l=1}_n} g(\xi_{k,l},\theta_{k,l})\Delta x_k\Delta y_l$$ назывется \textbf{интегральной суммой Римана} функции $g(x,y)$, соответствующей (отвечающей) размеченному разбиению $V$ стандартного прямоугольника $P$.
\end{object}
\par Длину диагонали $\sqrt{\Delta x_k^2+\Delta y_{l_2}}$ прямоугольника $P_{k,l}$ будем называть его \textbf{диаметром}.
\begin{object} 
\textbf{Диаметром разбиения} (размеченного $V$ и неразмеченного $T$) прямоугольника $P$ будем называть иаксимальное значение диаметров элементов разбиения $P_{k,l}$. Обозначать его будем символом $\Delta_V$ и, соответственно, $\Delta_T$.
\end{object}
\begin{object} 
Число $I$ называется (двойным) \textbf{интегралом Римана} от ограниченной функции $g(x,y)$ по прямоугольнику $P$, если для $\forall$ $\varepsilon >0$ $\exists$ $\delta =\delta (\varepsilon )>0$ такое, что для $\forall$ размеченного разбиения $V$ прямоугольника $P$ с условием $\Delta_{V}<\delta$ справедливо неравенство $$|\sigma (V)-I|<\varepsilon$$.
\end{object}
\par Здесь $\sigma (V)$ - интегральная сумма для функции $g(x,y)$, котрая соответствует размеченному разбиению $V$. Поэтому последнее неравенство можно записать еще и так: $$|\sum\limits_{k=1}^m\sum\limits_{{l=1}_n} g(\xi_{k,l},\theta_{k,l})\Delta x_k\Delta y_l-I|<\varepsilon .$$
\par В это случае будем говорить, что $g(x,y)$ является \textbf{интегрируемой по Риману} на прямоугольнике $P$.
\par Далее рассмотрим следующие вопросы:
\begin{enumerate}
\item убедимся, что интеграл $I$ есть предел по некоторой базе;
\item определим верхние и нижние суммы Дарбу и докажем критерий Римана интегрируемости функции от двух пременных;
\item установим свойства двойного интеграла, аналогичные свойствам однократного интеграла.
\end{enumerate}
\par Начнем с определения базы множеств $B$ и $B'$. Множество всех неразмеченных разбиений прямоугольника обозначим через $A_P$, а размеченных -  $A_P'$. В качестве окончаний $b'_{\delta}$ базы $B'$ возьмем множество \{$V|\Delta_V<\delta$\}, т.е. множество разбиений, состоящее из тех $V\in A\_P$, для которых диаметр $\Delta_V$ меньше, чем $\delta >0$.
\par Так как $\sigma (V)$ определена всюду на $A'_P$, то очевидно, тогда определение двойного интеграла, данное выше, эквивалентно определению предела $\lim_{B'} \sigma (V)$ по базе $B'$. Проверка справедливости этого утверждения состоит в том, что надо формально выписать определение предела по базе и сравнить его с данным выше определением. Далее базу $B'$ будем обозначать символом $\Delta_V \to 0$.
\par Совершенно аналогично определяем базу $\Delta_T \to 0$ для всех неразмеченных разбиений $A_P$.
\par Отметим, что неразмеченное разбиение $T$ прямоугольника $P$ можно определить и как пару $(T_x,T_y)$, состоящую из неразмеченного разбиения $T_x\quad :\quad a_1=x_0<x_1<\cdots <x_m=b_1$ отрезка $[a_1,b_1]$ на оси $Ox$ и неразмеченного разбиения $T_y\quad : \quad a_2=y_0<y_1<\cdots <y_n=b_2$ отрезка $[a_2,b_2]$ на оси $Oy$. Это разбиение $T$ получается проведением $m+1$ вертикальных прямых $x=x_k$,\quad $k=0,\ldots ,m$ и $n+1$ горизонтальных прямых  $y=y_l$,\quad $l=0,\ldots ,n$. Снова заметим, что если у размеченного разбиения $V$ отбросить разметку точками $(\xi_{k,l},\theta_{k,l})\in P_{k,l}$, то, очевидно, возникает неразмеченное разбиение, которое будем обозначать символом $T(V)=T$.
\begin{object} 
Множество всех размеченных разбиений $\{ V\}$, которым отвечает одно и то же неразмеченное разбиение $T_0$, будем называть \textbf{множеством разметок} $T_0$ и обозначать символом $A'_P(T_0)$. Если $V\in A'_P(T_0)$, то будем говорить, чт $V$ является \textbf{разметкой} $T_0$ или, что то же самое, $T(V)=T_0$.
\end{object}

\section{Суммы Дарбу и их свойства}

\par Переходим теперь к построению textit{теории Дарбу} для двойного интеграла Римана по прямоугольнику.
\par Обозначим для некоторого неразмеченного разбиения $T$ прямоугольника $P$ через $M_{k,l}$ и $m_{k,l}$ величины $$M_{k,l}=\quad \sup_{(x,y)\in P_{k,l}} g(x,y), m_{k,l}=\quad \inf_{(x,y)\in P_{k,l}} g(x,y).$$
\par Тогда \textbf{верхней суммой Дарбу} функции $g(x,y)$, соответствующей разбиению $T$, назывыется сумма $S(T)$, где $$S(t)=\sum_{k=1}^m \sum_{l=1}^n M_{k,l}\Delta x_k\Delta y_l,$$ а сумма $$s(T)=\sum_{k=1}^m \sum_{l=1}^n m_{k,l}\Delta x_k\Delta y_l$$ называетя \textbf{нижней суммой Дарбу}.
\textbf{Омега-суммой} $\Omega (T)$, отвечающей разбиению $T$, назовем величину $$\Omega (T)=S(T)-s(T)=\sum_{k=1}^m \sum_{l=1}^n \omega_{k,l}\Delta x_k\Delta y_l,$$ где $\omega_{k,l}=M_{k,l}-m{k,l}$.
\begin{object}
Число $I*=\quad \inf_{T\in A_P} S(T)$ называетя \textbf{верхним интегралом Дарбу} от функции $g(x,y)$ по прямоугольнику $P$, а число $I_{*}=\quad \sup_{T\in A_P} s(T)$ - \textbf{нижним интгралом Дарбу} от функции $g(x,y)$.
\end{object}
\par Нам потребуются следующие свойства сумм Дарбу.
\begin{lemma} 
Для любого размеченного разбиения $V\in A'_P$ имеем $$s(T(V))\le \sigma (V)\le S(T(V)).$$
\end{lemma}
\begin{lemma} 
Зафиксируем некоторе разбиение $T_0\in A_P$. Будем иметь следующие соотношения $$s(T_0)=\quad \inf_{V\in a'_P()T_0} \sigma (V),S(T_0)=\quad \sup_{V\in a'_P()T_0} \sigma V.$$
\end{lemma} 
\begin{lemma} 
Для любых неразмеченных разбиений $T_1$ и $T_2$ имеем $$s(T_1)\le S(T_2).$$
\end{lemma} 
\begin{lemma} 
\label{lemmfour}
Для ограниченной на прямоугольнике $P$ функции верний $I*$ и нижний $I_{*}$ интегралы Дарбу существуют, причем для любого разбиения $T\in A_P$ справедливы неравенства $$s(T)\le I_{*}\le I*\le S(T).$$
\end{lemma} 
\begin{lemma}
Размеченное разбиение $V$ принадлежит окончанию $b'_{\delta}\in B'$ $\leftrightarrow$ $T(V)\in_{\delta}$.
\end{lemma}
$\blacktriangleleft$ Доказательство лемм аналогично доказательству соответствующих утверждений в одномерном случае и не представляет большого труда. Стоит лишь сказать о лемме 3, поскольку там учавствуют два разных разбиения. Здесь, как и в одномерном случае, введем понятие \textbf{измельчения разбиения}.
\begin{object} 
Неразмеченное разбиение $T_2$ называется \textbf{измельчением разбиения} $T_1$, если разбиение $T_2$ получается из $T_1$ добавлением конечного числа новых точек разбиения на оси $Ox$ и по оси $Oy$. Говорят еще, что $T_2$ \textbf{следует за} $T_1$ и пишут $T_2\supset T_1$ или $T_1\subset T_2$. 
\end{object}
\par В частности, любое неразмеченное разбиение $T$ есть измельчение самого себя. Далее, очевидно, что при измельчении разбиения $T$ нижняя сумма Дарбу $s(T)$ не может уменьшиться, а верхняя сумма Дарбу $S(T)$ не может увеличиться. Поэтому для доказательства утверждения леммы 3 надо на каждой оси$Ox$ и $Oy$ взять разбиение $T_3$, объединяющее $T_1$ и $T_2$. Тогда получим $$s(T_1)\le s(T_3)\le S(T_3)\le S(T_2).$$ Отсюда имеем $s(T_1)\le S(T_2)$, что и доказывает утверждение леммы 3.
\par Отметим также, что утверждение леммы 4 по существу вытекает из леммы 3. Действительно, если образуем числовое множество $M_1$, состоящее изо всех значений величин $s(T)$,и множество $M_2$ значений величин $S(T)$, то утверждение леммы 3 означает, что любой элемент $a\in M_2$ есть верхняя грань множества $M_1$, а потому наименьшая верхняя грань множества $M_1$, т.е. $I_{*}$ не превосходит этого элемента $a\in M_1$. Отсюда для любого числа $a\in M_1$ имеем $I_{*}\le a$. Это значит, что $I_{*}$ является нижней гранью множества $M_2$. Но величина $I*$, по своему определению, есть точная нижняя грань множества $M_2$, и потому для любого разбиения $T\in A_P$ имеем $$s(T)\le I_{*}\le I*\le S(T)$$. Лемма 4 доказана.$\blacktriangleright$
\begin{lemma} 
Для $\forall$ $T$ имеем $\Omega (T)\ge I*-I_{*}$. 
Действительно, из леммы~\ref{lemmfour} получим $$\Omega (T)=S(T)-s(T)\ge I*- s(T)\ge I*-I_{*}.$$      
\end{lemma}
