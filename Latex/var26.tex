\documentclass[12pt,english,russian]{article}
\usepackage[utf8]{inputenc}
\usepackage[russian,english]{babel}
\usepackage{amsmath}
\usepackage{amsfonts}
\usepackage[dvips]{graphics,epsfig}
\frenchspacing
\pagestyle{empty}
\input set.txt
\begin{document}
Рассмотрим следующие задачи: 
\begin{equation}
	\label{integral}	
	\alpha \int_x^\infty \frac{\rmd t}{t\sqrt{1+t^4}} = \int_0^x \frac{\rmd t}{\sqrt{1+t^4}};
\end{equation}


\begin{equation}
		\label{uravneniya_1}
		\begin{array}{l}
			u_t=4u_{xx}-\sin t+\sin x,\quad x\in(0,\pi /2),\\
			u|_{x=0}=\cos t ,\qquad u_x|_{x=\pi /2}=0,\\
			u|_{t=0}=1-\sin{5x};
		\end{array}
	\end{equation}

Решить систему $\dot x=Ax$, $x\in\mathbb{R}^3$, и найти $e^{At}$
	
	\begin{equation}
		A=\begin{pmatrix}
			 7 & -1 & 7\\
		     10 & 0 & 10\\
			 -2 & 1 & -2
		  \end{pmatrix}
	\end{equation}
	
(хар-кий мн-н: $\lambda^3-5\lambda^2$)
	
	\begin{equation}
		\label{uravneniya_3}
		\begin{cases}
			u_{tt}=u_{xx},\quad t>0,\;x>0,\\
			u|_{t=0}=0,\quad {u_t|}_{t=0}=0,\\
			(u_{x}-2u)|_{x=0}=e^{t}.
		\end{cases}
	\end{equation}

Задача (\ref{integral}) состоит в нахождении корня уравнения, при её решении необходимо ознакомиться с пособием \cite{Valedinskiy}.
При решении задач (\ref{uravneniya_1})--(\ref{uravneniya_3}) необходимы знания из курса дифференициальных уравнений.

На четвёртом году обучения в рамках курса ``Численные методы''{} будет подробно рассматриваться проблематика численного решения подобных задач.


	\begin{thebibliography}{1}
		\bibitem{Valedinskiy}
		\emph {Валединский В.Д., Корнев А.А.} Методы программирования в примерах и задачах. М.: Изд-во механико--математического ф--та МГУ, 2000.
	\end{thebibliography}

\vskip 10mm
	\begin{center}
	{\epsfxsize= 90mm \epsfbox{001.eps}}
	\end{center}
	
\vskip 10mm
	\begin{center}
	{\bf $F(x,y) = \sqrt{x^2 + y^2} + 3\cos\sqrt{x^2 + y^2} + 5$}
	{\epsfxsize= 90mm \epsfbox{prim9.eps}}
	\end{center}
	
\vskip 10mm
	\begin{center}
	{\bf Приближение дифференциального уравнения}
	{\epsfxsize= 90mm \epsfbox{gamid.eps}}
	\end{center}
	
\vskip 30 mm
\begin{center}
	\begin{picture}(100,10)(0,0)
    \put(100,0){\vector(0,1){100}}
	\put(0,50){\vector(1,0){200}}
	\put(90,40){0}
	\put(100,75){\line(1,-1){25}}
	\put(100,75){\line(-1,-1){25}}
	\put(100,25){\line(1,1){25}}
	\put(100,25){\line(-1,1){25}}
	\put(125,40){$l$}
	\put(60,40){$-l$}
	\put(92,75){$h$}
	\put(102,22){$-h$}
	\end{picture}
\end{center}

\begin{equation}
z_c = \frac{\iiint_V z\rho(x,y,z)\rmd x \rmd y \rmd z}{\iiint_V \rho(x,y,z)\rmd x \rmd y \rmd z}
\end{equation}

\begin{equation}
\int_Q f(x)\cos{nx}\rmd x = \int_Q (f(x) - T_{n-1}(f,x)_1)\cos{nx}\rmd x + \int_Q T_{n-1}(f,x)_1\cos{nx}\rmd x 
\end{equation}

\begin{equation}
S_n(g,x) = \frac{1}{2\pi} \int_{-\pi}^\pi g(x + t)\frac{\sin{(n+\frac{1}{2}})}{\sin{\frac{t}{2}}} \rmd t
\end{equation}

\end{document}

