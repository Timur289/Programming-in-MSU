\documentclass{beamer}
\usepackage[english,russian]{babel}
\usepackage[utf8]{inputenc}
\usepackage {xcolor}
\usetheme{Warsaw}
\usepackage{graphicx}  % Для вставки рисунков
\graphicspath{{презентация/}{презентация/}}  % папки с картинками
\setlength\fboxsep{3pt} % Отступ рамки \fbox{} от рисунка
\setlength\fboxrule{1pt}
\begin{document}
\title{Все, что мы хотели бы рассказать об Илоне Маске}  
\author{Садигова Аян}
\institute{Московский государственный университет}
\date{Баку, 2018} 
\frame{\titlepage} 

\begin{frame}{Список литературы}
\begin{thebibliography}{10}
\beamertemplatebookbibitems
\bibitem{WikiMask}
{\sc Wikipedia}, {\em Илон Маск}.
\bibitem{LukMask}
{\sc Lukmore}, {\em Илон Маск}.
\end{thebibliography}
\centering\includegraphics[width=0.3\paperwidth]{elon.jpg}\par
\end{frame}

\begin{frame}
\frametitle{Самые известные достижения}
\framesubtitle{И вообще, как он крут...}
\begin{itemize}
 \item oснователь компании PayPal;
 \item основатель, совладелец, генеральный директор и главный инженер компании SpaceX;
 \item председатель совета директоров;
 \item генеральный директор и главный идейный вдохновитель компании Tesla;
 \item член совета директоров компании SolarCity, основанной его двоюродными братьями.
\end{itemize}
\centering\includegraphics[width=0.3\paperwidth]{space.jpg}\par
\end{frame}

\begin{frame}
\frametitle{А знаете ли вы, чтo ...}
\framesubtitle{Сведения об Илоне}
\begin{itemize}
 \item Илон Маск временами ночует на заводе Tesla.
 \item Маск по собственной воле получал доллар в день.
 \item Маск - фанат комиксов про <<Людей Икс>>, что делает его еще круче, чем он уже есть.
 \item Илон Маск работал уборщиком.
 \item Маск был награжден золотой космической медалью.
\end{itemize}
\centering\includegraphics[width=0.3\paperwidth]{elon1.jpg}\par
\end{frame}

\begin{frame}{Высказывания Маска}
\begin{block}{Про неудачи}
 "Неудачи есть всегда. Если у вас нет неудач, значит вы недостаточно инновационны."
\end{block}
\begin{block}{Про Марс}
 "Я бы хотел умереть на Марсе. Только не от удара о поверхность."
\end{block}
\begin{block}{С чем будет согласен Илья Сергеевич}
 "Главное — задать правильный вопрос. Все дело в вопросах."
\end{block}
\centering\includegraphics[width=0.3\paperwidth]{ilon.jpg}\par
\end{frame}
\begin{frame}
\frametitle{Его компании}
\begin{itemize}
	\item Tesla. Американская компания, производитель электромобилей и (через свой филиал SolarCity) решений для хранения электрической энергии.
	\item SolarCity. Американская энергетическая компания, расположенная в Сан-Матео, Калифорния, дочернее предприятие Tesla. 
\end{itemize}
\centering\includegraphics[width=0.3\paperwidth]{images.jpg}\par
\end{frame}

\begin{frame}
\frametitle{Его компании}
\begin{itemize}
	\item OpenAI. Некоммерческая исследовательская компания из Сан-Франциско, занимающаяся искусственным интеллектом. 
	Цель компании — развивать открытый, дружественный ИИ. Одним из основателей является Илон Маск. 
	\item Neuralink. Американская нейротехнологическая компания, основанная Илоном Маском, 
	планирующая заниматься разработкой и производством имплантируемых нейрокомпьютерных интерфейсов. 
	\item The Boring Company. Строительная компания, занимающаяся инфраструктурой и прокладкой тоннелей, основана Илоном Маском в 2016 году. 
\end{itemize}
\end{frame}
\begin{frame}
\frametitle{Так вот...}
\centering Мы могли бы описывать еще очень долго, как он крут, но хотим оставить хоть что-то
для вашего личного ознакомления!
\bigskip
 P.S. Илья Сергеевич, поставьте плюс, пожалуста :с
\end{frame}
\end{document}
