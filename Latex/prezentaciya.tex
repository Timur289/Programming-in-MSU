\documentclass{beamer}
\usepackage[english,russian]{babel}
\usepackage[utf8]{inputenc}
\usepackage {xcolor}
\usetheme{Warsaw}
\usepackage{graphicx}  % Для вставки рисунков
\graphicspath{{мояпрезентация/}{мояпрезентация/}}  % папки с картинками
\setlength\fboxsep{3pt} % Отступ рамки \fbox{} от рисунка
\setlength\fboxrule{1pt}
\begin{document}
\title{Тайный круг Стоунхенджа и алмазы}  
\author{Рамалданова Рабият и Исбатов Рустам}
\institute{Московский Государственный Университет}
\date{Баку, 2018} 
\frame{\titlepage} 

\begin{frame}{Список литературы}
\begin{thebibliography}{10}
\beamertemplatebookbibitems
\bibitem{WikiMask}
{\sc Wikipedia}, {\em Стоунхендж}.
\end{thebibliography}
\centering\includegraphics[width=0.3\paperwidth]{stounhedge.jpg}\par
\end{frame}

\begin{frame}{Стоунхендж}
Стоунхендж (англ. Stonehenge) — внесённое в список Всемирного наследия каменное мегалитическое сооружение (кромлех) в графстве Уилтшир (Англия). 
Находится примерно в 130 км к юго-западу от Лондона, примерно в 3,2 км к западу от Эймсбери и в 13 км к северу от Солсбери. 
\centering\includegraphics[width=0.3\paperwidth]{qw.jpg}\par
\end{frame}

\begin{frame}
\frametitle{Датировка Стоунхенджа}
%\framesubtitle{Три этапа строительства}
Первые исследователи связывали постройку Стоунхенджа с друидами. Раскопки, однако, отодвинули время создания Стоунхенджа к новокаменному и бронзовому векам.
Современная датировка элементов Стоунхенджа основана на радиоуглеродном методе.\par

\centering\includegraphics[width=0.3\paperwidth]{stoun.jpg}\par

\end{frame}
\begin{frame}
\frametitle{Фазы строительства}
\begin{itemize}
\item{Три фазы строительства:}
\begin{itemize}
 \item Фаза 1 — строительство главного рва и валов (3100 г. до н. э.).
 \item Фаза 2 — около 3000 г. до н. э. — установлены первые каменные колоссы.
 \item Фаза 3 — основной — создание первого круга камней (внешнего) с горизонтальными плитами на них, «ворот» — трилитов вокруг центра (2440—2100 до н. э.)
\end{itemize}
\end{itemize}
\end{frame}
\begin{frame}{Назначение Стоунхенджа}
В 1995 году британский астроном Данкан Стил выдвинул теорию, согласно которой Стоунхендж изначально служил для предсказания космических катастроф.
Легенды связывали постройку Стоунхенджа с именем Мерлина.
В начале XIX века утвердилась версия о Стоунхендже, как о святилище друидов. Некоторые считали, что это гробница Боадицеи — языческой королевы.
Ещё авторы XVIII века подметили, что положение камней можно увязать с астрономическими явлениями.
\centering\includegraphics[width=0.3\paperwidth]{stounhedge.jpg}\par
\end{frame}
\begin{frame}
\frametitle{Факты о Стоунхендже:}
\begin{itemize}
	\item По расчетам ученых, древние люди потратили на строительство камней 2 миллиона часов, 
	на их обработку – 20 миллионов часов, а для возведения всей постройки – около 20 столетий;
	\item В фундаменте Стоунхенджа археологами были найдены монеты, которыми расплачивались римляне в 7 веке до нашей эры;
\end{itemize}
\centering\includegraphics[width=0.3\paperwidth]{stoun.jpg}\par
\end{frame}

\begin{frame}
\frametitle{Факты о Стоунхендже}
\begin{itemize}
	\item В 1615 году архитектор Инго Джонс сообщил, что этот монумент построил дьявол и римляне в честь Кнелуса – языческого божества;
	\item Существует версия, что Стоунхендж был местом сжигания человеческих останков.
	 Историки аргументируют данную версию тем, что культура периода связывала с камнем смерть, с деревом – жизнь.
\end{itemize}
\end{frame}

\begin{frame}
\frametitle{Легенды}
\begin{itemize}
\item Как гласит древнее предание, до постройки Стоунхенджа король Артур вёл грандиозную по величине битву. 
На ней погибли более 300 достойных воинов, и впоследствии этого король захотел построить на этом месте монумент, как память о погибших воинах. 
Огромные камни переправил из Ирландии в Англию кельтский колдун Мерлин.
\item Есть версия, согласно которой монумент был возведен древними жрецами для вычисления траектории движения кометы, которая должна была уничтожить планету Земля.
 Как вывод, Стоунхендж является местом падения огромного метеорита.
 \end{itemize}
 \end{frame}
 
 \begin{frame}
 \frametitle{Алмазы}
 \centering\includegraphics[width=0.3\paperwidth]{almazi-e1430939453747.jpg}\par
Алмаз — минерал, кубическая аллотропная форма углерода. При нормальных условиях метастабилен, то есть может существовать неограниченно долго.
Самый твёрдый по шкале эталонных минералов твёрдости Мооса. 
 \end{frame}
 
 \begin{frame}
 Алмаз — редкий, но вместе с тем довольно широко распространённый минерал. Промышленные месторождения алмазов известны на всех континентах, кроме Антарктиды. Известно несколько видов месторождений алмазов. Уже несколько тысяч лет назад алмазы в промышленных масштабах добывались из россыпных месторождений. Только к концу XIX века, когда впервые были открыты алмазоносные кимберлитовые трубки, стало ясно, что алмазы не образуются в речных отложениях.\par
 \centering\includegraphics[width=0.3\paperwidth]{185px-Diamant_tropfen.jpg}\par
 \end{frame}
 
 \begin{frame}
 Известны метеоритные алмазы внеземного, возможно, досолнечного происхождения. Алмазы также образуются при ударном метаморфизме при падении крупных метеоритов, например, в Попигайской астроблеме на севере Сибири.
Кроме этого, алмазы были найдены в кровлевых породах в ассоциациях метаморфизма сверхвысоких давлений, например в Кумдыкульском месторождении алмазов на Кокчетавском массиве в Казахстане.\par
 \centering\includegraphics[width=0.3\paperwidth]{kaidun1c.jpg}\par
 \end{frame}
 
 \begin{frame}
 \frametitle{Типы алмазов}
 \begin{itemize}
 \item Тип I – наиболее распространенный в природе тип алмазов. В структуре кристаллической решетки таких алмазов присутствуют атомы азота.
 \begin{itemize}
 \item Тип Ia – алмазы, в кристаллической решетке которых имеются кластеры из 2-3 молекул азота. К этому типу относится примерно 98 \% продаваемых в мире алмазов, которые ценятся за свою красоту. Большинство из них обладает желтоватым оттенком.
 \item Тип Ib – алмазы, в решетку которых попало всего несколько атомов азота. Такие камни – чрезвычайная редкость. К этому типу относится только 0,1 \% всех добываемых на планете алмазов. Их цвет чаще всего ярко-канареечный, т.е. уже фантазийный. Встречаются также оранжевые, коричневые и даже зеленые экземпляры. К типу Ib относится большинство синтетических алмазов, создаваемых в условиях высоких температур под высоким давлением.
 \end{itemize}
 \end{itemize}
 \end{frame}
 
 \begin{frame}
 \begin{itemize}
 \item Тип II – алмазы, которые совсем или почти совсем не содержат азота. Такие алмазы встречаются гораздо реже, чем алмазы типа I.
 \begin{itemize}
 \item Тип IIa – «чистые», «стопроцентные» алмазы без атомов азота в кристаллической решетке. Среди всех алмазов мира таких примерно 1-2 \%. В большинстве своем алмазы типа IIa лишены всякого цвета, и им присваивается группа D. Если же молекулярная структура таких алмазов в процессе своего формирования подверглась деформации, они приобретают коричневый, розовый или пурпурный цвет.
 \item Тип IIb – алмазы, внутрь которых «попал» бор. Они обладают электропроводностью и полупроводниковыми свойствами. Чаще всего их цвет – голубой или серый, однако бывают и почти бесцветные алмазы типа IIa. Эти камни чрезвычайно редки. Во всем мире их не более 0,1 \%.
 \end{itemize}
 \end{itemize}
 \end{frame}
 
 \begin{frame}
 Знаменитый алмаз звезда Сьерра-Леоне\par
 \centering\includegraphics[width=0.3\paperwidth]{pic_1358815297.jpg}\par
 
 \end{frame}
\begin{frame}
\frametitle{Конец!}
\centering \Large Спасибо за внимание!
\end{frame}
\end{document}
